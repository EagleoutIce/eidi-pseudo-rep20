%% Basic Packages
% center stuff and maybe add bibliography? :D
\RequirePackage[T1]{fontenc}
\RequirePackage[utf8]{inputenc}
\RequirePackage[defaultfam,light,tabular,lining]{montserrat}
\RequirePackage{sfmath,tikz,qrcode} % we need it anyways - will load xcolor

\usetikzlibrary{calc, arrows.meta, decorations, decorations.pathreplacing, graphs}
\RequirePackage[english,main=ngerman]{babel}

\RequirePackage{microtype,csquotes}
\RequirePackage[noguardspace]{sopra-listings}
\RequirePackage{sopra-models,sopra-attachments,array,booktabs,colortbl}
\solLoadLanguage{bash}

\RequirePackage[biber,style=numeric-comp]{lecture-bibliography}

%% Theme
\usepackage[emblem,theorems]{beamerthemecolorful-dream}
\colorlet{@primary}{btcd@color@primary}
\colorlet{@secondary}{btcd@color@secondary}
\colorlet{@alerted}{btcd@color@alerted}
\colorlet{@alternative}{btcd@color@alternative}
\colorlet{@black}{btcd@color@black}
\colorlet{@white}{btcd@color@white}

\makeatletter
\lstcolorlet{keywordA}{@primary}
\lstcolorlet{keywordB}{@secondary}
\lstcolorlet{keywordC}{@secondary}
\lstcolorlet{comments}{@primary!20!@white!50!@black}
\lstcolorlet{numbers}{@alerted!90!@primary!90!@black}
\lstcolorlet{literals}{@alternative}

\sol@list@define@styles{%
  {keywordA: \@declaredcolor{sol@colors@lst@keywordA}\bfseries},%
  {keywordB: \@declaredcolor{sol@colors@lst@keywordB}},%
  {keywordC: \@declaredcolor{sol@colors@lst@keywordC}},%
}

%% Extra Defs
% Taken from lilly code extensions
\newenvironment{lillyHighlight}[1][]
{\begingroup\tikzset{lilly@Highlight@par/.style={#1}}\begin{lrbox}{\@tempboxa}}
{\end{lrbox}\lilly@lsthighlight@box[lilly@Highlight@par]{\@tempboxa}\endgroup}

\newcommand*\@lillyLstHL[1][]{%
  \begin{lillyHighlight}[#1]\bgroup\aftergroup\lilly@lsthighlight@endenv%
}
\def\lilly@lsthighlight@endenv{\end{lillyHighlight}\egroup}

\newcommand*\lilly@lsthighlight@box[2][]{%
  \tikz[#1]{%
    \pgfpathrectangle{\pgfpoint\p@\z@}{\pgfpoint{\wd #2}{\ht #2}}%
    \pgfusepath{use as bounding box}%
    \node[anchor=base west, outer sep=\z@,inner xsep=\p@, inner ysep=-1.105\p@, rounded corners=1.65\p@, minimum height=\ht\strutbox+\p@,#1]{\raisebox\p@{\strut}\strut\usebox{#2}};
  }%
}

\def\lstHLZWS{\@lillyLstHL[fill=@primary!21,draw=@primary!21]}
\def\lstHLVGL{\@lillyLstHL[fill=@alerted!36!@primary!21,draw=@alerted!36!@primary!21]}
\def\lstHLINK{\@lillyLstHL[fill=@alternative!60!@primary!21,draw=@alternative!60!@primary!21]}
\def\lstHLDEK{\@lillyLstHL[fill=@secondary!60!@primary!21,draw=@secondary!60!@primary!21]}

\lstdefinelanguage{xJava}{
    language=lJava,
    moredelim=**[is][{\lstHLZWS}]{|zws|}{|zws|},%
    moredelim=**[is][{\lstHLVGL}]{|vgl|}{|vgl|},%
    moredelim=**[is][{\lstHLINK}]{|inc|}{|inc|},%
    moredelim=**[is][{\lstHLDEK}]{|dec|}{|dec|}%
}

\def\linksort#1{\hyperlink{mrk:sort-#1}{#1}}
\def\alglabel#1{\hypertarget<1>{mrk:sort-#1}{}}
\def\<{\ensuremath{\langle}}\def\>{\ensuremath{\rangle}}
\def\mto{\ensuremath{\to}}\def\mprec{\ensuremath{\prec}}

%% General
\usepackage{lecture-algorithms,lilly-dir,xfrac}
\SetFolderFileSameIndent

\setcounter{tocdepth}{4}
\newcounter{ctrtasks}
\def\Task#1{\only<1>{\stepcounter{ctrtasks}\pdfbookmark[4]{Aufgabe \thectrtasks: #1}{id@bkmrk@e@\thectrtasks}}}
\def\Time#1{\ifx!#1!\else\null\hfill(#1\;\ifnum#1=1\relax Minute\else Minuten\fi)\fi}

\colorlet{solidF}{@primary}
\colorlet{fillA}{@primary!15!white}
\colorlet{fillB}{@secondary!15!white}
\colorlet{fillC}{@alerted!15!white}

\long\def\hl#1{\textcolor{@secondary}{#1}}
\long\def\imp#1{{\sbfamily#1}}
\let\say\enquote
\def\widei{\itemsep1em}
\newlength\@bdescdyn \@bdescdyn=3.5\p@ \@plus\z@ \@minus2\p@
\appto\@@description{\itemsep=\@bdescdyn}

\newcommand\twosplit[3][t]{%
\begin{columns}[#1]
\begin{column}{.475\linewidth}#2\end{column}\hfill
\begin{column}{.475\linewidth}#3\end{column}
\end{columns}}

\def\O{\ensuremath{\mathcal{O}}}
\let\emptyset\varnothing

\AtBeginDocument{%
    \tikzumlset{fill class=@white,font=\small\ttfamily,fill object=@white}%
    \attachfilesetup{color=@primary,author={Florian Sihler}}%
}

\usepackage[chess]{eagle-maps}
\usepackage[glows,manual-layers]{tikzpingus}
\def\sbfamily{\fontseries{sb}\selectfont}

%% TikZ
\tikzset{
    block/.style={rectangle,draw,#1,font=\@declaredcolor{@black},rounded corners=\z@,minimum height=\baselineskip,execute at begin node={\strut},rounded corners=\p@},
    block/.default=solidF,
    iblock/.style={block=@primary,fill=@primary!21!@white},
    ball/.style={circle,draw,#1,font=\@declaredcolor{@black}},
    iball/.style={ball=@primary,fill=@primary!21!@white},
    oball/.style={ball=@secondary,fill=@secondary!21!@white},
    ball/.default=solidF, link/.style={-Kite},
    enode/.style={inner sep=\z@},%
    every picture/.append style={line join=round,line cap=round}%
}


\def\mtx@blk{X}
\def\mtx@k{6mm}
% calculate the number of steps required to stay visible
\mode<presentation>{
\def\mtx@split@gol#1|handout:#2\@nil{\the\numexpr#1-\beamer@slideinframe\relax}
\def\mtx@split@tar#1|handout:#2\@nil{}
}\mode<handout>{% TODO: for handout, subtract handout trigger ids
\def\mtx@split@gol#1|handout:#2\@nil{%
    \ifnum\beamer@slideinframe=1
    \the\numexpr#1-\triga\relax
    \else\ifnum\beamer@slideinframe=2
    \the\numexpr#1-\trigb\relax
    \else\ifnum\beamer@slideinframe=3
    \the\numexpr#1-\trigc\relax
    \else\ifnum\beamer@slideinframe=4
    \the\numexpr#1-\trigd\relax
    \else\ifnum\beamer@slideinframe=5
    \the\numexpr#1-\trige\relax
    \ifnum\beamer@slideinframe=6
    \the\numexpr#1-\trigf\relax
    \fi\fi\fi\fi\fi\fi
}
\let\mtx@split@tar\mtx@split@gol
}
\tikzset{MTX@BLK/.style={rectangle,minimum size=\mtx@k,inner sep=\z@,outer sep=\z@,fill=#1,draw=@primary!90!white,line width=.66\p@}}
\def\Matrix#1#2#3#4#5#6#7#8#9{%
\def\triga{#4}\def\trigb{#5}\def\trigc{#6}\def\trigd{#7}\def\trige{#8}\def\trigf{#9}
\scope[scale=1.4,every node/.append style={transform shape}]
\scope
\clip[rounded corners=\p@] (.5*\mtx@k, -.5*\mtx@k) rectangle ++(#2*\mtx@k, -#3*\mtx@k);
\foreach[count=\y] \line in {#1} {
    \foreach[count=\x] \tar/\gol/\ded in \line {
        \ifx\tar\mtx@blk
            \node[MTX@BLK=gray!38!white] (\x-\y) at (\x*\mtx@k, -\y*\mtx@k) {};
        \else
            \edef\mtx@tmp{\gol}%
            % overdraw me baby
            \node[MTX@BLK=white,text=lightgray] (\x-\y) at (\x*\mtx@k, -\y*\mtx@k) {\footnotesize\expandafter\mtx@split@tar\mtx@tmp\@nil};
            \only<\tar>{
                \node[MTX@BLK=@alternative!30!white,text=gray] (\x-\y) at (\x*\mtx@k, -\y*\mtx@k) {\footnotesize\expandafter\mtx@split@gol\mtx@tmp\@nil};
            }
            \only<\gol>{
                \node[MTX@BLK=@secondary!40!white] (\x-\y) at (\x*\mtx@k, -\y*\mtx@k) {};
            }
            \only<\ded>{
                \node[MTX@BLK=gray!20!white] (\x-\y) at (\x*\mtx@k, -\y*\mtx@k) {\T{\color{gray!80!white}\textbullet}};
            }
        \fi\
    }
}
\endscope
\draw[rounded corners=\p@,@primary!90!white,line width=1.65\p@] (.5*\mtx@k, -.5*\mtx@k) rectangle ++(#2*\mtx@k, -#3*\mtx@k);
\endscope}

%% Heap and Stack
\usepackage{lecture-heap-n-stack}
\let\lhnselementformat\texttt
\def\lhnsielementformat#1{\textbf{\vphantom{A}#1}}
\let\lhnselementbox\pbox
\lhns@minborderheight=4.75cm
\lhns@borderbuff=1.33em
\tikzset{lhns@impstyle/.style={}}%

%% Different Versions
\usepackage{lithie-profiles}
\RegisterProfile{short}{}

\newif\ifull
\ifprofile{default}{%
  \def\profileshorttitle{Volle Version}\ulltrue
}{\ifprofile{short}{%
  \def\profileshorttitle{Kurze Version}\ullfalse
}{}}
% => empty übungsaufgaben (gray + no link)
\ifull\let\fullsubsection\subsection
\else\let\fullsubsection\fakesubsection\fi
\makeatother

\addbibresource{data/references.bib}
\nocite{gh:tikzpingus, gh:recursion, gh:heaps, gh:traversals}