\section{Theoretische Grundlagen}
\subsection{Der Algorithmusbegriff}
\begin{frame}{Was ist ein Algorithmus?}
    \pause{}
    \begin{definition}[Algorithmus]
        \pause{}Ein \emph{Algorithmus} ist eine \emph{eindeutige Handlungsvorschrift} zur Lösung eines Problems.
    \end{definition}
    \begin{itemize}[<+(1)->]
        \widei
        \item Es existieren verschiedene Darstellungsformen: \begin{itemize}
            \item Textuell \pause(1. Tue dies, 2. Tue das, 3. Falls \(X\)\ldots)
            \item Grafisch \pause(Instruktionsabläufe, Graphen, Grafiken, \ldots)
            \item Pseudocode
        \end{itemize}
        \item Klassische Alltagsbeispiele:\pause{} (Koch-)Rezepte,\pause{} Gebrauchsanleitungen,\pause{} \ldots
    \end{itemize}
\end{frame}

\subsection{Algorithmen analysieren}
\begin{frame}{Algorithmuseigenschaften}
    \begin{itemize}[<+(1)->]
        \widei
        \item Zu Beginn gilt es einige Begriffe zu klären: \begin{description}[Elementaroperation]
            \item[Prozess] Die Ausführung der Schritte eines Algorithmus
            \item[Prozessor] Der Ausführende (Mensch, Computer, \ldots)
            \item[Elementaroperation] Eine einzelne, eindeutige Handlung.
        \end{description}
        \item Ein Algorithmus besitzt einige Eigenschaften: \begin{description}[Ausführbarkeit]
            \item[Ausführbarkeit] Die Anleitung muss ausführ- und reproduzierbar sein.
            \item[Endlichkeit] Der gesamte Algorithmus muss mit endlich viel Text beschrieben werden können.
            \item[Termination] Der Algorithmus muss nach endlich vielen Schritten zum Ende kommen.
            \item[Verarbeitung] Ein Algorithmus erhält Eingabeobjekte \(E\),\pause{} hält lokale Daten \(D\)\pause{} und erzeugt Ausgabedaten \(A\).
        \end{description}
    \end{itemize}
\end{frame}

\begin{frame}[c]{Eigenschaften für die Analyse}
    \twosplit{%
    \begin{block}<2->{Korrektheit}
        \onslide<3->{Ein Algorithmus ist korrekt, wenn er für jede (definierte) Eingabe, die korrekte Ausgabe erzeugt.}
    \end{block}\bigskip
    \begin{block}<6->{Partielle Korrektheit}
        \onslide<7->{Wenn der Algorithmus terminiert, ist er korrekt.}
    \end{block}
    }{%
    \begin{block}<4->{Robust}
        \onslide<5->{Ein Algorithmus ist (maximal) robust, wenn er falsche Eingabe erkennen und abfangen kann.}
    \end{block}\bigskip
    \begin{block}<8->{Totale Korrektheit}
        \onslide<9->{Der Algorithmus ist partiell korrekt und terminiert.}
    \end{block}
    }
\end{frame}

\begin{frame}{Eigenschaften für die Analyse}
    \twosplit{%
    \begin{block}<2->{Determinismus}
        \onslide<3->{Der Ablauf des Algorithmus ist \emph{eindeutig}. Für jede Anweisung folgt bei gleichen Vorraussetzungen dieselbe.}
    \end{block}
    }{%
    \begin{block}<4->{Determiniertheit}
        \onslide<5->{Die selben Eingabedaten erzeugen immer die selben Ausgabedaten.}
    \end{block}
    }\bigskip
    \begin{itemize}
        \item<6-> Die Begriffe sind zu unterscheiden!
        \item<7-> So können bei einem Algorithmus verschiedene Wege zum selben Ziel führen.
        \item<8-> (Sinnfreies) Beispiel: \onslide<9->{Der Algorithmus kann jedes Element eines Arrays quadrieren. \onslide<10->{Die Reihenfolge wählt er zufällig aus!}}
        \item<11-> Dieser Algorithmus \emph{determiniert}, \onslide<12->{ist aber \emph{nicht-deterministisch}!}
    \end{itemize}
\end{frame}


\begin{frame}{Schema eines Algorithmus}
    \onslide<2->{
    \begin{figure}
        \begin{tikzpicture}
            \node[block] (A) at(0,0) {Algorithmus};
            \node (V) at(-5,0) {Vorbedingung};
            \node (N) at(5,0) {Nachbedingung};
            \draw[link] (V) edge (A) (A) edge (N);
        \end{tikzpicture}
    \end{figure}}
    \begin{itemize}[<+(1)->]
        \widei
        \item Ein Algorithmus benötigt (formal) eine: \begin{description}[iV]
            \item[V] Vorbedingung (wie \say{positive ganze Zahl})
            \item[N] Nachbedingung (wie \say{negative Fließkommazahl})
        \end{description}
        \item Algorithmen fundieren auf einem \emph{Problem} der realen Welt und werden durch Abstraktion formuliert.
        \item (Java-)Programme sind eine Darstellung von Algorithmen,\pause{} die von Maschinen interpretiert und realisiert werden.
     \end{itemize}
\end{frame}

\begin{frame}{Schema eines Algorithmus}
    \onslide<2->{
    \begin{figure}
\begin{tikzpicture}[
    every node/.style={transform shape},block/.append style={minimum width={3.5cm}},
    scale=0.75]
    \node[block] at(0,0) (PM)  {Problem};
    \node[block] at(0,-3) (AL)  {Algorithmus};
    \node[block] at(0,-5) (PR)  {Programm};
    \node[block] at(0,-7) (MA)  {Maschine};
    \draw[link] (PM) -- (AL);
    \draw[link] (AL) -- (PR);
    \draw[link] (PR) -- (MA);
    \node[left=0.5cm] at (0,-1.5) {Abstraktion};
    \node[left=0.5cm] at (0,-4) {Implementierung};
    \node[left=0.5cm] at (0,-6) {Realisierung};

    \node[right,align=left,text width=5cm] at (0.5,-1.45) {\begin{itemize}
        \item Problemanalyse
        \item Funktionale\\ Spezifikation
        \item Algorithmenentwurf
    \end{itemize}
    };

    \node[right,align=left,text width=5cm] at (0.5,-3.85) {\begin{itemize}
             \item Darstellung in\\Programmiersprache
         \end{itemize}
    };

    \node[right,align=left,text width=5cm] at (0.5,-5.85) {\begin{itemize}
            \item Umsetzung in\\ Maschinenbefehle
        \end{itemize}
    };
    % \node[right,align=left] at (0,-4) {\begin{itemize}
    %         \item Darstellung in\\Programmiersprache
    %     \end{itemize}
    %     };
    % \node[right,align=left] at (0,-6) {\begin{itemize}
    %         \item Umsetzung in\\ Maschinenbefehle
    %     \end{itemize}
    %     };
\end{tikzpicture}
    \end{figure}}
\end{frame}

\subsection{Konzeptionalisierung}

\begin{frame}{Wie ein Programm entsteht}
    \begin{itemize}[<+(1)->]
        \widei
        \item Ein Programm selbst fällt nicht einfach aus dem Wolken (leider).
        \item Grundlegend gliedert man (im Softwareengineering) die Konstruktion in sechs Phasen.
        \item Die sechste Phase (Dokumentation) unterscheidet sich in sofern,\pause{} als dass sie Parallel zu den anderen stets aktiv ist.
        \item Die Dokumentation bezeichnet nicht nur die exakte Beschreibung aller für die Umsetzung notwendiger Komponenten,\pause{} sondern schließt auch Beschreibungen zum Code, zu verschiedenen Lösungsansätzen,\pause{} sowie dem Ablauf selbst mit ein!
    \end{itemize}
\end{frame}

\begin{frame}[c]{Wie ein Programm entsteht}
    \begin{center}
        \resizebox{0.95\linewidth}{!}{
        \begin{tikzpicture}
            \onslide<2->{\node[iblock,minimum width=3cm] (init) at (-5.5,0) {Problem};}
            \foreach[count=\i,remember=\i as \li (initially init)] \l in {Analyse, Entwurf, Implementierung, Test \& Revision} {
                \pgfmathsetmacro\ovlskip{int(\i+2)};
                \onslide<\ovlskip->{
                \node[block,minimum width=5cm] (\i) at(0,-1.6*\i+1.6) {\l};
                \draw[link] (\li) -- (\i);}
            }
            \onslide<3->{
                \node[above right=0.75cm,yshift=-0.5cm,align=left,font=\scriptsize,text width=4.25cm] (d1) at (1.north east) {Untersuche Problem, in der Regel in Zusammenarbeit mit Nicht-Informatikern.};
                \draw[densely dashed] (d1.west) -- (1.east);
            }
            \onslide<4->{
                \node[above right=0.75cm,yshift=-1cm,align=left,font=\scriptsize,text width=4.25cm] (d2) at (2.north east) {Entwickle Algorithmus, kreativer Prozess (beantwortet auch Fragen wie: \say{gibt es das schon?})};
                \draw[densely dashed] (d2.west) -- (2.east);
            }
            \onslide<5->{
                \node[right=0.75cm,align=left,font=\scriptsize,text width=4.25cm] (d3) at (3.north east) {Übersetze Algorithmus in Programmiersprache};
                \draw[densely dashed] (d3.west) -- (3.east);
            }
            \onslide<6->{
                \node[left=0.75cm,yshift=0.25cm,align=right,font=\scriptsize,text width=4.25cm] (d4) at (4.north west) {Prüft das Programm auf (logische und technische) Fehler.\\[0.25em] Diese Phase kann im Allgemeinen nur die Anwesenheit und nicht die Abwesenheit von Fehlern aufzeigen.};
                \draw[densely dashed] (d4.east) -- (4.west);
            }
            \onslide<7->{
                \node[iblock,minimum width=3cm] (exit) at (5.5,-3*1.6) {Programm};
            \draw[link] (4) -- (exit);
            }\onslide<1->{}
        \end{tikzpicture}}
    \end{center}
\end{frame}

\fullsubsection{Übungsaufgaben}
\ifull
\begin{frame}[c]{Übungsaufgabe}
    \Task{Algorithmus beschreiben}
    \begin{exercise}<2->[Algorithmus beschreiben \Time{2}]
        Beschreibe einen Algorithmus, der das maximale Element einer Datenmenge bestimmt.\pause{}
        \textit{Hinweis: Auf den Elementen muss keine numerische Ordnung definiert sein.}
    \end{exercise}
\end{frame}

\begin{frame}[c]{Lösung}
    \begin{solve}<2->[Algorithmus beschreiben]
        \pause{}Wir betrachten den textuellen Ansatz: \begin{enumerate}[<+(1)->]
            \item Keine Eingabedaten?\pause{} Fertig.
            \item Setze Maximum auf erstes Element der Eingabedaten.
            \item Tue für jedes weitere Element der Eingabedaten: \begin{enumerate}
                \item Ist das zu betrachtende Element \say{größer} als das bisherige Maximum? \begin{description}
                    \item[Ja] ersetze Maximum durch dieses Element
                    \item[Nein] tue nichts
                \end{description}
            \end{enumerate}
        \end{enumerate}
    \end{solve}
\end{frame}

\begin{frame}[c,fragile]{Übungsaufgabe}
    \Task{Charakteristika eines Algorithmus}
    \begin{exercise}<2->[Charakteristika \Time{2}]
        Ist der folgende Algorithmus (total) korrekt?\pause{}
        \begin{plainjava}
public static int simpleMod(int a, int b){
    while(a >= b)
        a -= b;
    return a;
}
        \end{plainjava}
    \end{exercise}
\end{frame}

\begin{frame}[c]{Lösung}
    \begin{solve}<2->[Charakteristika]
       \pause{}Der Algorithmus ist nicht total korrekt.\pause{} So liefert er für ein negatives \T{b} kein korrektes Ergebnis,\pause{} da hier \T{a -= b} zu einem Überlauf führt.\pause{}\par{}
       Ferner ist er auch nicht partiell korrekt.\pause{} So liefert er für ein negatives \T{a} bei einem positiven \T{b} einfach nur \T{a} zurück.
    \end{solve}
\end{frame}

\begin{frame}[c]{Übungsaufgabe}
    \Task{Determinismus und Determiniertheit}
    \begin{exercise}<2->[Deterministische Determiniertheit \Time{2}]
        \pause{}Nenne je ein Beispiel für einen Algorithmus der:
        \begin{enumerate}[<+(1)->]
            \item[i)] nicht-deterministisch ist und determiniert.
            \item[ii)] sowohl deterministisch ist, als auch determiniert.
        \end{enumerate}
    \end{exercise}
\end{frame}

\begin{frame}[c]{Lösung}
    \begin{solve}<2->[Deterministische Determiniertheit]
        \pause{}\begin{enumerate}[<+(1)->]
            \item[i)] Suche, ob ein Element in den Eingabedaten vorkommt.\pause{} Die Elemente werden in einer zufälligen Reihenfolge verglichen.
            \item[ii)] Ein Algorithmus der eine Zahl verdoppelt, in dem er sie mit \(2\) multipliziert.
        \end{enumerate}
    \end{solve}
\end{frame}
\fi
