\section{Theoretische Grundlagen}
\subsection{Der Algorithmusbegriff}
\begin{frame}{Was ist ein Algorithmus?}
    \pause{}
    \begin{definition}[Algorithmus]
        \pause{}Eine \emph{eindeutige Handlungsvorschrift} zur Lösung eines Problems.
    \end{definition}
    \begin{itemize}[<+(1)->]
        \widei
        \item Es existieren verschiedene Darstellungsformen: \begin{itemize}
            \item Textuell \pause(1. Tue dies, 2. Tue das, 3. Falls \(X\)\ldots)
            \item Grafisch \pause(Instruktionsabläufe, Graphen, Grafiken,~\ldots)
            \item Pseudocode
        \end{itemize}
        \item Klassische Alltagsbeispiele:\pause{} (Koch-)Rezepte,\pause{} Gebrauchsanleitungen,\pause{} \ldots
    \end{itemize}
\end{frame}

\subsection{Algorithmen analysieren}
\begin{frame}{Algorithmuseigenschaften}
    \begin{itemize}[<+(1)->]
        \widei
        \item Zu Beginn gilt es einige \emph{Begriffe} zu klären: \begin{description}[Elementaroperation]
            \item[Prozess] Die Ausführung der Schritte eines Algorithmus
            \item[Prozessor] Der Ausführende (Mensch, Computer,~\ldots)
            \item[Elementaroperation] Eine einzelne, eindeutige Handlung.
        \end{description}
        \item Ein Algorithmus besitzt einige \emph{Eigenschaften}: \begin{description}[Ausführbarkeit]
            \item[Ausführbarkeit] Die Anleitung muss ausführ- und reproduzierbar sein.
            \item[Endlichkeit] Der gesamte Algorithmus muss mit endlich viel Text beschrieben werden können.
            \item[Termination] Der Algorithmus muss nach endlich vielen Schritten zum Ende kommen.
            \item[Verarbeitung] Ein Algorithmus erhält Eingabeobjekte \(E\),\pause{} hält lokale Daten \(D\)\pause{} und erzeugt Ausgabedaten \(A\).
        \end{description}
    \end{itemize}
\end{frame}

\begin{frame}[c]{Eigenschaften für die Analyse}
    \twosplit{%
    \begin{block}<2->{Korrektheit}
        \onslide<3->{Ein Algorithmus ist korrekt, wenn er für jede (definierte) Eingabe, die korrekte Ausgabe erzeugt.}
    \end{block}\bigskip
    \begin{block}<6->{Partielle Korrektheit}
        \onslide<7->{Wenn der Algorithmus terminiert, ist er korrekt.}
    \end{block}
    }{%
    \begin{block}<4->{Robust}
        \onslide<5->{Ein Algorithmus ist (maximal) robust, wenn er falsche Eingabe erkennen und abfangen kann.}
    \end{block}\bigskip
    \begin{block}<8->{Totale Korrektheit}
        \onslide<9->{Der Algorithmus ist partiell korrekt und terminiert.}
    \end{block}
    }
\end{frame}

\begin{frame}{Eigenschaften für die Analyse}
    \twosplit{%
    \begin{block}<2->{Determinismus}
        \onslide<3->{Der Ablauf ist \emph{eindeutig}. Jeder Anweisung folgt (bei gleichen Vorraussetzungen) dieselbe.}
    \end{block}
    }{%
    \begin{block}<4->{Determiniertheit}
        \onslide<5->{Die selben Eingabedaten erzeugen immer die selben Ausgabedaten.}
    \end{block}
    }\bigskip
    \begin{itemize}
        \item<6-> Die Begriffe sind zu unterscheiden!
        \item<7-> Es können verschiedene Wege zum selben Ziel führen.
        \item<8-> (Sinnfreies) Beispiel: \onslide<9->{Der Algorithmus kann jedes Element eines Arrays quadrieren. \onslide<10->{Die Reihenfolge wählt er zufällig aus!}}\smallskip\par
        \onslide<11->{\faAngleRight~Dieser Algorithmus \emph{determiniert}, \onslide<12->{ist aber \emph{nicht-deterministisch}!}}
    \end{itemize}
\end{frame}


\begin{frame}{Schema eines Algorithmus}
    \pause\begin{center}
        \begin{tikzpicture}
            \node[block] (A) at(0,0) {Algorithmus};
            \node (V) at(-5,0) {Vorbedingung};
            \node (N) at(5,0) {Nachbedingung};
            \draw[link] (V) edge (A) (A) edge (N);
        \end{tikzpicture}
    \end{center}
    \begin{itemize}[<+(1)->]
        \widei
        \item Ein Algorithmus benötigt (formal) eine: \begin{itemize}
            \item Vorbedingung (wie \say{positive ganze Zahl})
            \item Nachbedingung (wie \say{negative Fließkommazahl})
        \end{itemize}
        \item Algorithmen fundieren auf einem \emph{Problem} der realen Welt und werden durch Abstraktion formuliert.
        \item (Java-)Programme sind eine Darstellung von Algorithmen,\pause{} die von Maschinen interpretiert und realisiert werden.
     \end{itemize}
\end{frame}

\begin{frame}[c]{Schema eines Algorithmus}
    \centering\onslide<2->{%
\scalebox{.75}{\begin{tikzpicture}[yscale=1.1,block/.append style={minimum width={3.5cm}}]
    \node[block] at (0,0) (PM)  {Problem};
    \node[block] at (0,-3) (AL)  {Algorithmus};
    \node[block] at (0,-5) (PR)  {Programm};
    \node[block] at (0,-7) (MA)  {Maschine};
    \draw[link] (PM) -- (AL);
    \draw[link] (AL) -- (PR);
    \draw[link] (PR) -- (MA);
    \node[left=.5cm] at (0,-1.5) {Abstraktion};
    \node[left=.5cm] at (0,-4) {Implementierung};
    \node[left=.5cm] at (0,-6) {Realisierung};

    \node[right,align=left,text width=5cm] at (.5,-1.45) {\begin{itemize}
        \item Problemanalyse
        \item Funktionale\par Spezifikation
        \item Algorithmenentwurf
    \end{itemize}
    };

    \node[right,align=left,text width=5cm] at (.5,-3.85) {\begin{itemize}
             \item Darstellung in\par Programmiersprache
         \end{itemize}
    };

    \node[right,align=left,text width=5cm] at (.5,-5.85) {\begin{itemize}
            \item Umsetzung in\par Maschinenbefehle
        \end{itemize}
    };
\end{tikzpicture}
}}
\end{frame}

\subsection{Konzeptionalisierung}

\begin{frame}{Wie ein Programm entsteht}
    \begin{itemize}[<+(1)->]
        \widei
        \item Ein Programm fällt nicht einfach aus dem Wolken (leider).
        \item Im Softwareengineering gliedert man den Prozess meist in sechs Phasen.
        \item Die sechste Phase (Dokumentation) unterscheidet sich: sie verläuft parallel zu den anderen.
        \item Die Dokumentation bezeichnet nicht nur die exakte Beschreibung aller für die Umsetzung notwendiger Komponenten,\pause{} sondern schließt auch Beschreibungen zum Code (zu verschiedenen Lösungsansätzen),\pause{} sowie dem Ablauf selbst mit ein!
    \end{itemize}
\end{frame}

\begin{frame}[c]{Wie ein Programm entsteht}
\begin{center}
    \resizebox{.95\linewidth}{!}{
    \begin{tikzpicture}
        \onslide<2->{\node[iblock,minimum width=3cm] (init) at (-5.5,0) {Problem};}
        \foreach[count=\i,remember=\i as \li (initially init)] \l in {Analyse, Entwurf, Implementierung, Test \& Revision} {
            \pgfmathsetmacro\ovlskip{int(\i+2)};
            \onslide<\ovlskip->{
            \node[block,minimum width=5cm] (\i) at(0,-1.6*\i+1.6) {\l};
            \draw[link] (\li) -- (\i);}
        }
        \onslide<3->{
            \node[above right=0.75cm,yshift=-.5cm,align=left,font=\scriptsize,text width=4.25cm] (d1) at (1.north east) {Untersuche Problem (in der Regel in Zusammenarbeit mit Nicht-Informatikern).};
            \draw[densely dashed] (d1.west) -- (1.east);
        }
        \onslide<4->{
            \node[above right=0.75cm,yshift=-1cm,align=left,font=\scriptsize,text width=4.25cm] (d2) at (2.north east) {Entwickle Algorithmus. Kreativer Prozess (beantwortet auch Fragen wie: \say{gibt es das schon?})};
            \draw[densely dashed] (d2.west) -- (2.east);
        }
        \onslide<5->{
            \node[right=0.75cm,align=left,font=\scriptsize,text width=4.25cm] (d3) at (3.north east) {Übersetze Algorithmus in Programmiersprache};
            \draw[densely dashed] (d3.west) -- (3.east);
        }
        \onslide<6->{
            \node[left=0.75cm,yshift=.25cm,align=right,font=\scriptsize,text width=4.25cm] (d4) at (4.north west) {Prüft das Programm auf (logische und technische) Fehler.\\[0.25em] Diese Phase kann im Allgemeinen nur die Anwesenheit und nicht die Abwesenheit von Fehlern aufzeigen.};
            \draw[densely dashed] (d4.east) -- (4.west);
        }
        \onslide<7->{
            \node[iblock,minimum width=3cm] (exit) at (5.5,-3*1.6) {Programm};
        \draw[link] (4) -- (exit);
        }\onslide<1->{}
    \end{tikzpicture}}
\end{center}
\end{frame}

% #region Übungsaufgaben
\fullsubsection{Übungsaufgaben}
\ifull
\begin{frame}[c]{Übungsaufgabe}
    \Task{Algorithmus beschreiben}
    \begin{exercise}<2->[Algorithmus beschreiben \Time{2}]
        Beschreiben Sie einen Algorithmus, der das maximale Element einer Datenmenge bestimmt.\pause{}
        \textit{Hinweis: Auf den Elementen muss keine \say{numerische} Ordnung definiert sein.}
    \end{exercise}
\end{frame}

\begin{frame}[c]{Lösung}
    \begin{solve}<2->[Algorithmus beschreiben]
        \pause{}Wir betrachten den textuellen Ansatz: \begin{enumerate}[<+(1)->]
            \item Keine Eingabedaten? \(\to\) Fertig.
            \item Setze Maximum auf erstes Element der Eingabedaten.
            \item Tue für jedes weitere Element der Eingabedaten: \begin{enumerate}
                \item Ist das zu betrachtende Element \say{größer} als das bisherige Maximum? \begin{description}[iNein]
                    \item[Ja] ersetze Maximum durch dieses Element
                    \item[Nein] tue nichts
                \end{description}
            \end{enumerate}
        \end{enumerate}
    \end{solve}
\end{frame}

\begin{frame}[c,fragile]{Übungsaufgabe}
    \Task{Charakteristika eines Algorithmus}
    \begin{exercise}<2->[Charakteristika \Time{2}]
        Ist der folgende Algorithmus für die Berechnung des Modulo \(a \bmod b\) (total) korrekt?\pause{}
        \begin{plainjava}
public static int simpleMod(int a, int b){
    while(a >= b)
        a -= b;
    return a;
}
        \end{plainjava}
    \end{exercise}
\end{frame}

\begin{frame}[c]{Lösung}
    \begin{solve}<2->[Charakteristika]
       \pause{}Der Algorithmus ist nicht total korrekt.\pause{} So liefert er für ein negatives \T{b} kein korrektes Ergebnis,\pause{} da hier \T{a -= b} zu einem Überlauf führt (bis dahin ist stets \bjava{a >= b}).\medskip\par
       \pause{}Ferner ist er auch nicht partiell korrekt.\pause{} So liefert er für ein negatives \T{a} bei einem positiven \T{b} einfach nur \T{a} zurück.
    \end{solve}
\end{frame}

\begin{frame}[c]{Übungsaufgabe}
    \Task{Korrektheitsbeweis}
    \begin{exercise}<2->[Korrektheitsbeweis \Time{4}]
        Zeigen Sie die totale Korrektheit des folgenden Algorithmus:\smallskip\par
\onslide<3->{%
\begin{algorithm}[H]
    \PreCode
        \KwIn{$a, b, c \in \mathbb{N}$}
        \KwOut{Minimum of $a, b, c$}
    \StartCode
        $elems := (a, b, c)$\;
        $min := a$\;\label{alg:ex-min1}
        \ForEach{elem \textbf{in} elems}{\label{alg:ex-fe1}%
            \lIf{$elem < min$}{
                $min := elem$
            }
        }
        \KwRet{min}\;
    \caption{Finden des Minimum}
\end{algorithm}}
    \end{exercise}
\end{frame}

\begin{frame}[c]{Lösung}
    \begin{solve}<2->[Korrektheitsbeweis]
       \pause{}Wir zeigen die Termination und die partielle Korrektheit getrennt: \begin{itemize}[<+(1)->]
           \item Das for-each aus Zeile~\ref{alg:ex-fe1} endet sicher nach drei Durchläufen. Es gibt keine weiteren Sprünge (wie Rekursion,~\ldots): der Algorithmus \emph{terminiert} für alle \(a, b, c \in \mathbb{N}\).
           \item Hier kann man alle Fälle \(a = b = c\), \(a > b = c\),~\ldots einzeln prüfen. Sonst: \begin{itemize}
             \item \(a = \min\{a, b, c\}\) so initialisiert Zeile~\ref{alg:ex-min1} \(min\) schon korrekt, \(b \geq a\) und \(c \geq a\).
             \item \(b = \min\{a, b, c\}\) so überschreibt Zeile~\ref{alg:ex-fe1} \(min\) für \(b\), da \(b \leq a\) und \(c \geq b\).
             \item \(c = \min\{a, b, c\}\) so überschreibt Zeile~\ref{alg:ex-fe1} \(min\) für \(c\), da \(c \leq a\) und \(b \geq c\).
           \end{itemize}
       \end{itemize}
    \end{solve}
\end{frame}

\begin{frame}[c]{Übungsaufgabe}
    \Task{Determinismus und Determiniertheit}
    \begin{exercise}<2->[Deterministische Determiniertheit \Time{2}]
        \pause{}Nennen Sie je ein Beispiel für einen Algorithmus der:
        \begin{enumerate}[<+(1)->]
            \item[i)] nicht-deterministisch ist und determiniert.
            \item[ii)] sowohl deterministisch ist, als auch determiniert.
        \end{enumerate}
    \end{exercise}
\end{frame}

\begin{frame}[c]{Lösung}
    \begin{solve}<2->[Deterministische Determiniertheit]
        \pause{}\begin{enumerate}[<+(1)->]
            \item[i)] Suche, ob ein Element in den Eingabedaten vorkommt.\pause{} Die Elemente werden in einer zufälligen Reihenfolge verglichen.
            \item[ii)] Ein Algorithmus der eine Zahl verdoppelt, in dem er sie mit \(2\) multipliziert.
        \end{enumerate}
    \end{solve}
\end{frame}
\fi
% #endregion