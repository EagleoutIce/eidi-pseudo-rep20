\section{Java-Basics}
\subsection{Wie ein Java-Programm entsteht}
\begin{frame}[c]{Vom Text zum Programm}
\begin{center}
\onslide<2->{
\begin{tikzpicture}[scale=1.1, every node/.style={transform shape}, rec/.style={block,minimum width=3.25cm},irec/.style={iblock,minimum width=3.25cm}]
    \draw (0,0) node[right,irec] (prg) {Programmtext};
    \draw (2,-1) node[right,rec] (bib) {Bibliotheken};
    \draw (2,-2) node[right,rec] (adMac) {Maschinencode};
    \draw (6,-1.5) node[right,rec] (link) {Linker};
    \begin{scope}[every path/.style={link,rounded corners=1.25pt}]
        \draw (prg) -- ++(2,0) node[right,rec] (comp) {Compiler};
        \draw (comp) -- ++(2,0) node[right, rec] (mac) {Maschinencode};
        \draw (bib) -| ++ (2,-.4) -- ([yshift=0.1cm]link.west);
        \draw (adMac) -| ++ (2,.4) -- ([yshift=-0.1cm]link.west);
        \draw (link) -- ++(2,0) node[right, irec] (prog) {Program};
        \draw (mac) -- ++ (0,-.62) -| (link.north);
        \draw (prg)++(0,.75) node[above] {\tiny Das machen wir} -- (prg);
        \draw (prog)++(0,-.75) node[below] {\tiny Das wollen wir} -- (prog);
        \draw (comp)++(-1,1) node[above] {\tiny Semantik-Analyse} -- (comp);
        \draw (comp)++(0,.75) node[above] {\tiny Syntax-Analyse} -- (comp);
        \draw[btcd@color@alerted] (comp)--++(1,.65) node[right] {\tiny Compile-Fehler};
        \draw[btcd@color@alerted] (prog)--++(0,.65) node[above] {\tiny Laufzeit-Fehler};
        \draw (link)++(0,-0.75) node[below] {\parbox{12ex}{\tiny statisch/dynamisch\index{Java!Konvertierung!statisches\thinspace /\thinspace dynmaisches Linken}\\\hphantom{statisch\thinspace /\thinspace dy}(Java)}} -- (link);
    \end{scope}
\end{tikzpicture}
}
\end{center}
\end{frame}

\begin{frame}{Vom Text zum Programm}
    \begin{itemize}[<+(1)->]
        \widei
        \item Mit \bbash{javac :lan:Name:ran:.java} übersetzt Java die Datei in \emph{Java-Bytecode}\pause{} (\bbash{:lan:Name:ran:.class})
        \item Dieser Code kann vom Java-Interpreter ausgeführt werden:\pause{} \bbash{java :lan:Name:ran:}
        \item Ein paar wichtige Punkte: \begin{itemize}
            \widei
            \item Bei den \T{.class}-Dateien handelt es sich um \emph{keine} \T{.zip}-Dateien!
            \item Die \T{.jar}-Dateien von Java sind \T{.zip}-Archive.
            \item Der Interpreter (\bbash{java}) ist sowohl im Java-Runtime-Environment (JRE) als auch im Java-Development-Kit (JDK) enthalten.
            \item Der Compiler wird (in der Regel) nur mit der JDK ausgeliefert.
        \end{itemize}
    \end{itemize}
\end{frame}

\subsection{Bezeichner \& Variablen}

\begin{frame}{Gültige Java-Bezeichner}
    \begin{itemize}[<+(1)->]
        \widei
        \item Java stellt folgende Regeln an Bezeichner für Variablen, Klassen,~\ldots \begin{itemize}
            \widei
            \item Diese dürfen nur aus Buchstaben, Ziffern, dem Unterstrich (\_) und dem Dollarzeichen (\$) bestehen.
            \item Ein Bezeichner darf \emph{nicht} mit einer Ziffer beginnen.
            \item Schlüsselbegriffe (wie \bjava{int}, \bjava{double},~\ldots) dürfen nicht als Bezeichner verwendet werden.
        \end{itemize}
    \end{itemize}
\end{frame}

\begin{frame}{Variablen und Wertzuweisungen}
    \pause{}
    \begin{center}
        \parbox{7cm}{\!\bjava{int zahl = -42, super_zahl = 4;}\\\bjava{zahl = 42;}
        }
    \end{center}
    \begin{itemize}[<+(1)->]
        \widei
        \item Java ist eine \emph{streng typisierte} Sprache.\pause{} Jede Variable kann nur Daten eines (bestimmten) Typs speichern.
        \item Wir unterscheiden zwischen \emph{primitiven} und \emph{komplexen} Datentypen. \begin{description}[iiPrimitive:]
            \item[Primitive] \bjava{byte}, \bjava{short}, \bjava{int}, \bjava{long}, \bjava{float}, \bjava{double}, \bjava{char}, \bjava{boolean}\par Hinweis: \emph{\hyperlink{mrk:call-by-val}{call-by-value}}
            \item[Komplexe] \bjava{String}, \bjava{Random},~\ldots\par Hinweis: \emph{\hyperlink{mrk:call-by-ref}{call-by-reference}}
        \end{description}
    \end{itemize}
\end{frame}

\begin{frame}{Konvertierung primitiver Datentypen}
    \begin{itemize}[<+(1)->]
        \widei
        \item Java kann manche Datentypen implizit Umwandeln.\pause{} Es gilt:\pause{}
        \begin{center}
            \bjava{byte} \mprec{} \bjava{short} \mprec{} \bjava{int} \mprec{} \bjava{long} \mprec{} \bjava{float} \mprec{} \bjava{double}
        \end{center}
        \pause{}Sowie:\pause{}
        \begin{center}
            \bjava{char} \mprec{} \bjava{int}
        \end{center}
        \item So kann Java einen \bjava{char} implizit in einen \bjava{int} umwandeln\pause{} (da der Wertebereich größer ist).\pause{} Umgekehrt allerdings nicht.
        \item \imp{Floats und Doubles unterliegen einem Rundungsproblem.}
        \item Unter der Einschränkung des Wertebereichs kann durch \bjava{(:lan:Datentyp:ran:):lan:Variable:ran:} eine explizite Konvertierung auch in umgekehrte Richtung erfolgen.\par
        \pause{}Beispiel: \bjava{(byte)42}
    \end{itemize}
\end{frame}


\begin{frame}[fragile]{Konstanten und Hinweise}
    \begin{itemize}[<+(1)->]
        \widei
        \item Konstanten werden in Java mit dem Schlüsselbegriff \bjava{final} gekennzeichnet: \begin{plainjava}
final int SUPER_ZAHL = 42;
        \end{plainjava}
        \item Schreiben wir eine Fließkommazahl wie \bjava{3.1415}\pause{} interpretiert sie Java erstmal als \bjava{double}.\pause{} Damit es sie als \bjava{float} interpretiert müssen wir ein \bjava{f} anstellen.\pause{} Also: \bjava{3.1415f}.
        \item Zeichen (\bjava{char}) werden in Java im UTF-16 Format gespeichert.\pause{} Für die unteren \(7\)-Bit ist es identisch zur ASCII-Kodierung.
        \item Der \say{Datentyp} \bjava{void} gibt bei Methoden an,\pause{} dass diese nichts zurück geben!
    \end{itemize}
\end{frame}

\begin{frame}[fragile]{Präzedenzregeln}
    \begin{itemize}[<+(1)->]
        \widei
        \item Operationen werden in Java in einer gewissen Reihenfolge ausgeführt.\pause{} Diese wird durch die Präzedenzregeln bestimmt:\onslide<1->
        \begin{alignat*}{3}
            \onslide<4->{\Big[a\text{++},a\text{-\!\! -}\Big]}&\onslide<5->{\to~\Big[!a, -a, \text{++}a,\text{-\!\! -}a\Big]}&&\onslide<6->{\pause\to~\Big[*,/,\%\Big]} &&\onslide<7->{\to~\Big[a+b,a-b\Big]}\\
            &\onslide<8->{\to~\Big[==,>=,<,\ldots\Big]} &&\onslide<9->{\to~\Big[\&\&\Big]}&&\onslide<10->{\to~\Big[||\Big]}\onslide<1->
        \end{alignat*}
        \vspace*{-0.5cm}\onslide<1->
    \end{itemize}
\ifull
    \Task{Code zur Präzedenz}
    \begin{exercise}<11->[Was liefert dieser Code? \Time{3}]
        \begin{plainjava}[columns={[c]fullflexible}]
int x = 40, m = 3;
System.out.println(-x++ - --m);
System.out.println(x > 41 || m > 1 && 2 != 3);
        \end{plainjava}
    \end{exercise}
    \fi\onslide<1->
\end{frame}

\ifull
\begin{frame}[fragile,c]{Präzedenzregeln -- Lösung}
    \begin{solve}<2->[Was liefert dieser Code?]
        \pause{}\begin{plainjava}[columns={[c]fullflexible}]
System.out.println(-x++ - --m); // :yields: -42
        \end{plainjava}
        \pause{}Aufgrund der Postfix-Notation \bjava{x++} wird \bjava{x} \emph{nach} dem Ausdruck um \(1\) erhöht!\pause{} Wegen des Präfix-Dekrements von \bjava{m} wird \bjava{m} um eins verringert.\pause{} Nun werden \bjava{-x} (\(=-40\)) mit \bjava{m} (\(= 2\)) subtrahiert (\bjava{= -42}).\pause{} \bjava{x} ist nun \bjava{41}.
        \begin{plainjava}[columns={[c]fullflexible}]
System.out.println(x > 41 || m > 1 && 2 != 3); // :yields: true
        \end{plainjava}
        Da \bjava{||} am schwächsten bindet,\pause{} werden zuerst \bjava{x > 41} (\bjava{41 > 41} \(\to\) \bjava{false}) und \bjava{m > 1:ws:&& 2:ws:!= 3} betrachtet.\pause{} Nun wird also \bjava{m > 1} (\bjava{2 > 1} \(\to\) \bjava{true}) und \bjava{2 != 3} (\bjava{true}) ausgewertet.\pause{} Damit evaluiert das \bjava{true:ws:&& true} zu \bjava{true} und somit auch das \bjava{false:ws:|| true} zu \bjava{true}.
    \end{solve}
\end{frame}

\begin{frame}[fragile,c]{Präzedenzregeln, II}
\Task{Code zur Präzedenz, II}
\begin{exercise}<2->[Was liefert dieser Code?, II \Time{2}]
    \begin{plainjava}[columns={[c]fullflexible}]
int x = 5;
System.out.println(x);
System.out.println(x++ + ++x + x++ + ++x);
System.out.println(x);
    \end{plainjava}
    \pause{}\textit{Hinweis: Dies ist lediglich eine Bonusübung, um den Unterschied zwischen der Präfix- und der Postfixnotation klar zu machen.}
\end{exercise}
\end{frame}

\begin{frame}[fragile,c]{Präzedenzregeln, II -- Lösung}
    \begin{solve}<2->[Was liefert dieser Code?, II]
        \pause{}\begin{plainjava}[columns={[c]fullflexible}]
System.out.println(x); // :yields: 5
        \end{plainjava}
        \pause{}Zu Beginn hält \bjava{x} den Wert \(5\), so weit nichts besonderes.
        \begin{plainjava}[columns={[c]fullflexible}]
System.out.println(x++ + ++x + x++ + ++x); // :yields: 28
        \end{plainjava}
        \pause{}Dies ist wohl die schwierigste Zeile.\pause{} Aufgeschlüsselt wird hier addiert: \(5 + 7 + 7 + 9 = 28\).\pause{} Die erste \(5\), da \bjava{x} erst nachträglich erhöht wird.\pause{} Dann wird \bjava{x}, welches jetzt \(6\) ist, direkt um eins erhöht.\pause{} \bjava{x} ist jetzt \(7\), das Spiel wiederholt sich für das Hintere \bjava{++x}.\pause{}
        \begin{plainjava}[columns={[c]fullflexible}]
System.out.println(x); // :yields: 9
        \end{plainjava}
        \pause{}Durch die \(4\) Inkremente ist \bjava{x} nun \(9\).\pause{} (\textit{Das ist \emph{kein} guter Code!})
    \end{solve}
\end{frame}
\fi

\begin{frame}{Präzedenzregeln -- Kommentar}
    \begin{itemize}[<+(1)->]
        \widei
        \item Java wendet die mathematischen Rechenregeln wie bekannt an.
        \item Auch das Klammern funktioniert wie gewohnt.
        \item Boolesche Operatoren funktionieren wie in der Aussagenlogik.
    \end{itemize}
\end{frame}

\subsection{Komplexe Datentypen}

\begin{frame}[fragile]{Komplexe Datentypen}
    \begin{itemize}[<+(1)->]
        \widei
        \item Java erlaubt es, mit Klassen komplexe Datentypen zu konstruieren.
        \item Wichtig ist \bjava{String}, der von Java eine Sonderbehandlung erfährt.
        \item Bei der Deklaration eines komplexen Datentyps:\pause{}
\begin{plainjava}
Random rnd = new Random();
\end{plainjava}
        \pause{}speichert Java für \bjava{rnd} die Adresse,\pause{} an der sich die eigentlichen Daten des Objekts befinden. \textcolor{gray}{Stichwort: Heap und Stack.}
        \item Deswegen sollten komplexe Datentypen mittels \bjava{.equals()} verglichen werden.\pause{} \bjava{==} vergleicht die Speicheradressen und damit die \emph{Identität}.
    \end{itemize}
\end{frame}

\begin{frame}{Strings}
    \setbeamertemplate{description item}[default]%
    \begin{itemize}[<+(1)->]
        \widei
        \item Jede Variable/jedes Objekt kann in Java mittels \bjava{toString()} in einen \bjava{String} konvertiert werden.
        \item Wir können auf einem String-Objekt diverse Operationen aufrufen: \begin{description}[{iis1.substring(a,b):}]
            \item[{s1.equals(s2)}:] prüft ob die Zeichenketten gleich sind.
            \item[{s1.length()}:] liefert die Länge einer Zeichenkette.
            \item[{s1.charAt(k)}:] liefert das \T{k}-te Zeichen.
            \item[{s1.substring(a,b)}:] liefert einen Ausschnitt der Zeichenkette vom \(a\)-ten bis zum \(b-1\)-ten Zeichen.
            \item[{s1.toUpperCase()}:] liefert die Zeichenkette in Großbuchstaben.
        \end{description}
    \end{itemize}
\end{frame}

\subsection{Subroutinen}
\begin{frame}[fragile]{Funktionen}
    \begin{itemize}[<+(1)->]
        \item Analog zur \bjava{main}-Funktion, lassen sich in Java Routinen implementieren.
        \item Die allgemeine Syntax lautet:\pause{}
{\footnotesize
\begin{plainjava}
:lan:Zugriffsmodifikatoren:ran: :lan:Rückgabetyp:ran: :lan:Name:ran:(:lan:Parameterliste:ran:) {
    :lan:Körper:ran:
}
\end{plainjava}
}
        \item Beispiel: \pause{}
{\footnotesize
\begin{plainjava}
private static String generateWelcomeMessage(String name, int age,
        boolean married) {
    return "Welcome, " + name /* ... */ ;
}
\end{plainjava}
}
        \item Hinweis:\pause{} Bevor wir zu Klassen an sich kommen,\pause{} machen wir alle Funktionen \bjava{static},\pause{} um sie auch von der \bjava{main}-Methode aus aufrufen zu können.
    \end{itemize}
\end{frame}

\begin{frame}{Signatur \& Überladung}
    \begin{itemize}[<+(1)->]
        \widei
        \item Die Signatur einer Methode besteht aus dem Namen sowie den Datentypen der Parameter.\pause{} Der Rückgabetyp ist \emph{kein} Teil!\par
         \faAngleRight~Beispiel: \bjava{generateWelcomeMessage(String, int, boolean)}.
        \item Java verbietet zwei Funktionen mit gleicher Signatur im selben Gültigkeitsbereich.
        \item Gestattet sind Funktionen mit gleichem Namen, aber verschiedener Parameterliste.\pause{} Dieses Prinzip wird \emph{Überladen} (Overloading) genannt.
        \item Existieren mehrere Überladungen mit gleicher Parameteranzahl\pause{} (wie \bjava{m(int,double)}, \bjava{m(char,float)},~\ldots),\pause{} entscheidet Java welche Methode aufgerufen werden soll.\pause{} Es nimmt die \say{nächste}.
    \end{itemize}
\end{frame}


\subsection{Zugriffsmodifikatoren}

\begin{frame}{private und public}
    \begin{itemize}[<+(1)->]
        \widei
        \item Java erlaubt vier verschiedene Zugriffsmodifikatoren: \begin{description}[protected]
            \item[private] Zugriff ist nur innerhalb der Klasse erlaubt.
            \item[protected] ist sichtbar im gesamten Paket sowie in allen Unterklassen (\hyperlink{mrk:Vererbung}{Vererbung}).
            \item[public] ist überall sichtbar (auch über Pakete hinweg).
            \item[\say{default}] gibt man nichts an, so kann die Variable von Überall im Paket (\say{selber Ordner}) erreicht werden.
        \end{description}
        \item Hinweis (Vorgriff):\pause{} Objekte einer Klasse können auch auf private Elemente anderer Objekte der gleichen Klasse zugreifen.
    \end{itemize}
\end{frame}

\begin{frame}[fragile]{Gültigkeitsbereiche -- Scopes}
    \begin{itemize}[<+(1)->]
        \item Es gibt vier \emph{Geltungsbereiche} (\textit{engl.} scopes) für Variablen.
\begin{plainjava}[language=xJava]
!*\onslide<3->*!|inc|public|inc| class Pinguin { // !*\solGet{comments}{\textbf{1.}}*! global
!*\onslide<4->*!    !**!|inc|private|inc| int sozialversicherungsnummer; // !*\solGet{comments}{\textbf{2.}}*! Klasse
!*\onslide<5->*!    !**!|inc|protected|inc| int watschelIndex; // !*\solGet{comments}{\textbf{3.}}*! geschützt
!*\onslide<6->*!    !**!|inc|public|inc| void watschel() { /* ... */ } // !*\solGet{comments}{\textbf{4.}}*! (erneut: global)
!*\onslide<7->*!    void piepsen() { /* ... */ } // !*\solGet{comments}{\textbf{5.}}*! Standard: package
!*\onslide<3->*!}
\end{plainjava}
    \item<8-> Die Klasse \bjava{Tiger} sei im selben, \bjava{Auto} sei in einem anderen Paket, \bjava{Felspingu} erbe von \bjava{Pinguin} aber sei in einem anderem Paket: \medskip\par
    \onslide<9->{\def\y{\faCheckSquareO}\def\n{\hspace*{-1pt}\faSquareO}\scriptsize\begin{minipage}{.45\linewidth}
        \begin{tabular}{p{6em}ccccc}
             \toprule
                            & 1 & 2 & 3 & 4 & 5 \\
            \midrule
                \T{Pinguin} & \y & \y & \y & \y & \y \\
                \T{Tiger}   & \y & \n & \y & \y & \y \\
            \bottomrule
        \end{tabular}
            \end{minipage}\hfill\begin{minipage}{0.45\linewidth}
        \begin{tabular}{p{6em}ccccc}
                \toprule
                            & 1 & 2 & 3 & 4 & 5 \\
            \midrule
                \T{Auto} & \y & \n & \n & \y & \n \\
                \T{Felspingu} & \y & \n & \y & \y & \n \\
            \bottomrule
        \end{tabular}
            \end{minipage}}
    \end{itemize}
\end{frame}

\ifull
\begin{frame}[c,fragile]{Übungsaufgabe}
    \colorlet{folderbg}{btcd@color@secondary}\Task{Sichtbarkeiten in Java}
    \begin{exercise}<2->[Sichtbarkeiten \Time{6}]
        Welche der Aufrufe (1 bis 6) sind gültig oder ungültig und warum? Nehmen Sie an, dass die Dateien und Importe korrekt sind. \par
\begin{onlyenv}<4->
\begin{minipage}{.7\linewidth}
\lstfs{8}%
\begin{plainjava}[morekeywords={[3]{Apple,Banana}},belowskip=0pt]
public class Apple { void eat() { /* ... */ } }
public class Banana extends Apple {
  public int lookAt(int x) { /* ... */ }
  public void eat() { /* ... */ }
  protected boolean eat(float amount) { /* ... */ }
}
\end{plainjava}
\begin{plainjava}[multicols=2,morekeywords={[3]{Apple,Banana,Skyrim,Morrowind}},aboveskip=0pt]
class Morrowind {
  static int test() {
    new Apple().eat(); // 1
    Banana b = new Banana();
    b.lookAt(14); // 2
    b.eat(); // 3
}}
class Skyrim extends Banana {
  static int test() {
    new Apple().eat(); // 4
    eat(3f); // 5
    Banana b = new Banana();
    b.lookAt(14); // 6
}}
\end{plainjava}
\end{minipage}%
\end{onlyenv}%
\hfill\onslide<3->{\parbox{.275\linewidth}{\begin{fancydir}
[worlds, dir
    [fruits, dir
        [Apple.java, cfile={btcd@color@primary}{J}]
        [Banana.java, cfile={btcd@color@primary}{J}]
    ]
    [Skyrim.java, cfile={btcd@color@primary}{J}]
    [Morrowind.java, cfile={btcd@color@primary}{J}]
]
\end{fancydir}}}
    \end{exercise}
\end{frame}

\begin{frame}[c]{Lösung}
    \begin{solve}<2->[Sichtbarkeiten]
       \begin{enumerate}[<+(1)->]
            \item \emph{Ungültig}: Die Konstruktion eines \say{Apple-Objektes} funktioniert, \bjava[morekeywords={[3]{Apple}}]{Apple::eat()} ist in \bjava[morekeywords={[3]{Morrowind}}]{Morrowind} aber nicht sichtbar (package private).
            \item \emph{Gültig}: Von \bjava[morekeywords={[3]{Banana}}]{Banana} kann ein Objekt erzeugt werden, \bjava[morekeywords={[3]{Banana}}]{Banana::lookAt(int)} ist \bjava{public} sichtbar.
            \item \emph{Gültig}: Von \bjava[morekeywords={[3]{Banana}}]{Banana} kann ein Objekt erzeugt werden, weiter erweitert \bjava[morekeywords={[3]{Banana}}]{Banana::eat()} mit dem Überschreiben die Sichtbarkeit von \bjava[morekeywords={[3]{Apple}}]{Apple::eat()} auf \bjava{public}.
       \end{enumerate}
    \end{solve}
\end{frame}

\begin{frame}[c]{Lösung}
    \addtocounter{solve}{-1}%
    \begin{solve}<2->[Sichtbarkeiten\hfill(Fortsetzung)]
       \begin{enumerate}[<+(1)->]
            \setcounter{enumi}{3}
           \item \emph{Ungültig}: Die Konstruktion eines \say{Apple-Objektes} funktioniert, \bjava[morekeywords={[3]{Apple}}]{Apple::eat()} ist in \bjava[morekeywords={[3]{Skyrim}}]{Skyrim} aber nicht sichtbar (package private).
           \item \emph{Ungültig}: \bjava[morekeywords={[3]{Skyrim}}]{Skyrim::test()} ist statisch, für \bjava[morekeywords={[3]{Banana}}]{Banana::eat(float)} benötigt es aber ein Objekt.
           \item \emph{Gültig}: Von \bjava[morekeywords={[3]{Banana}}]{Banana} kann ein Objekt erzeugt werden, \bjava[morekeywords={[3]{Banana}}]{Banana::lookAt(int)} ist \bjava{public} sichtbar.
       \end{enumerate}
    \end{solve}
\end{frame}
\fi

\begin{frame}[c]{Es gilt festzuhalten\ldots}
    \begin{definition}[Subroutinen]
        Eine Subroutine/Funktion ist ein \emph{Unterprogramm},\pause{} das von anderen Programm(teilen) aufgerufen und so wiederverwendet werden kann.\pause{}\medskip\newline Sie wird durch ihre \emph{Signatur} eindeutig identifiziert,\pause{} die in Java aus einem Bezeichner sowie einer Liste an Datentypen der Parameter besteht. Ein Unterprogramm ist in der Lage Daten an den aufrufenden Teil zurückzuliefern.
    \end{definition}
\end{frame}

\fullsubsection{Übungsaufgaben}
\ifull
\begin{frame}[c,fragile]{Übungsaufgabe}
    \Task{Suche syntaktischer Fehler, I}
    \begin{exercise}<2->[Fehler finden, I \Time{3}]
        \pause{}Finden und korrigieren Sie alle (syntaktischen) Fehler:\pause{}
        \begin{plainvoid}
public void main(String[] hihi) {
    private final int X = 12;
    System.out.printLn("Huhu " + (byte) X)
}
        \end{plainvoid}
    \end{exercise}
\end{frame}

\begin{frame}[c,fragile]{Lösung}
    \begin{solve}<2->[Fehler finden, I]
        \pause{}\begin{plainjava}
public static void main(String[] hihi) {
    final int X = 12;
    System.out.println("Huhu " + (byte) X);
}
        \end{plainjava}
    \begin{enumerate}[<+(1)->]
        \item Die \bjava{main}-Methode muss \bjava{static} sein!
        \item Zugriffsmodifikatoren wie \bjava{private} dürfen nicht in einer Funktion auftreten.
        \item Die Funktion \bjava{printLn} heißt \bjava{println}.
        \item In der Ausgabezeile fehlt ein Semikolon!
    \end{enumerate}
    \pause{}\textit{Hinweis: das \say{hihi} ist kein Fehler!}
    \end{solve}
\end{frame}

\begin{frame}[c,fragile]{Übungsaufgabe}
    \Task{Suche syntaktischer Fehler, II}
    \begin{exercise}<2->[Fehler finden, II \Time{3}]
        \pause{}Finden und korrigieren Sie alle (syntaktischen) Fehler:\pause{}
        \begin{plainvoid}
public int double(int x) { return (2*)x; }

public static void main() {
    System.in.print(double(21));
}
        \end{plainvoid}
    \end{exercise}
\end{frame}

\begin{frame}[c,fragile]{Lösung}
    \begin{solve}<2->[Fehler finden, II]
        \pause{}\begin{plainjava}
public static int doDouble(int x) { return (2*x); }
public static void main(String[] args) {
    System.out.print(doDouble(21));
}
        \end{plainjava}
    \begin{enumerate}[<+(1)->]
        \item Der Bezeichner \bjava{double} ist für eine Funktion nicht erlaubt.
        \item Der Ausdruck in der Funktion ist ungültig geklammert.
        \item Die Funktion muss \bjava{static} sein um aufgerufen werden zu können.
        \item Die \bjava{main}-Methode benötigt die Signatur \bjava{main(String[])}!
        \item \bjava{System.in} hat die Funktion \bjava{print(String)} nicht.
    \end{enumerate}
    \end{solve}
\end{frame}

\begin{frame}[c,fragile]{Übungsaufgabe}
    \Task{Methodensignaturen}
    \begin{exercise}<2->[Signaturen \Time{2}]
        \pause{}Geben Sie die Signaturen folgender Funktionen an.\pause{} Sind die Java-Funktionen ungültig, geben Sie bitte eine kurze Begründung an:\pause{}
        \begin{plainvoid}
String a(int a1, double a2) { /* ... */ }
int b(String[] b1, int b2, double[] b3) { /* ... */ }
double c(String... c1, char c2) { /* ... */ }
double d(int d1, double... d2) { /* ... */ }
        \end{plainvoid}
    \end{exercise}
\end{frame}

\begin{frame}[c]{Lösung}
    \begin{solve}<2->[Signaturen]
        \begin{enumerate}[<+(1)->]
            \item \bjava{String a(int a1, double a2)} hat die Signatur \bjava{a(int, double)}
            \item \bjava{int b(String[] b1, int b2, double[] b3)} hat die Signatur \bjava{b(String[], int, double[])}
            \item \bjava{double c(String... c1, char c2)} ist nicht möglich,\pause{} da nach \emph{varargs} keine verpflichtenden Argumente mehr kommen dürfen.
            \item \bjava{double d(int d1, double... d2)} hat die Signatur \bjava{d(int,double...)}
        \end{enumerate}
    \end{solve}
\end{frame}


\begin{frame}[c,fragile]{Übungsaufgabe}
    \Task{Konkatenation}
    \begin{exercise}<2->[Konkatenation \Time{2}]
        \pause{}Welche Ausgabe erzeugen die folgenden Zeilen (jeweils)?\pause{}\lstfs{10}
        \begin{plainjava}
System.out.println("2 + 3 = " + 2 + 3);
System.out.println(-'=' + 2 * 3 + '=' + " 2*3");
System.out.println(2 - 12 + " * 2 = " + -1!*\solGet{numbers}{\_}*!0 * 2 + ("_" + 'b' + 10));
System.out.println("c - a - 1 = " + ('c' - 1 - 'a'));
        \end{plainjava}
    \end{exercise}
\end{frame}

\begin{frame}[c,fragile]{Lösung}
    \begin{solve}<2->[Konkatenation]
        \pause{}Hinweis: ich schreibe \bjava{println} kurz für \bjava{System.out.println}. \pause{}
\begin{plainjava}
println("2 + 3 = " + 2 + 3); // :yields: '2 + 3 = 23'
\end{plainjava}
    \pause{}Da der linke Operand von \bjava{+} ein \bjava{String} ist,\pause{} wird erst die \bjava{2} als String konkateniert,\pause{} und dann die \bjava{3} ebenfalls als Zeichen.\pause{}
\begin{plainjava}
println(-'=' + 2 * 3 + '=' + " 2*3"); // :yields: '6 2*3'
\end{plainjava}
    \pause{}Da die \bjava{char}s einfach zu ihrem Integerwert konvertiert werden,\pause{} findet erst beim letzten Plus eine Konkatenation statt.
    \end{solve}
\end{frame}

\begin{frame}[c,fragile]{Lösung}
    \addtocounter{solve}{-1}
    \begin{solve}<1->[Konkatenation\hfill(Fortsetzung)]
\pause{}
\begin{plainjava}
println(2 - 12 + " * 2 = " + -1!*\solGet{numbers}{\_}*!0 * 2 + ("_" + 'b' + 10));
    // :yields: '-10 * 2 = -20_b10'
\end{plainjava}
    \pause{}Die erste arithmetische Operation liefert \bjava{-10},\pause{} welches dann zum String konkateniert wird.\pause{} Anschließend greift die Präzedenzregel \say{Punkt vor Strich}\pause{} wobei das Ergebnis anschließend als String konkateniert wird.\pause{} Dann wird in der Klammer durch erneute Konkatenation \bjava{_b10} erzeugt.\pause{}\\
    \textit{Hinweis: Der Unterstich bei \bjava{-10} ist in Java (zwischen Zahlen) komplett valide\pause{} und wird in der Regel zum trennen von \(1000\)er Blöcken verwendet.}
    \end{solve}
\end{frame}

\begin{frame}[c,fragile]{Lösung}
    \addtocounter{solve}{-1}
    \begin{solve}<1->[Konkatenation\hfill(Fortsetzung)]
\pause{}
\begin{plainjava}
println("c - a - 1 = " + ('c' - 1 - 'a'));
    // :yields: 'c - a - 1 = 1'
\end{plainjava}
    \pause{}Java berechnet zuerst die Operationen in der Klammer und wandelt die Zeichen in ihre ASCII-Werte um.\pause{} Diese muss man hierfür nicht kennen,\pause{} da es genügt zu Wissen, dass zwischen \T{a} und \T{c} ein Abstand von \(2\) vorliegt der um eins verringert wird.\pause{}
    Wir berechnen also in der Klammer den wert \(1\).\pause{} Dieser wir nun normal zum String konkateniert, was die Ausgabe erzeugt.
    \end{solve}
\end{frame}


\begin{frame}[c]{Übungsaufgabe}
    \Task{Javas' Typkonvertierung}
    \begin{exercise}<2->[Javas' Typkonvertierung \Time{5}]
        \pause{}\pause{}Bitte geben Sie für jeden Ausdruck den Typ des resultierenden Wertes oder Objekts an und erklären Sie kurz\pause{} (Tipp: Alle Ausdrücke sind korrekt):\pause
\begin{multicols}{2}
    \begin{enumerate}
        \item \bjava{2 + 3}
        \item \bjava{7 - 2.0}
        \item \bjava{'x' * (byte) 4}
        \item \bjava{5L/(char) 2}
        \item \bjava{(short)3 + (byte)-3.1415}
        \item \bjava{(float)0b1001 + (long)0x42)}
        \item \bjava{new Scanner(System.in)}
        \item \bjava{\"x\" + 3}
    \end{enumerate}
\end{multicols}
    \end{exercise}
\end{frame}

\begin{frame}[c]{Lösung}
    \begin{solve}<2->[Javas' Typkonvertierung]
\pause\begin{enumerate}[<+(1)->]
    \widei
    \item \say{\bjava{2 + 3}}: \bjava{int} \textcolor{gray}{(5)}\pause\par Die Addition zweier Integer liefert wieder einen Integer.
    \item \say{\bjava{7 - 2.0}}: \bjava{double} \textcolor{gray}{(5.0)}\pause\par Die Integer und Double können nicht subtrahiert werden. Deswegen konvertiert Java \bjava{7} implizit zu einem \bjava{double} und die Differenz liefert wieder ein Double.
    \item \say{\bjava{'x' * (byte) 4}}: \bjava{int} \textcolor{gray}{(480)}\pause\par
    Ein Character kann nicht mit einem Byte multipliziert werden. Deswegen konvertiert Java \bjava{'x'} zu einem Integer (zugehöriger UTF16-Wert) und anschließend das Byte ebenfalls zu einem Integer.
\end{enumerate}
    \end{solve}
\end{frame}

\begin{frame}[c]{Lösung}
    \addtocounter{solve}{-1}%
    \begin{solve}<2->[Javas' Typkonvertierung\hfill{Fortsetzung}]
\pause\begin{enumerate}[<+(1)->]
    \widei
    \setcounter{enumi}{3}
    \item \say{\bjava{5L/(char) 2}}: \bjava{long} \textcolor{gray}{(2)}\pause\par
    \bjava{5L} besitzt den Typ \bjava{long}, damit wird auch der Character mit UTF16-Wert \(2\) in ein \bjava{long} konvertiert. Das Ergebnis der Division aus \bjava{long} und \bjava{long} ergibt wieder \bjava{long}
    \item \say{\bjava{(short)3 + (byte)-3.1415}}: \bjava{int} \textcolor{gray}{(0)}\pause\par
    Java konvertiert die Summe aus einem Short und einem Byte implizit zu einem Integer.
    \item \say{\bjava{(float)0b1001 + (long)0x42}}: \bjava{float} \textcolor{gray}{(75.0)}\pause\par
    Java konvertiert die Summe aus einem Float und einem Long zu einem Float (da Float die Werte von Long inkludiert).
\end{enumerate}
    \end{solve}
\end{frame}

\begin{frame}[c]{Lösung}
    \addtocounter{solve}{-1}%
    \begin{solve}<2->[Javas' Typkonvertierung\hfill{Fortsetzung}]
\pause\begin{enumerate}[<+(1)->]
    \widei
    \setcounter{enumi}{6}
    \item \say{\bjava{new Scanner(System.in)}}: \bjava{Scanner} \textcolor{gray}{(oder voll: \T{java.util.Scanner})}\pause\par
    Mit \bjava{new} wird ein neues Objekt der angegeben Klasse erstellt.
    \item \say{\bjava{\"x\" + 3}}: \bjava{String} \textcolor{gray}{(x3)}\pause\par
    Analog zur Aufgabe zur String-Konkatenation wird hier \bjava{3} in eine Zeichenkette konvertiert.
\end{enumerate}
    \end{solve}
\end{frame}
\fi