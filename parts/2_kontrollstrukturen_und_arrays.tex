\section{Kontrollstrukturen und Arrays}
\subsection{Fallunterscheidungen}

\begin{frame}[fragile]{Fallunterscheidung mit if}
    \begin{itemize}[<+(1)->]
        \item Java erlaubt Fallunterscheidungen mit \bjava{if}:\pause{}
\begin{plainjava}
if(:lan:Bedingung |plain|1|plain|:ran:) {
    // Bedingung 1 ist wahr
} else if (:lan:Bedingung |plain|2|plain|:ran:) {
    // Bedingung 1 ist falsch, Bedingung 2 ist wahr
} else {
    // Beide Bedingungen sind falsch
}
\end{plainjava}
    \item Folgt der \bjava{if}-Instruktion nur eine Anweisung,\pause{} so können die geschwungenen Klammern weggelassen werden.
    \end{itemize}
\end{frame}


\begin{frame}[fragile]{Kompakte if-Notation}
    \begin{itemize}[<+(1)->]
        \widei
        \item Eine (einfache) Fallunterscheidung können wir verkürzen:\pause{}
\begin{plainjava}
:lan:Bedingung:ran: ? :lan:Bedingung-wahr:ran: : :lan:Bedingung-falsch:ran:;
\end{plainjava}
        \pause{}Trifft die Bedingung zu, wird der \emph{wahr}-Teil, sonst der \emph{falsch}-Teil ausgeführt.
        \item Ein Beispiel:\pause{}
\begin{plainjava}
int n = 14;
return (n >= 18) ? "Volljährig" : "Nicht volljährig";
\end{plainjava}
        \pause{}\faAngleRight~Liefert: \bjava{\"Nicht volljährig\"}.
        \item Diese Anweisung kann (beliebig tief) verschachtelt werden.
    \end{itemize}
\end{frame}


\begin{frame}[fragile]{Fallunterscheidung mit switch-case}
    \begin{itemize}[<+(1)->]
        \item Java erlaubt Fallunterscheidungen mit \bjava{switch-case}:\pause{}
\begin{plainjava}
!*\onslide<3->*!switch(:lan:Ausdruck:ran:){
!*\onslide<4->*!    case :lan:Fall |plain|1|plain|:ran:: // Code für Fall 1
!*\onslide<5->*!    case :lan:Fall |plain|2|plain|:ran::
!*\onslide<6->*!        // Code für Fall 1 & 2
!*\onslide<6->*!        break; // Verlässt Anweisung
!*\onslide<7->*!    case :lan:Fall |plain|3|plain|:ran::
!*\onslide<7->*!        // Code für Fall 3
!*\onslide<7->*!        break;
!*\onslide<8->*!    default: // Code, wenn keiner der Fälle greift.
!*\onslide<3->*!}
\end{plainjava}
    \item<10-> Dies funktioniert für: \onslide<11->{Zahlen (also auch \bjava{char}, das konvertiert werden kann), \onslide<12->{\bjava{enum}-Konstanten \onslide<13->{und (seit Java Version-7) auch für \bjava{String}s.}}}
    \end{itemize}
\end{frame}

\subsection{Schleifen}

\begin{frame}[fragile]{while}
    \begin{itemize}[<+(1)->]
        \item Schleifen erlauben es in Java bestimmte Programmanweisungen mehrfach auszuführen:\pause{}
        \begin{plainjava}
while (:lan:Bedingung:ran:) {
    :lan:Anweisung(en):ran:
}
        \end{plainjava}
        \pause{}Die Schleife wird so lange ausgeführt, wie die \bjava{Bedingung} \emph{wahr} ist.
    \end{itemize}
\end{frame}

\begin{frame}[fragile]{while und do-while}
    \begin{itemize}[<+(1)->]
        \item Im Gegensatz zur \bjava{while}-Schleife führt die \bjava{do-while}-Schleife die Prüfung am Ende des Durchlaufs durch.\pause{} Sie wird also mindestens einmal durchgeführt,\pause{} auch wenn die Bedingung von Beginn an \emph{falsch} ist:
        \begin{plainjava}
do {
    :lan:Anweisung(en):ran:
} while (:lan:Bedingung:ran:)|bhl|;|bhl| // <- Semikolon!
        \end{plainjava}
        \item Wichtig ist das Semikolon am Ende der \bjava{do-while}-Anweisung.\par \textcolor{gray}{Es wird gern vergessen}
    \end{itemize}
\end{frame}

\ifull
\begin{frame}[c,fragile]{Übungsaufgabe}
\Task{While vs. Do-While, I}
\begin{exercise}<2->[While vs. Do-While, I \Time{3}]
    Erzeugen die Schleifen für \(i = 0\) die selbe Ausgabe?\pause{}
    Was ändert sich, wenn \(i = 42\) anstelle von \(i = 0\)?\pause
\begin{plainjava}[multicols=2,aboveskip=0pt]
int i = ?; /* a) 0, b) 42 */
do {
    System.out.print(i++);
} while(i < 5);
int i = ?; /* a) 0, b) 42 */
while(i < 5) {
    System.out.print(i++);
}
\end{plainjava}
\end{exercise}
\end{frame}

\begin{frame}[c,fragile]{Lösung}
    \begin{solve}<2->[While vs. Do-While, I]
        \begin{itemize}[<+(1)->]
        \item Für \(i = 0\) produzieren beide Schleifen \say{01234} und brechen dann für \bjava{5 < 5} ab.
        \item Für \(i = 42\) unterscheiden sich die Ausgaben. Hier gibt do-while noch \say{42} aus, während die while Schleife gar nicht erst betreten wird.
        \end{itemize}
    \end{solve}
\end{frame}

\begin{frame}[c,fragile]{Übungsaufgabe}
    \Task{While vs. Do-While, II}
    \begin{exercise}<2->[While vs. Do-While, II \Time{3}]
        Erzeugen die Schleifen für \(i = 0\) die selbe Ausgabe?\pause{}
        Gilt dies auch für \(i = 15\)?\pause
    \begin{plainjava}[multicols=2,aboveskip=0pt]
int i = ?; /* a) 0, b) 15 */
do {
    System.out.print(i);
} while(i++ < 7);
int i = ?; /* a) 0, b) 15 */
while(i++ < 7) {
    System.out.print(i);
}
    \end{plainjava}
    \end{exercise}
    \end{frame}

    \begin{frame}[c,fragile]{Lösung}
        \begin{solve}<2->[While vs. Do-While, II]
            \begin{itemize}[<+(1)->]
            \item Für \(i = 0\) unterscheiden sich die Ausgaben! do-while produziert hier \say{01234567} und while \say{1234567} (da die Inkremente einmal vor der Ausgabe und einmal danach stattfinden). Bis \(7\) gehen die Ausgaben, da in \bjava{i++ < 7} die Variable \bjava{i} immer erst \emph{nach} dem Vergleich inkrementiert wird.
            \item Für \(i = 15\) unterscheiden sich die Ausgaben ebenfalls. Hier gibt do-while noch \say{15} aus, während die while Schleife wieder nicht betreten wird.
            \end{itemize}
        \end{solve}
    \end{frame}
\fi

\begin{frame}[fragile]{Zählschleifen mit for}
    \begin{itemize}[<+(1)->]
        \widei
        \item Die \bjava{for}-Schleife in Java besteht aus drei optionalen Komponenten,\pause{} die alle leer sein können.
        \begin{plainjava}
for(:lan:Start:ran:; :lan:Schleifenbedingung:ran:; :lan:Nach:ran:) {
    :lan:Anweisung(en):ran:
}
        \end{plainjava}
        \item Betrachten wir ein Beispiel:\pause{}
\begin{plainjava}
for(int i = 0, k = 1; i < 20; i++, k *= 2)
    System.out.println(i + ": " + k);
\end{plainjava}
    \pause{}Diese Schleife liefert im \T{i}-ten Durchlauf \(k = 2^i\),\pause{} bis zu \(i = 19\).
        \item Hinweis: \bjava{for(;;) \{ /* ... */:ws:\}} ist eine Endlosschleife.
    \end{itemize}
\end{frame}

\begin{frame}[fragile]{break und continue}
    \begin{itemize}[<+(1)->]
        \widei
        \item Das Schlüsselwort \bjava{break} bricht die \say{innerste} Schleife ab.
        \item Mit \bjava{continue} wird nur der aktuelle Durchlauf abgebrochen und,\pause{} zum Beispiel in einer \bjava{for}-Schleife,\pause{} das Inkrement durchgeführt:\pause{}
\begin{plainjava}
for(int i = 0, k = 1; i < 20; i++, k *= 2) {
    if(i % 2 == 0) continue;
    System.out.println(i + ": " + k);
}
\end{plainjava}
        \pause{}Alle geraden \T{i}-Werte werden übersprungen, die Ausgabe erfolgt nicht.
    \end{itemize}
\end{frame}

\begin{frame}[fragile]{Iterieren mit foreach}
    \begin{itemize}[<+(1)->]
        \widei
        \item Mit Java Version 5 gibt es eine weitere Variante der \bjava{for}-Schleife:\pause{}
\begin{plainjava}
for(:lan:Variable-Deklaration:ran: : :lan:Iterierbare Variable:ran:) {
    // Etwas mit der Variable machen
}
\end{plainjava}
        \pause{}\textit{Hinweis: Ohne Wissen über \hyperref[mrk:int-iterable]{Interfaces} ist es schwer zu \say{erfassen}, welche Datentypen erlaubt sind.\pause{} Darunter: Arrays, Listen,~\ldots}
        \item Ein Beispiel:\pause{}
\begin{plainjava}
int[] dinge = new int[] {42, 21, 12, -4};
for(int ding : dinge){
    System.out.println(ding);
}
\end{plainjava}
    \end{itemize}
\end{frame}

\subsection{Arrays}
\begin{frame}[fragile]{Arrays, Grundlagen}
    \begin{itemize}[<+(1)->]
        \widei
        \item Arrays sind eine komplexe Datenstruktur,\pause{} die aus mehreren Elementen des gleichen Typs aufgebaut ist.
        \item Der Index eines Arrays beginnt bei \(0\).
        \item Die Länge eines Arrays ist \emph{fest}:\pause{}
\begin{plainjava}
double[] a = new double[42]; // 42 Elemente, alle: 0.0d
int[] b = {2, 4, 6, 8, 10}; // 5 Elemente
char[] c = new char[] {'a', 'z', '9'}; // 3 Elemente
\end{plainjava}
        \item Die Länge erhalten wir durch \bjava{:lan:array:ran:.length}.\pause{} Dies ist (im Gegensatz zu \bjava{:lan:string:ran:.length()}) \emph{keine} Methode!
    \end{itemize}
\end{frame}

\begin{frame}[fragile]{Arrays, Technischer Hintergrund}
    \begin{itemize}[<+(1)->]
        \widei
        \item Arrays werden in Java als ein Block gespeichert.\pause{} Das bedeutet, ein Array aus \(12\) \bjava{int}-Elementen nimmt einen Speicherblock von \(12 \cdot 32\,\text{bit}\) ein.
        \item Greift man auf einen nicht erlaubten Index zu,\pause{} so wird eine \bjava{ArrayIndexOutOfBoundsException} geworfen.
        \item Da Arrays eine komplexe Datenstruktur sind, wird bei der Erstellung in der Variable selbst nur die Speicheradressen abgelegt.\pause{}
\begin{plainjava}
int[] a = {2, 4, 6, 8, 10}, b = a;
\end{plainjava}
        \pause{}Hier verweisen beide Variablen auf den selben Datensatz:\pause{}
\begin{plainjava}
a[4] = 42; b[2] = 7;
for(int i = 0; i < a.length; i++)
    System.out.print(a[i] + " "); // :yields: 2 4 7 8 42
\end{plainjava}
    \end{itemize}
\end{frame}

\begin{frame}[fragile]{Über Arrays iterieren}
    \begin{itemize}[<+(1)->]
        \widei
        \item Wir können,\pause{} mit einer for-Schleife über ein Array iterieren:\pause{}
\begin{plainjava}
for(int i = 0; i < a.length; i++) {
    System.out.print(a[i] + " "); // :yields: 2 4 7 8 42
}
\end{plainjava}
        \item Oder mit \say{for-each}:\pause{}
\begin{plainjava}
for(int i : a) {
    System.out.print(i + " "); // :yields: 2 4 7 8 42
}
\end{plainjava}
    \end{itemize}
\end{frame}


\ifull
\begin{frame}[c,fragile]{Übungsaufgabe}
\Task{Arrays halbieren}
\begin{exercise}<2->[Arrays halbieren \Time{3}]
    Gegeben ein Array \bjava{int[] arr} \emph{variabler} Länge. Schreiben Sie Java-Code, der ein neues Array \bjava{int[] newArr} erstellt und so befüllt, dass es nur noch die Elemente mit geradem Index aus \bjava{arr} enthält.\par\pause{}
    Beispiele:\pause
\begin{plainjava}
{7, 23, 15, 8, 10, 2} !*$\Rightarrow$*! {7, 15, 10}
{1, 2, 3} !*$\Rightarrow$*! {1, 2}
{31} !*$\Rightarrow$*! {31}
\end{plainjava}
\end{exercise}
\end{frame}

\begin{frame}[c,fragile]{Lösung}
    \begin{solve}<2->[Arrays halbieren]
    Zunächst müssen wir die Größe des neuen Arrays berechnen:
\begin{plainjava}
int[] newArr = new int[(arr.length + 1)/2];
\end{plainjava}
    Sauberer (aber nicht gefordert) wäre:
\begin{plainjava}
int[] newArr = new int[(int) Math.ceil(arr.length/2.0)];
\end{plainjava}
    Unabhängig davon, der Übertrag der Elemente:
\begin{plainjava}
for(int i = 0; i < arr.length; i += 2) {
    newArr[i/2] = arr[i];
}
\end{plainjava}
    \end{solve}
\end{frame}

\fi


\subsection{Mehrdimensionale Konzepte}%Array und Schleifenkonzepte}

\begin{frame}[fragile]{Mehrdimensionale Arrays}
    \begin{itemize}[<+(1)->]
        \widei
        \item Wir sind in der Lage, mehrdimensionale Arrays zu erstellen:\pause{}
\begin{plainjava}
double[][][] a = new double[2][4][6];
int[][] b = {{1,2}, {2,3}, {1,2,4,5}};
\end{plainjava}
        \item Wir erstellen also ein Array von Arrays von Arrays von \ldots
        \item Der Zugriff auf ein spezifisches Element erfolgt über die Angaben mehrerer Indizes:
\begin{plainjava}
int x = b[2][3]; // x :yields: 5
\end{plainjava}
    \end{itemize}
\end{frame}

\begin{frame}[fragile]{Verschachtelte for-Schleifen}
    \begin{itemize}[<+(1)->]
        \widei
        \item Über ein solches Array können wir auch iterieren:\pause{}
\begin{plainjava}
!*\onslide<3->*!public static void printMatrix(int[][] m){
!*\onslide<4->*!    for(int row = 0; row < m.length; row++) {
!*\onslide<5->*!        for(int col = 0; col < m[row].length; col++) {
!*\onslide<6->*!            System.out.print(m[row][col] + " ");
!*\onslide<5->*!        }
!*\onslide<7->*!        System.out.println();
!*\onslide<4->*!    }
!*\onslide<3->*!}
\end{plainjava}
    \end{itemize}
\ifull
    \Task{Zählschleife in Iterationsschleife umwandeln}
    \begin{exercise}<8->[Iteration mit for-each \Time{4}]
        Schreiben Sie diese Java-Methode so um, dass sie den for-each Mechanismus verwendet.
    \end{exercise}
\fi
\end{frame}

\ifull
\begin{frame}[c,fragile]{Verschachtelte for-Schleifen -- Lösung}
    \begin{solve}<2->[Iteration mit for-each]
\begin{plainjava}
!*\onslide<3->*!public static void printMatrix(int[][] m){
!*\onslide<4->*!    for(int[] row : m) {
!*\onslide<5->*!        for(int col : row) {
!*\onslide<6->*!            System.out.print(col + " ");
!*\onslide<5->*!        }
!*\onslide<7->*!        System.out.println();
!*\onslide<4->*!    }
!*\onslide<3->*!}
\end{plainjava}
    \end{solve}
\end{frame}
\fi

% #region Übungsaufgaben
\fullsubsection{Übungsaufgaben}
\ifull
\begin{frame}[c,fragile]{Übungsaufgabe}
    \Task{Suche Kompilier- und Laufzeitfehler, III}
    \begin{exercise}<2->[Fehler finden, III \Time{4}]
        \pause{}Finden und korrigieren Sie alle Kompilier- und Laufzeitfehler:\pause{}
        \begin{plainvoid}
int[][][] x = {{}, {{1,2},{2},{5}}, {{3}}};
foreach (int[][] 'ex' : x) {
    int[] eex = ex[0];
    for(int i = 0; i < ex.length; i++)
        System.out.print(eex[i] + " ");
}
        \end{plainvoid}
        Wie lautet die Ausgabe Ihres korrigierten Codes?
    \end{exercise}
\end{frame}

\begin{frame}[c,fragile]{Lösung}
    \begin{solve}<2->[Fehler finden, III]
        \pause{}\begin{plainjava}
for (int[][] ex : x) {
    if(ex.length == 0) continue;
    int[] eex = ex[0];
    for(int i = 0; i < eex.length; i++)
        System.out.print(eex[i] + " ");
}
        \end{plainjava}
    \begin{enumerate}[<+(1)->]
        \item \bjava{foreach} ist kein gültiger Schlüsselbegriff, weiter darf ein Variablen\-be\-zei\-chner keine Anführungszeichen enthalten.
        \item Die folgende Zuweisung scheitert, wenn \bjava{ex} kein nulltes Element besitzt.
    \end{enumerate}
    \end{solve}
\end{frame}

\begin{frame}[c,fragile]{Lösung}
    \addtocounter{solve}{-1}
    \begin{solve}<1->[Fehler finden, III\hfill(Fortsetzung)]
        \begin{plainjava}
for (int[][] ex : x) {
    if(ex.length == 0) continue;
    int[] eex = ex[0];
    for(int i = 0; i < eex.length; i++)
        System.out.print(eex[i] + " ");
}
        \end{plainjava}
    \begin{enumerate}[<+(1)->]
        \item[3.] Die innere \bjava{for}-Schleife prüft weiterhin auf die Länge eines anderen Arrays als das es zugreift.
    \end{enumerate}
    \pause{}Die Ausgabe lautet: \bjava{\"1 2 3 \"}. \textcolor{gray}{(Anführungszeichen nur zur Übersicht)}
    \end{solve}
\end{frame}

% int[][][][][] x = new inr[8][4][][][]

\begin{frame}[c]{Übungsaufgabe}
    \Task{Begriffserklärungen}
    \begin{exercise}<2->[Begriffserklärungen \Time{4}]
        \pause{}Erklären Sie bitte jeweils in ein bis zwei Sätzen: \begin{enumerate}[<+(1)->]
            \item[i)] Was ist ein Deadlock (und wann tritt er auf)?
            \item[ii)] Was veranschaulicht ein Histogramm?
            \item[iii)] Was versteht man unter einem deterministischen Algorithmus?
        \end{enumerate}
    \end{exercise}
\end{frame}

\begin{frame}[c,fragile]{Lösung}
    \begin{solve}<2->[Begriffserklärungen]
        \begin{enumerate}[<+(1)->]
            \item[i)] Ein Deadlock (Verklemmung) ist ein gemeinsamer Zustand parallel laufender Programme,\pause{} bei der diese sich gegenseitig blockieren.\pause{} \textcolor{gray}{Beispiel:\pause{} Prozess \(A\) benötigt die Ressource \(R_1\) um \(R_2\) fertigzustellen,\pause{} Prozess \(B\) hingegen hält \(R_1\) und benötigt \(R_2\) um sie fertigzustellen.}
            \item[ii)] Ein Histogramm bezeichnet die grafische Veranschaulichung von Häufigkeiten.
            \item[iii)] Ein Algorithmus ist deterministisch, sofern sein nächster Schritt zu jeder Zeit eindeutig ist.
        \end{enumerate}
    \end{solve}
\end{frame}

\begin{frame}[c]{Übungsaufgabe}
    \Task{Programmieren eines Häufigkeitszähler}
    \begin{exercise}<2->[Programmieren eines Häufigkeitszähler \Time{5}]
        \pause{}Schreiben Sie eine Java-Routine \bjava{countChars(String)}, welche ein Array \bjava{int[26]} zurückliefert,\pause{} das von \(0 \;\widehat{=}\; \T{a}\)\pause{} bis \(25 \;\widehat{=}\; \T{z}\) die Häufigkeit der Buchstaben angibt.\pause{} Groß- und Kleinschreibung, sowie andere Zeichen sollen dabei ignoriert werden.
    \end{exercise}
\end{frame}

\begin{frame}[c,fragile]{Lösung}
    \begin{solve}<2->[Programmieren eines Häufigkeitszähler -- Lösung]
        \begin{plainjava}
!*\onslide<3->*!public static int[] countChars(String text){
!*\onslide<4->*!    int[] counts = new int[26];
!*\onslide<5->*!    for(char c : text.toLowerCase().toCharArray()){
!*\onslide<6->*!        if(c >= 'a' && c <= 'z')
!*\onslide<6->*!            counts[c - 'a'] += 1;
!*\onslide<5->*!    }
!*\onslide<7->*!    return counts;
!*\onslide<3->*!}
        \end{plainjava}
    \end{solve}
\end{frame}


\begin{frame}[c]{Übungsaufgabe}
    \Task{Programmieren: Turtlewalk}
    \begin{exercise}<2->[Programmieren: Turtlewalk \Time{6}]
        \pause{}Das Array \bjava{static boolean[][] field} repräsentiere ein rechteckiges Gitter (welches mindestens \(1\) Feld umfasst).\pause{} Für jedes Feld \bjava{field[i][j]} gibt der Wert an, ob es passierbar (\bjava{true}) oder blockiert (\bjava{false}) ist.\pause{} Die Variablen \bjava{static int turtleX, turtleY;}\pause{} repräsentieren eine Schildkröte an Position \bjava{field[turtleY][turtleX]}.\medskip\par
        \pause{}Schreiben Sie eine Funktion \bjava{moveTurtle(int,int)}, die versucht die Schildkröte relativ um \(x\) und \(y\) zu bewegen.\medskip\par\pause{} Eine Bewegung gelingt nur, wenn das Zielfeld frei ist.\pause{} In diesem Fall gilt es, \bjava{turtleX} und \bjava{turtleY} zu modifizieren und \bjava{true} zurück zu liefern.\pause{} Andernfalls sollen die Koordinaten unverändert bleiben, sowie ein \bjava{false} zurückgegeben werden.
    \end{exercise}
\end{frame}

\begin{frame}[c,fragile]{Lösung}
    \begin{solve}<2->[Programmieren: Turtlewalk]
        \begin{plainjava}
!*\onslide<3->*!public static boolean moveTurtle(int tx, int ty){
!*\onslide<4->*!    int nx = turtleX + tx, ny = turtleY + ty;
!*\onslide<5->*!    if(nx >= field[0].length || nx < 0 // x-Richtung
!*\onslide<6->*!       || ny >= field.length || ny < 0 // y-Richtung
!*\onslide<6->*!       || !field[ny][nx]) // blockiert
!*\onslide<7->*!            return false;
!*\onslide<8->*!    turtleX = nx;
!*\onslide<8->*!    turtleY = ny;
!*\onslide<9->*!    return true;
!*\onslide<3->*!}
        \end{plainjava}
    \end{solve}
\end{frame}

\begin{frame}[c,fragile]{Übungsaufgabe}
    \Task{Programmieren: Turtlewalk, II}
    \begin{exercise}<2->[Programmieren: Turtlewalk, II \Time{6}]
        \pause{}Im Kontext der vorherigen Aufgabe existiere die Methode \bjava{moveTurtle(int tx, int ty)}.\pause{} Weiter seien nun diese Konstanten gegeben:\pause{}
\begin{plainjava}
static final int UP = 0, LEFT = 1, DOWN = 2, RIGHT = 3;
\end{plainjava}
    \pause{}Schreiben Sie eine Methode \bjava{moveTurtle(int)}, welche die relative Bewegung um \emph{so viele Felder wie möglich} vornimmt.\pause{} Die Methode soll die Anzahl der passierten Felder zurückgeben.\pause{}\medskip\par Dabei sei \say{Oben} als eine Erhöhung der \(y\)-Koordinate definiert.
    \end{exercise}
\end{frame}

\begin{frame}[c,fragile]{Lösung}
    \begin{solve}<2->[Programmieren: Turtlewalk, II]
        \begin{plainjava}
!*\onslide<3->*!public static int moveTurtle(int dir){
!*\onslide<4->*!    int tx = 0, ty = 0;
!*\onslide<5->*!    switch (dir) {
!*\onslide<6->*!        case UP: ty = 1; break; case DOWN: ty = -1; break;
!*\onslide<6->*!        case RIGHT: tx = 1; break; case LEFT: tx = -1; break;
!*\onslide<6->*!        default: return -1; // Unterscheiden von kein Feld
!*\onslide<5->*!    }
!*\onslide<7->*!    int total = 0;
!*\onslide<8->*!    while(moveTurtle(tx, ty)) { total++; }
!*\onslide<9->*!    return total;
!*\onslide<3->*!}
        \end{plainjava}
    \end{solve}
\end{frame}
\fi
% #endregion