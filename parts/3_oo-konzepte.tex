
\section{OO-Konzepte}

\subsection{Das Paradigma}
\begin{frame}{Paradigma: Objektorientierung}
    \begin{itemize}[<+(1)->]
        \item Klassen erlauben es uns Objekte der realen Welt abzubilden.
        \item Eine Klasse wie zum Beispiel \solGet{keywordC}{Person} besteht aus zwei Komponenten: \begin{description}[Funktionen]
            \item[Daten] diese, wie der Name, das Alter oder die Größe der Person,\pause{} werden als Variablen an die Klasse gebunden. Sie sind die \imp{Attribute} der Klasse und definieren den Zustand.
            \item[Funktionen] diese definieren, was eine \solGet{keywordC}{Person} kann.\pause{} Also \bjava{gehen()}, \bjava{tanzen()}, \bjava{reden()}, \ldots.\pause{} Dies sind die \imp{Methoden} der Klasse, sie definieren das Verhalten.
        \end{description}
        \item Information:\pause{} Wir nennen eine Funktion (in Java) \emph{Methode}, wenn sie an eine Objekt gebunden ist.
        \item Klassen bieten uns die Möglichkeit komplexe Probleme zu abstrahieren und Funktionalität zu kapseln.
    \end{itemize}
\end{frame}

\begin{frame}{Paradigma: Objektorientierung}
    \begin{definition}<2->[Objekt]
        Ein Objekt bezeichnet eine spezifische Ausprägung (eine sogenannte \emph{Instanz}) einer Klasse. \begin{description}[Objekt-Verhalten]
            \item[Objektzustand] wird durch die ihm zugehörigen Attribute definiert.
            \item[Objektverhalten] bezeichnet die Reaktion des Objekts auf das Aufrufen von Methoden.
            \item[Objektidentität] ermöglicht eindeutige Identifikation (in Java: Speicheradresse).
        \end{description}
    \end{definition}
    \begin{itemize}[<+(1)->]
        \item Hinweis:\pause{} Legt man die Objektorientierung streng aus,\pause{} so darf ein Objekt keinen direkten Zugriff auf seinen Zustand gestatten.\pause{} (\(\Rightarrow\) Getter \& Setter).
    \end{itemize}
\end{frame}

\begin{frame}{Programmkontext}
    \begin{itemize}[<+(1)->]
        \item Das Verhalten von Klassen wird oft auch als ein Botschaft- beziehungsweise Nachrichtsystem betrachtet.
        \item Hier bezeichnet das senden einer Botschaft\pause{} mit Inhalt das Aufrufen der Methode mit entsprechendem Inhalt.
        \item Aus Sicht der Objektorientierung ist ein \emph{Programm}\pause{} nicht mehr als das (wechselseitige) Aufrufen eben dieser Methoden.
    \end{itemize}
\end{frame}

\ifull
\begin{frame}[c]{Eine Aufgabe zwischendurch}
    \Task{Vorteile der OOP}
    \begin{exercise}<2->[Vorteile der OOP \Time{2}]
        Nenne drei Vorteile der Objektorientierten Programmierung,\pause{} neben der Möglichkeit die Realität zu abstrahieren.
    \end{exercise}
\end{frame}

\begin{frame}[c]{Lösung}
    \begin{solve}<2->[Vorteile der OOP]
        Hier angegeben werden exemplarisch vier, jeweils mit kurzer Erklärung:\begin{description}
            \item[Modularität] Eine Klasse kann unabhängig von anderen geschrieben und gehalten werden.
            \item[Informationsverdeckung] Die Implementierungsdetails des Objekts werden verborgen.
            \item[Wiederverwendbarkeit] Eine Klasse kann so geschrieben werden, dass sie auch in anderen Projekten wiederverwendet werden kann\pause{} (ein Datentyp wie ein Ringbuffer zum Beispiel).
            \item[Komposition/Polymorphie] Eine Klasse definiert eine Schnittstelle, die sie von der Implementation abstrahiert,\pause{} so dass diese sich problemlos austauschen lässt.
        \end{description}
    \end{solve}
\end{frame}
\fi

\subsection{Klassen}
\begin{frame}[c]{Vom Klassenkonzept zum Javacode}
    \begin{minipage}{0.65\linewidth}
        \begin{itemize}[<+(1)->]
            \item Es gelte eine Klasse \solGet{keywordC}{Punkt2D} zu kreieren.
            \item Ein solcher Punkt \((x,y)\) benötigt zwei Attribute: \begin{description}[y-Koordinate]
                \item[x-Koordinate] die x-Komponente (Fließ\-kom\-ma\-zahl).
                \item[y-Koordinate] die y-Komponente (Fließ\-kom\-ma\-zahl).
            \end{description}
            \item Wir möchten ein paar Dinge mit dem Punkt anstellen können: \begin{itemize}
                \item Den Punkt (relativ) verschieben.
                \item Den Abstand zu einem anderen Punkt berechnen.
                \item Den Punkt auf einen anderen Punkt setzen.
            \end{itemize}
        \end{itemize}
    \end{minipage}~\quad~\begin{minipage}{0.35\linewidth}
        \centering\resizebox{0.95\linewidth}{!}{\begin{tikzpicture}
                \umlclass[x=0,y=0]{Point2D}{
                    \bjava{private} x : \bjava{double} \\
                    \bjava{private} y : \bjava{double} \\
                }{
                    \bjava{public} distance(\solGet{keywordC}{Point2D}) : \bjava{double} \\
                    \bjava{public} shift(\bjava{double},\bjava{double}) : \bjava{void} \\
                    \bjava{public} copy(\solGet{keywordC}{Point2D}) : \bjava{void}
                }
            \end{tikzpicture}}
    \end{minipage}
\end{frame}

\begin{frame}{Implementation in Java}
    \begin{itemize}[<+(1)->]
        \item Die Implementation einer Klasse erfolgt mit der Schlüsselwort \bjava{class}.
        \item Diesem folgt der Name der Klasse,\pause{} der dem Dateinamen entsprechen \emph{muss}.\pause{} Es wird auf groß-Kleinschreibung geachtet.
        \item Innerhalb einer Klasse referenziert \bjava{this} auf das jeweilige Objekt.
        \item Bei der Implementation müssen wir den Konstruktor beachten:
    \end{itemize}
    \begin{definition}<8->[Konstruktor]
        \pause{}Ein Konstruktor verhält sich nur bedingt wie eine Methode\pause{} (er ist \emph{keine},\pause{} er gehört zum Klassenkonstrukt und kann nur mit \bjava{new} aufgerufen werden.\pause{} Allerdings kann man ihn überladen) und hat keinen Rückgabetyp.\medskip\newline
        Ein Konstruktor trägt stets den selben Namen wie die Klasse selbst und kann den Aufruf an andere Überladungen des Konstrukturs mittels \bjava{this} durchreichen.
    \end{definition}
\end{frame}

\lstset{add to literate={Point2D}{{\solIfPmode{\solGet{keywordC}{Point2D}}{Point2D}}}7}

\begin{frame}[fragile]{Implementation: Konstruktor}
\begin{plainjava}
!*\onslide<2->*!public class Point2D {
!*\onslide<2->*!
!*\onslide<3->*!    private double x, y;
!*\onslide<2->*!
!*\onslide<4->*!    // Leerer Konstruktor
!*\onslide<4->*!    public Point2D() { this(0.0,0.0); }
!*\onslide<2->*!
!*\onslide<5->*!    // Initialisiere mit Punkt
!*\onslide<5->*!    public Point2D(double x, double y) {
!*\onslide<6->*!        this.x = x;
!*\onslide<6->*!        this.y = y;
!*\onslide<5->*!    }
!*\onslide<2->*!}
\end{plainjava}
\end{frame}

\begin{frame}[fragile]{Implementation: Methoden}
\begin{itemize}[<+(1)->]
    \item Die Implementation der Methoden läuft wie bekannt!
    \item Exemplarisch das relative Verschieben:\pause{}
\begin{plainjava}
!*\onslide<4->*!public class Point2D {
!*\onslide<5->*!    //...
!*\onslide<4->*!
!*\onslide<6->*!    public void shift(double sx, double sy) {
!*\onslide<7->*!        this.x += sx;
!*\onslide<7->*!        this.y += sy;
!*\onslide<6->*!    }
!*\onslide<4->*!}
\end{plainjava}
\end{itemize}
\end{frame}

\begin{frame}[fragile]{Besondere Methoden}
\begin{itemize}[<+(1)->]
    \item Eine Klasse in Java übernimmt Methoden der \bjava{Object}-Klasse.
    \item Auf diese Weise gibt es einige wichtige Methoden,\pause{} die im Kontext von Java eine besondere Bedeutung haben\pause{} (aber dafür natürlich, wie \bjava{equals} überschrieben werden müssen).
    \item Hier sind die wichtigsten: \begin{description}[toString]
        \item[equals] die Methode \bjava{equals(Object)} prüft das Objekt mit dem Übergebenen auf \say{Gleichheit}.\pause{} So können die hierfür relevanten Attribute genau festgelegt werden.
        \item[toString] wird aufgerufen, um eine Repräsentation als Zeichenkette zu erhalten.
    \end{description}
\end{itemize}
\end{frame}

\ifull
\begin{frame}[c]{Eine Aufgabe zwischendurch}
    \Task{Eine .equals()-Methode}
    % TODO: attribute des Punktes wiederholen
    \begin{exercise}<2->[Eine .equals()-Methode \Time{6}]
        Schreibe eine \bjava{equals(Object)}-Methode für \bjava{Point2D}.\pause{} Hinweis: Denke an \bjava{instanceof} beziehungsweise \bjava{getClass()}.
    \end{exercise}
\end{frame}

\begin{frame}[c,fragile]{Lösung}
    \begin{solve}<2->[Eine .equals()-Methode]
\begin{plainjava}
!*\onslide<3->*!public boolean equals(Object obj) {
!*\onslide<4->*!    if (this == obj) return true; // Sind identisch
!*\onslide<5->*!    if (obj == null) return false; // Wir sind nicht null
!*\onslide<6->*!    // ist es von der selben Klasse?
!*\onslide<7->*!    // Oder: 'if (obj instanceof Point2D)'
!*\onslide<7->*!    if (this.getClass() != obj.getClass())
!*\onslide<7->*!        return false;
!*\onslide<8->*!    Point2D p1 = (Point2D) obj; // Es ist ein Point2D
!*\onslide<9->*!    return this.x == p1.x && this.y == p1.y;
!*\onslide<3->*!}
\end{plainjava}
    \end{solve}
\end{frame}
\fi

\begin{frame}[fragile]{Statische Methoden/Attribute}
\begin{itemize}[<+(1)->]
    \item<1-> Eine ausführliche Erklärung liegt hier: \attachPdfTextDesc{data/static.pdf}{static.pdf}{PDF-Dokument welches den Umgang mit statischen Funktionen und Methoden (in Java) genauer erklärt.}.
    \item Alle Methoden die wir so definieren sind an ein Objekt gebunden.
    \item Manche Methoden sind aber nur logisch an eine Klasse gebunden\pause{} und nicht von den Ausprägungen eines Objekts.
    \item So zum Beispiel die Mathe-Funktionen wie \bjava{Math.random()}, \bjava{Math.floor(double)}.
    \item Diese Methoden deklarieren wir mit \bjava{static},\pause{} sie sind nun auch ohne ein Objekt aufrufbar.\pause{} So kann man zum Beispiel auch eine \bjava{distance(Point2D, Point2D)} für zwei Punkte machen.
    \item Auch Variablen die für alle Objekte einer Klasse identisch sind,\pause{} können wir mit \bjava{static} deklarieren.\pause{} So zum Beispiel die Gesamtzahl der erstellten \solGet{keywordC}{Car}-Objekte.
\end{itemize}
\end{frame}

\begin{frame}[fragile]{Der Lebenszyklus eines Objekts}
    \begin{itemize}[<+(1)->]
        \item Mit dem Erstellen eines Objekts,\pause{} haben wir eine Variable, die auf es zeigt.
        \item Durch weitere Zuweisungen\pause{} oder Methodenaufrufe können wir weitere Zeiger darauf erstellen.
        \item Überschreiben wir diese Variablen, oder verlassen ihren Gültigkeitsbereich,\pause{} verlieren wir einen Zeiger.
        \item Wenn keine Variable mehr auf das Objekt zeigt,\pause{} gibt es (zumindest an sich) keine Möglichkeit mehr auf das Objekt zuzugreifen.
        \item In diesem Fall wird es (irgendwann) vom Garbage-Collector aufgeräumt
    \end{itemize}
\ifull
    \Task{Garbage Collector}
    \begin{exercise}<8->[Garbage Collector \Time{2}]
        \pause{}Kann der Java Garbage Collector direkt beeinflusst werden? Was sind die Möglichkeiten?
    \end{exercise}
\fi\onslide<1->
\end{frame}

\ifull
\begin{frame}[c]{Lösung}
    \begin{solve}<2->[Garbage Collector]
        Wir können den Collector nicht direkt beeinflussen!\pause{} Allerdings können wir nicht mehr benötigte Objekte auf \bjava{null} setzen\pause{} und mittels \bjava{System.gc()} den Prozess anstoßen.\pause{}\medskip\newline
        Information: Der GC räumt nicht nur Speicher auf sondern füllt auch Speicherlücken.\pause{} Es gibt verschiedene Varianten für einen GC (Referencecount,\pause{} Mark \& Sweep,\pause{} Stop \& Copy).\pause{} Wird ein Objekt aufgeräumt,\pause{} ruft Java die Methode \bjava{finalize()} auf.
    \end{solve}
\end{frame}
\fi

\subsection{Enumerationen}
\begin{frame}[fragile]{Enumerationen}
    \begin{itemize}[<+(1)->]
        \widei
        \item Mit Java-5 gibt es die Möglichkeit durch \imp{Enumerationen} eigene Datentypen zu definieren.
        \item Sie können überall dort definiert werden,\pause{} wo man auch eine Klasse definieren kann.\pause{} \textit{Technisch betrachtet sind Enumerationen spezielle Java-Klassen.}
        \item Die einfachste Möglichkeit eine Enumeration zu definieren,\pause{} funktioniert über das \bjava{enum}-Schlüsselwort:\pause{}
\begin{plainjava}
enum :lan:Name der Enumeration:ran: {
    :lan:Komma separierte Liste an Werten:ran:
}
\end{plainjava}
    \end{itemize}
\end{frame}

\begin{frame}[fragile]{Enumerationen definieren}
    \begin{itemize}[<+(1)->]
        \widei
        \item Der Konvention nach, werden alle Werte einer Enumeration in Großbuchstaben\pause{} und mit Unterstrichen geschrieben.
        \item Betrachten wir ein Beispiel:\pause{}
\begin{plainjava}
enum Richtung {
    HOCH, RUNTER, LINKS, RECHTS
}
\end{plainjava}
        \item Wir können diese Typen als Parametertypen in Methoden verwenden,\pause{} und über die Punkt-Syntax auf Elemente zugreifen.
        \item Hinweis:\pause{} Anders als in Sprachen wie C++,\pause{} wird den Konstanten kein (Integer)-Wert zugeordnet.\pause{} (dafür gibt es \bjava{ordinal()})
    \end{itemize}
\end{frame}

\begin{frame}[fragile]{Enumerationen verwenden}
    \begin{itemize}[<+(1)->]
        \widei
        \item Betrachten wir ein Beispiel mit \bjava{Richtung} und \bjava{switch-case}:\pause{}
\begin{plainjava}
!*\onslide<3->*!public static String wohinGehtEs(Richtung ziel){
!*\onslide<4->*!    switch(ziel) {
!*\onslide<5->*!        case HOCH:   return "Es geht nach oben!";
!*\onslide<6->*!        case RUNTER: return "Es geht nach unten!";
!*\onslide<7->*!        case LINKS:  return "Es geht nach links!";
!*\onslide<8->*!        case RECHTS: return "Es geht nach rechts!";
!*\onslide<4->*!    }
!*\onslide<9->*!    return "Fehler! Richtung unbekannt";
!*\onslide<3->*!}
\end{plainjava}
    \item<10-> Beispielhafte Verwendung:\onslide<11->
\begin{plainjava}
System.out.println(wohinGehtEs(Richtung.RECHTS));
\end{plainjava}
    \end{itemize}
\end{frame}

\begin{frame}[fragile]{Was Enumerationen liefern}
    \begin{itemize}[<+(1)->]
        \widei
        \item Mittels \bjava{:lan:Enum Name:ran:.values()} erhalten wir ein Array aller Werte.
        \item \bjava{:lan:Enum Name:ran:.valueOf(String)} liefert die Enumkonstante mit übergebenem Namen,\pause{} sofern vorhanden.
        \item Enumerationen besitzen ein wie zu erwarten funktionierendes \bjava{equals(Object)}.
        \item Analog funktioniert \bjava{toString()} wie zu erwarten.
        \item Da es sich auch um Klassen handelt, können wir den Konstanten Datentypen zuordnen!
    \end{itemize}
\end{frame}

\begin{frame}[fragile,c]{Alternative Richtungs-Enum}
\begin{plainjava}
!*\onslide<2->*!enum Richtung {
!*\onslide<6->*!    HOCH("Es geht nach oben!"),
!*\onslide<7->*!    RUNTER("Es geht nach unten!"),
!*\onslide<8->*!    LINKS("Es geht nach links!"),
!*\onslide<9->*!    RECHTS("Es geht nach rechts!");
!*\onslide<2->*!
!*\onslide<3->*!    private String text;
!*\onslide<4->*!    public String getText() { return this.text; }
!*\onslide<2->*!
!*\onslide<5->*!    Richtung(String text) {
!*\onslide<5->*!        this.text = text;
!*\onslide<5->*!    }
!*\onslide<2->*!}
\end{plainjava}
\end{frame}

\begin{frame}[fragile]{Abschluss zu Enumerationen}
    \begin{itemize}[<+(1)->]
        \widei
        \item Die Methode \bjava{wohinGehtEs} lässt sich nun kompakter fassen:\pause{}
\begin{plainjava}
public static String wohinGehtEs(Richtung ziel){
    return ziel.getText();
}
\end{plainjava}
        \item Das Kompilieren einer Datei mit Enumeration erzeugt eine weitere \T{.class}-Datei,\pause{} deren Namen dem Schema \bvoid{:lan:Klassennamen:ran:\$:lan:Enumname:ran:.class} folgt.
    \end{itemize}
\end{frame}


% Füge Gültigkeitsbereiche hinzu
\subsection{call-by-value/reference}
\begin{frame}[fragile]{Klassen als Parameter: call-by-reference}
    \begin{itemize}[<+(1)->]
        \widei
        \item \hypertarget<1>{mrk:call-by-ref}{Wenn} wir komplexe Datentypen als Parameter übergeben oder zuweisen,\pause{} wird \emph{keine} Kopie des Objekts erstellt,\pause{} sondern eine Referenz auf das Objekt übergeben.
        \item Dieser (Parameter-)Übergabemechanismus wird \emph{call-by-reference} genannt.
        \item Für eine genauere Betrachtung hilft der Foliensatz zu Übungsblatt \(7\).
    \end{itemize}
\end{frame}

\ifull
\begin{frame}[fragile,c]{Eine leichte Aufgabe als Einstieg}
    \Task{Übung zu call-by-reference}
    \begin{exercise}<2->[Was liefert dieser Code? \Time{1}]
        \begin{plainjava}
!*\onslide<3->*!public static void multBy2(int[] x){
!*\onslide<3->*!    x[0] = x[0]*2;
!*\onslide<3->*!}
!*\onslide<3->*!
!*\onslide<4->*!public static void main(String[] args) {
!*\onslide<5->*!    int[] arr = { 21 };
!*\onslide<5->*!    System.out.println(arr[0]);
!*\onslide<5->*!    multBy2(arr);
!*\onslide<5->*!    System.out.println(arr[0]);
!*\onslide<4->*!}
        \end{plainjava}
    \end{exercise}\onslide<1->
\end{frame}


\long\def\PresentHS<#1>#2#3{%
\onslide<#1>{\resizebox{4.865cm}{!}{\begin{tikzpicture}
\begin{heap-n-stack}{#2}
#3
\end{heap-n-stack}
\end{tikzpicture}}}%
}
\begin{frame}[fragile,c]{Eine leichte Aufgabe als Einstieg -- Lösung}
    \begin{solve}<2->[Was liefert dieser Code?]
        \pause{}Der Code liefert \(21\) und \(42\). Ersterer Wert, da wir das Array ja so zuweisen,\pause{} den zweiten, weil durch die Übergabe des Arrays mittels \emph{call-by-reference} auch das Ursprungsarray verändert wird.
        \begin{center}
\PresentHS<4->{1}{%
\istack{main}
\stack{arr:}
\renderstack

\iheap{Globales}
\heap{\T{\{ 21 \}}}
\renderheap
\draw[lhns] (stack-1) -- (stack-1-|heap-1-box.west);
}\qquad\PresentHS<5->{2}{%
\istack{main}
\stack{arr:}
\istack{multBy2}
\stack{x:\phantom{rr}}
\renderstack

\iheap{Globales}
\heap{\T{\{ 21 \}}}
\renderheap
\draw[lhns] (stack-1) -- (stack-1-|heap-1-box.west) coordinate (r1);
\draw[lhns] (stack-3) -- ++(1.25,0) -- ([xshift=-0.75cm]r1) -- (r1);
}
        \end{center}
    \end{solve}
\end{frame}


\begin{frame}[fragile,c]{Eine leichte Aufgabe als Einstieg -- Lösung}
    \addtocounter{solve}{-1}%
    \begin{solve}<1->[Was liefert dieser Code?\hfill{}(Fortsetzung)]
        \pause{}\begin{center}
\PresentHS<2->{3}{%
\istack{main}
\stack{arr:}
\istack{multBy2}
\stack{x:\phantom{rr}}
\renderstack

\iheap{Globales}
\heap{\T{\{ 42 \}}}
\renderheap
\draw[lhns] (stack-1) -- (stack-1-|heap-1-box.west) coordinate (r1);
\draw[lhns] (stack-3) -- ++(1.25,0) -- ([xshift=-0.75cm]r1) -- (r1);
}\qquad\PresentHS<3->{4}{%
\istack{main}
\stack{arr:}
\renderstack

\iheap{Globales}
\heap{\T{\{ 42 \}}}
\renderheap
\draw[lhns] (stack-1) -- (stack-1-|heap-1-box.west) coordinate (r1);
}
        \end{center}
        \onslide<4->{\textit{Hinweis: eine derartige Grafik wird nicht gefordert werden, sie kann aber durchaus beim Verständnis der Thematik hilfreich sein.}}
    \end{solve}
\end{frame}
\fi

\begin{frame}[fragile]{Der nette Bruder: call-by-value}
    \begin{itemize}[<+(1)->]
        \widei
        \item \hypertarget<1>{mrk:call-by-val}{Bei} Klassen und Arrays wird nun also nicht das Element selbst,\pause{} sondern nur die Referenz übergeben.
        \item Bei primitiven Datentypen hingegen wird der Wert \emph{kopiert},\pause{} wan übergibt also den eigentlichen Wert der Variable.
        \item Man sagt auch, dass Methoden mit \emph{call-by-value} \say{keine} (nach bisheriger Betrachtung) Seiteneffekte hervorrufen,\pause{} da sie die übergebenen Daten nicht modifizieren.
        \item Wenn man auch komplexe Datentypen kopieren möchte,\pause{} so erstellt man sie in der Regel neu (\bjava{clone()}) und greift auf die \emph{call-by-value}-Charakteristik der primitiven Datentypen zurück.
    \end{itemize}
\end{frame}

\fullsubsection{Übungsaufgaben}
\ifull
\begin{frame}[c,fragile]{Übungsaufgabe}
    \Task{Suche syntaktischer Fehler, IV}
    \begin{exercise}<2->[Fehler finden, IV \Time{3}]
        \pause{}Finde und korrigiere alle (syntaktischen) Fehler:\pause{}
        \begin{plainvoid}
class Laenge {
    final double METER KILOMETER = 1_000;
    static final float meter2kilometer(final int meter){
        return meter/METER KILOMETER;
    }
    static int kilometer2meter(double kilometer){
        return kilometer * METER KILOMETER;
    }
}
        \end{plainvoid}
    \end{exercise}
\end{frame}

\begin{frame}[c,fragile]{Lösung}
    \begin{solve}<2->[Fehler finden, IV]
        \pause{}\begin{plainjava}
class Laenge {
    static final double METER_KILOMETER = 1!*\solGet{numbers}{\_}*!000;
    static final float meter2kilometer(final int meter){
        return (float) (meter/METER_KILOMETER);
    }
    static int kilometer2meter(double kilometer){
        return (int) (kilometer * METER_KILOMETER);
    }
}
        \end{plainjava}
    \end{solve}
\end{frame}


\begin{frame}[c,fragile]{Lösung}
    \addtocounter{solve}{-1}
    \begin{solve}<1->[Fehler finden, IV\hfill{}(Fortsetzung)]
    \begin{enumerate}[<+(1)->]
        \item  Die Konstante \bjava{METER KILOMETER} darf kein Leerfeld enthalten und muss \bjava{static} sein.
        \item In \bjava{meter2kilometer} und \bjava{kilometer2meter} muss eine explizite Konvertierung erfolgen.
    \end{enumerate}
    \pause{}\textit{Hinweis: das \bjava{final} in den Parametern und der Unterstrich in \bjava{1000} sind keine Fehler!}
    \end{solve}
\end{frame}

\begin{frame}[c,fragile]{Übungsaufgabe}
    \Task{Eine Kreisklasse}
    \begin{exercise}<2->[Kreisklasse \Time{4}]
        \pause{}Schreibe eine Klasse \bjava{Circle} die einen Kreis repräsentiert. Ein Kreis besitzt eine \(x\) und eine \(y\) Koordinate im Fließkommabereich, sowie einen Radius.\pause{}
        Es soll möglich sein Umfang sowie Fläche des Kreises berechnen zu lassen.\pause{}
        Auf Datenkapselung muss hierbei keine Rücksicht genommen werden,\pause{} allerdings muss es möglich sein den Kreis unter der Angabe aller Attribute zu konstruieren.\pause{} (\(U = 2\cdot r\cdot\pi \), \(A = r^2\cdot\pi\))
    \end{exercise}
\end{frame}

\begin{frame}[c,fragile]{Lösung}
    \begin{solve}<2->[Kreisklasse]
        \pause{}\begin{plainjava}
!*\onslide<3->*!class Circle {
!*\onslide<4->*!    public double x, y, r;
!*\onslide<5->*!    public Circle(double _x, double _y, double _r) {
!*\onslide<5->*!        x = _x; y = _y; r = _r;
!*\onslide<5->*!    }
!*\onslide<6->*!    public double getCircumference(){
!*\onslide<6->*!        return 2 * r * Math.PI;
!*\onslide<6->*!    }
!*\onslide<3->*!
!*\onslide<7->*!    public double getArea() { return r * r * Math.PI; }
!*\onslide<3->*!}
        \end{plainjava}
    \end{solve}
\end{frame}


\begin{frame}[c,fragile]{Übungsaufgabe}
    \Task{Ein Schienennetz}
    \begin{exercise}<2->[Schienennetz \Time{4}]
        \onslide<3->{Ein Feld in einem gitterartigen Schienennetz kann entweder \onslide<4->{frei (\(\widehat{=}\) keine Schiene), von einer Schiene besetzt, von einem Waggon oder einer Lokomotive belegt sein.\par{}}}
        \onslide<5->{Schreibe eine Enumeration die einen Feldzustand darstellt,\onslide<6->{ jede Konstante soll gleich mitspeichern, ob das Feld von einer Lok befahrbar ist, \onslide<7->{wobei dies nur für \say{Schienenfelder} der Fall ist.}}}
    \end{exercise}
\end{frame}

\begin{frame}[c,fragile]{Lösung}
    \begin{solve}<2->[Schienennetz]
        \begin{plainjava}
!*\onslide<3->*!enum Schiene {
!*\onslide<4->*!    FREI(false),
!*\onslide<5->*!    SCHIENE(true),
!*\onslide<6->*!    WAGGON(false),
!*\onslide<7->*!    LOKOMOTIVE(false);
!*\onslide<3->*!
!*\onslide<8->*!    boolean befahrbar;
!*\onslide<9->*!    Schiene(boolean befahrbar) {
!*\onslide<9->*!        this.befahrbar = befahrbar;
!*\onslide<9->*!    }
!*\onslide<3->*!}
        \end{plainjava}
    \end{solve}
\end{frame}

\begin{frame}[c,fragile]{Übungsaufgabe}
    \Task{Ein Kamera-Singleton erstellen.}
    \begin{exercise}<2->[Kamera \Time{5}]
        \onslide<3->{Schreibe eine \bjava{Camera} Klasse, \onslide<4->{mit drei Fließkommazahlen für die aktuelle Position (\(x,y,z\)) \onslide<5->{und zwei für den jeweiligen Drehwinkel \(r_x, r_z\). \onslide<6->{Von der Klasse soll es nicht möglich sein mehr als eine Instanz zu erzeugen. \onslide<7->{Weitere \say{Anfragen} sollen dasselbe Objekt zurückliefern.}}}}}
    \end{exercise}
\end{frame}

\begin{frame}[c,fragile]{Lösung}
    \begin{solve}<2->[Kamera (Singleton)]
        \begin{plainjava}
!*\onslide<3->*!public class Camera {
!*\onslide<4->*!    double x, y, z, rx, rz;
!*\onslide<6->*!    private static Camera instance;
!*\onslide<5->*!    private Camera() {
!*\onslide<5->*!        x = y = z = rx = rz = 0.0;
!*\onslide<5->*!    }
!*\onslide<7->*!    public static Camera getInstance() {
!*\onslide<8->*!        if(instance == null) instance = new Camera();
!*\onslide<9->*!        return instance;
!*\onslide<7->*!    }!*\onslide<10->*! // ...
!*\onslide<3->*!}
        \end{plainjava}
    \end{solve}
\end{frame}
\fi