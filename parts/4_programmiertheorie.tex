\section{Programmiertheorie}
\subsection{Rekursion}
\begin{frame}{Rekursive und iterative Programmierung}
    \begin{definition}<2->[Rekursive Methode]
        \onslide<3->{Eine Methode oder Funktion bezeichnen wir als \emph{rekursiv}, wenn sie sich:} \begin{itemize}
            \item<4-> \emph{direkt} selbst aufruft, sich also selbst referenziert.
            \item<5-> \emph{indirekt} selbst aufruft, also eine andere Methode verwendet, die (über irgendwelche Umwege) wieder die Methode aufruft.
        \end{itemize}
    \end{definition}
    \begin{itemize}
        \widei
        \item<6-> Eine rekursive Methode lässt sich in zwei Komponenten gliedern: \begin{description}[Abbruchbedingung]
            \item<7->[Abbruchbedingung] Das lösbare Teilproblem. Ende der Rekursion. Hier wird die Methode nicht weiter referenziert.
            \item<8->[Rekursiver Zweig] Enthält die rekursiv auszuführende Prozedur.
        \end{description}
        \item<9-> Rekursion und Iteration sind \textit{gleichmächtig}. Iteration ist in der Regel \say{performanter}, Rekursion ist oft \say{eleganter} (rekursive Datenstrukturen,~\ldots).
    \end{itemize}
\end{frame}

\begin{frame}[fragile]{Arten von Rekursion: Head und Tail}
    \begin{itemize}[<+(1)->]
        \widei
        \item Wir unterscheiden (unter anderem) zwei besondere Rekursionen: \begin{description}
            \item[Head] Der rekursive Aufruf erfolgt zu Beginn: \say{Alles passiert im Aufstieg.}
            \item[Tail] Der rekursive Aufruf erfolgt zu Ende der Rekursion.\pause{} Ein solcher Abstieg kann einfach in eine Iteration transformiert werden (z.B. durch den Compiler): \say{Alles passiert im Abstieg.}
        \end{description}
    \end{itemize}
\begin{center}\scriptsize%
\begin{minipage}{.45\linewidth}
\begin{plainjava}
!*\onslide<11->*!// :yields: 1 2 3 4 5 6 :ldots:
!*\onslide<6->*!public void headDecrement(int i){
!*\onslide<7->*!    // Abbruchbedingung
!*\onslide<7->*!    if(i == 0) return;
!*\onslide<8->*!    // Rekursion
!*\onslide<8->*!    else headDecrement(i - 1);
!*\onslide<10->*!    // Verarbeitung
!*\onslide<10->*!    System.out.println(i);
!*\onslide<6->*!}
\end{plainjava}
\end{minipage}\qquad\begin{minipage}{.45\linewidth}
\begin{plainjava}
!*\onslide<11->*!// :yields: 10 9 8 7 6 :ldots:
!*\onslide<6->*!public void tailDecrement(int i){
!*\onslide<7->*!    // Abbruchbedingung
!*\onslide<7->*!    if(i == 0) return;
!*\onslide<9->*!    // Verarbeitung
!*\onslide<9->*!    else System.out.println(i);
!*\onslide<10->*!    // Rekursion
!*\onslide<10->*!    tailDecrement(i - 1);
!*\onslide<6->*!}
\end{plainjava}
\end{minipage}
\end{center}
\end{frame}

\begin{frame}[fragile]{Arten von Rekursion: Baumartige Rekursion}
    \begin{itemize}[<+(1)->]
        \widei
        \item Eine Methode kann sich auch mehrfach (auch indirekt) selbst referenzieren.
        \item In diesem Fall entsteht kein \say{linearer} Abstieg,\pause{} sondern vielmehr eine baum-/kaskadenartige Verzweigung.
        \item Beliebte Beispiele: die Fibonaccifolge und der Binomialkoeffizient.
    \end{itemize}
    \begin{uncoverenv}<6->%
\vfill\null
\begin{minipage}{.65\linewidth}
\begin{plainjava}
int fib(int n) {
    if(n <= 1) return 1;
    return fib(n - 1) + fib(n - 2);
}
\end{plainjava}
\end{minipage}\hfill\onslide<7->{\raisebox{-.45\height}{\scalebox{.75}{\begin{forest}
[f(12)
    [f(11) [f(10) [\ldots][\ldots]] [f(9) [\ldots][\ldots]] ]
    [f(10) [\ldots][\ldots]]
]
\end{forest}
    }}}
\end{uncoverenv}\vfill\null
\end{frame}


\begin{frame}[fragile]{Verschränkte und geschachtelte Rekursion}
    \begin{itemize}[<+(1)->]
        \widei
        \item Es gibt weitere Arten der Rekursion: \begin{description}[geschachtelt]
            \widei
            \item[geschachtelt] Hier ist (mindestens) ein Parameter im rekursiven Aufruf selbst ein rekursiver Aufruf.\pause{} Beispielsweise die Ackermannfunktion (nach Péter):
\begin{plainjava}[language=xJava]
!*\onslide<5->*!int a(int n, int m) {
!*\onslide<5->*!    if(n == 0) return m + 1;
!*\onslide<5->*!    else if (m == 0) return a(n - 1, 1);
!*\onslide<5->*!    else return a(n - 1, |zws|a(n, m - 1)|zws|);
!*\onslide<5->*!}
\end{plainjava}
            \item<6->[verschränkt] Hier rufen sich mehrere Funktionen rekursiv gegenseitig auf.\pause{} Diese Variante lässt sich schwer bis gar nicht (direkt) in eine Schleife übersetzten.
        \end{description}
    \end{itemize}
\end{frame}

\ifull
% \begin{frame}[c]{Übung zu Rekursionstransformationen}
%     \Task{Head- und Tail-Rekursion}%
%     \begin{exercise}<2->[Head- und Tail-Rekursion \Time{5}]
%         \onslide<3->{}Schreiben Sie eine Java-Methode der Gestalt \bjava{int fib(int n)} welche die \(n\)-te Fibonacci-Zahl berechnet (\(f(0) = f(1) = 1\), \(f(n) = f(n - 1) + f(n - 2)\)). Verwenden Sie dabei ausschließlich Tail-Rekursion um das Problem zu lösen. Iterationen oder andere Rekursionsarten sind nicht gestattet. Hilfsmethoden sind erlaubt.
%     \end{exercise}
% \end{frame}

% \begin{frame}[c,fragile]{Lösung}
%     \begin{solve}<2->[Head- und Tail-Rekursion]
%         Im Allgemeinen ist es nicht leicht, jede Rekursion Tail-Rekursiv zu schreiben.
%     \end{solve}
% \end{frame}

\begingroup
\def\f(#1,#2,#3){f(#1,\;#2,\;#3)}%
\begin{frame}[c]{Etwas Rekursion}
    \Task{Formel zu Rekursion}%
    \begin{exercise}<2->[Formel zu Rekursion \Time{4}]
        \onslide<3->{Implementieren Sie eine rekursive Java-Methode, welche die folgende Funktion berechnet (\(a, b, c \in \mathbb{N}\)).
\begin{equation*}
    \f(a, b, c) = \begin{cases}
        \max(a,b) & \text{ wenn } c = 0, \\
        \f(a + b, b - 1, c) & \text{ wenn } b \geq 0, \\
        \f(a - 1, b - 1, |c - 1|) & \text{ sonst.}
    \end{cases}
\end{equation*}}%
    \onslide<4->{Approximieren Sie \(\mathbb{N}\) durch den Java Datentyp \bjava{long}. Sie dürfen \emph{keine} bestehenden Java Funktionen (wie beispielsweise \bjava{Math::abs}) benutzen.\medskip\par
    Welchen Wert liefert die Berechnung von \(f(12, 3, -2)\)?}
    \end{exercise}
\end{frame}

\begin{frame}[c,fragile]{Lösung}
    \begin{solve}<2->[Formel zu Rekursion]
\begin{plainjava}
!*\onslide<3->*!public long f(long a, long b, long c) {
!*\onslide<4->*!    if(c == 0)
!*\onslide<4->*!        return a > b ? a : b;
!*\onslide<5->*!    if(b >= 0)
!*\onslide<5->*!        return f(a + b, b - 1, c);
!*\onslide<6->*!    return f(a - 1, b - 1 , c > 0 ? (c - 1) : -(c - 1));
!*\onslide<3->*!}
\end{plainjava}
    \onslide<7->{Weiter gilt \(\f(12, 3, -2) = 14\).}
    \onslide<8->{\textcolor{gray}{Warum?}}
    \end{solve}
\end{frame}

\begin{frame}[c]{Lösung}
    \addtocounter{solve}{-1}%
    \begin{solve}<2->[Formel zu Rekursion\hfill(Fortsetzung)]
        \pause{}\pause{}Für \(\f(12, 3, -2)\) gilt mit \(c = -2 \neq 0\) und \(b = 3 \geq 0\) die Berechnung: \(\f(a + b, b - 1, c)\). Damit erhalten wir:\pause{}
\begin{equation*}
    \f(12 + 3, 3 - 1, -2) = \f(15, 2, -2)
\end{equation*}
        \pause{}Damit ist wieder  \(c = -2 \neq 0\) und \(b = 2 \geq 0\). Daraus ergibt sich:\pause{}
\begin{align*}
    \f(15 + 2, 2 - 1, -2) & = \f(17, 1, -2)\\
    \f(17 + 1, 1 - 1, -2) &= \f(18, 0, -2)\\
    \f(18 + 0, 0 - 1, -2) &= \f(18, -1, -2)
\end{align*}
        \pause{}Mit \(b = -1 \not\geq 0\) betreten wir nun den \say{sonst}-Teil.
    \end{solve}
\end{frame}

\begin{frame}[c]{Lösung}
    \addtocounter{solve}{-1}%
    \begin{solve}<2->[Formel zu Rekursion\hfill(Fortsetzung)]
        \pause{}\pause{}Nun bei \(\f(18, -1, -2)\) berechnen wir \(\f(a - 1, b -1, |c - 1|)\):\pause{}
\begin{align*}
    \f(18 - 1, -1 - 1, |-2 - 1 |) &= \f(17, -2, 3) \\
    \f(17 - 1, -2 - 1, |3 - 1 |) &= \f(16, -3, 2) \\
    \f(16 - 1, -3 - 1, |2 - 1 |) &= \f(15, -4, 1) \\
    \f(15 - 1, -4 - 1, |1 - 1 |) &= \f(14, -5, 0)
\end{align*}
    \pause{}Mit \(c = 0\) terminiert die Funktion für \(\max(14, -5) = 14\).
    \end{solve}
\end{frame}
\endgroup
\fi

\begin{frame}{Das Paradigma: Divide and Conquer}
    \begin{itemize}[<+(1)->]
        \widei
        \item Manche komplizierte Probleme,\pause{} lassen sich durch Rekursion in immer kleinere Probleme aufspalten, die dann beherrschbarer sind.
        \item Dies wird uns bei den Sortieralgorithmen \hyperlink{mrk:sort-Mergesort}{Merge-} und \linksort{Quicksort} wieder begegnen.
    \end{itemize}
\end{frame}


\begin{frame}{Das Paradigma: Backtracking}
    \begin{itemize}[<+(1)->]
        \widei
        \item Backtracking ist eine rekursive Lösungsstrategie.
        \item Das Problem wird von einer Teillösung aus bis zur Gesamtlösung erweitert.\pause{} Eine \say{Sackgasse} veranlasst einen neuen Versuch (trial and error).
        \item Im Sackgassen-Fall macht man die Entscheidungen solange rückgängig, bis man eine andere Erweiterung wählen kann und erweitert die Teillösung dann durch diese.
    \end{itemize}
\end{frame}


\begin{frame}{Das Paradigma: Backtracking, Beispiele}
    \begin{itemize}[<+(1)->]
        \widei
        \item \emph{Wegfindung im Labyrinth:}\pause{} Gehe vom Startfeld aus solange einen Weg entlang, bis am Ziel angekommen.\pause{} Im Fall einer Sackgasse: springe zur letzten Position zurück, an der es noch einen anderen Weg gab.
        \item \emph{Lösung eines Sudoku.}\pause{} Füge in erstes freies Feld eine der dort möglichen Zahlen ein.\pause{} Verfahre so, bis alle Felder gefüllt sind oder für ein Feld keine Zahl mehr möglich ist.\pause{} In diesem Fall: springe zum letzten Punkt zurück, an dem noch andere Zahlen möglich sind. Probiere weiter.
        \item Rucksackproblem, 8-Damen Problem,~\ldots
    \end{itemize}
\end{frame}


\immediate\write18{wget https://media.githubusercontent.com/media/EagleoutIce/Episode-Recursion/gh-pages/preview-01.png -O logoRecursion-\jobname.png}
\begin{frame}[c]{Zur Vertiefung}
\centering\vspace*{2em}\par\begin{tikzpicture}
    \onslide<2->{\draw[thick,@primary,rounded corners=2.5pt,path picture={\node at(path picture bounding box.center) {\href{https://media.githubusercontent.com/media/EagleoutIce/Episode-Recursion/gh-pages/noanim_rekursion.pdf}{\includegraphics[width=8.5cm,height=4.788cm,keepaspectratio]{logoRecursion-\jobname.png}}};}] (0,0) rectangle (8.5cm,4.788cm);}
    \onslide<3->{\node[below=2mm] at (current bounding box.south) {Mehr zum Thema \say{Rekursion} per Klick\ldots};}
    \onslide<3->{\node[right=1cm,yshift=4.5mm,scale=1.125] at (current bounding box.east) {\fancyqr[link]{https://github.com/EagleoutIce/Episode-Recursion/tree/main}};}
\end{tikzpicture}
\end{frame}

\subsection{Laufzeitkomplexität}
\begin{frame}{Komplexitätsbetrachtung}
    \begin{itemize}[<+(1)->]
        \widei
        \item Da die Ausführungszeit eines Programms von vielen Parametern\pause{} \textcolor{gray}{(Taktrate des Prozessors, andere laufende Programme,~\ldots)} abhängig ist,\pause{} betrachten wir oft nur dessen Skalierung.
    \end{itemize}
    \begin{definition}<5->[Effizienz]
        \onslide<6->{Die Effizienz eines Programms wird durch dessen \emph{Speicher-}, sowie \emph{Laufzeitaufwand} bestimmt.}
    \end{definition}
    \begin{itemize}
        \widei
        \item<7-> Letztere werden wir ausführlicher betrachten. Genauer: In welcher Komplexitätsklasse liegt der Algorithmus?
    \end{itemize}
\end{frame}

\begin{frame}{Komplexitätsbetrachtung}
    \begin{itemize}[<+(1)->]
        \widei
        \item Für die Laufzeitkomplexität unterscheidet man: \begin{description}[average-case]
            \item[worst-case] Die Laufzeitkomplexität im schlechtesten Fall für den (spezifischen) Algorithmus.
            \item[best-case] Die Laufzeitkomplexität im günstigsten Fall für den (spezifischen) Algorithmus.
            \item[\color{gray}average-case] \textcolor{gray}{Die Laufzeitkomplexität im durchschnittlichen Fall für den (spezifischen) Algorithmus.\pause{} Dies bezeichnet in der Regel gleichverteilt zufällige Eingaben.}
        \end{description}
        \item Dabei werden wir den \textit{average-case} vernachlässigen.
    \end{itemize}
\end{frame}

\begin{frame}[fragile]{Erfassen der Komplexität -- Präzise}
    \begin{itemize}[<+(1)->]
        \widei
        \item Die Erfassung der Laufzeitkomplexität erfolgt durch die Auflistung der notwendigen (Rechen-)Schritte:
\begin{onlyenv}<3|handout:0>
\begin{plainjava}[language=xJava]
static int methode(int n) {
    int count = 2;
    for(int i = 1; i <= n; i++) {
        for(int j = n; j > i; j--)
            count++;
    }
    return count;
}
\end{plainjava}
\end{onlyenv}\begin{onlyenv}<4-5|handout:0>
\begin{plainjava}[language=xJava]
static int methode(int n) {
    !**!|zws|int count = 2;|zws|
    for(|zws|int i = 1;|zws| i <= n; i++) {
        for(|zws|int j = n;|zws| j > i; j--)
            count++;
    }
    return count;
}
\end{plainjava}
\end{onlyenv}\begin{onlyenv}<6-7|handout:0>
\begin{plainjava}[language=xJava]
static int methode(int n) {
    !**!|zws|int count = 2;|zws|
    for(|zws|int i = 1;|zws| |vgl|i <= n;|vgl| i++) {
        for(|zws|int j = n;|zws| |vgl|j > i;|vgl| j--)
            count++;
    }
    return count;
}
\end{plainjava}
\end{onlyenv}\begin{onlyenv}<8-9|handout:0>
\begin{plainjava}[language=xJava]
static int methode(int n) {
    !**!|zws|int count = 2;|zws|
    for(|zws|int i = 1;|zws| |vgl|i <= n;|vgl| |inc|i++|inc|) {
        for(|zws|int j = n;|zws| |vgl|j > i;|vgl| j--)
            !**!|inc|count++;|inc|
    }
    return count;
}
\end{plainjava}
\end{onlyenv}\begin{onlyenv}<10-11|handout:1>
\begin{plainjava}[language=xJava]
static int methode(int n) {
    !**!|zws|int count = 2;|zws|
    for(|zws|int i = 1;|zws| |vgl|i <= n;|vgl| |inc|i++|inc|) {
        for(|zws|int j = n;|zws| |vgl|j > i;|vgl| |dec|j--|dec|)
            !**!|inc|count++;|inc|
    }
    return count;
}
\end{plainjava}
\end{onlyenv}
    \end{itemize}
\vspace*{-.75\baselineskip}
\begin{multicols}{2}
    \begin{itemize}
        \item<4-> \bjava[language=xJava]{|zws|Zuweisungen|zws|}\,: \onslide<5->{\(2+n\)}
        \item<6-> \bjava[language=xJava]{|vgl|Vergleiche|vgl|}\,: \onslide<7->{\(n+1+\frac{n(n+1)}{2}\)}
        \item<8-> \bjava[language=xJava]{|inc|Inkrementierungen|inc|}\,: \onslide<9->{\(n+\frac{n(n-1)}{2}\)}
        \item<10-> \bjava[language=xJava]{|dec|Dekrementierungen|dec|}\,: \onslide<11->{\(\frac{n(n-1)}{2}\)}
    \end{itemize}
\end{multicols}
\end{frame}

\begin{frame}{Erfassen der Komplexität}
    \begin{itemize}[<+(1)->]
        \item Insgesamt ergibt sich damit ein Aufwand von \(\frac{3n^2}{2} + \frac{5n}{2} + 3\).
        \item Da für große Datenmengen Konstanten und Faktoren irrelevant werden, interessiert wie die Funktion skaliert/wächst.
    \end{itemize}
    \begin{definition}<4->[\O-Notation]
        \onslide<5->{Es gilt \(T(n) \in \O(f(n))\), wenn \(f(n)\) eine obere Schranke von \(T(n)\) ist, also:} \onslide<6->{\[T(n) \in \O(f(n))\pause{}\iff \exists n_0 \in \mathbb{N}\: c \in \mathbb{R}^+\: \forall n \geq n_0: T(n) \leq c \cdot f(n).\vspace*{-\topskip}\]}
    \end{definition}
\end{frame}

\begin{frame}{Erfassen der Komplexität}
    \begin{itemize}[<+(1)->]
        \widei
        \item Bei der Berechnung helfen gängige mathematische Gesetze\pause{} (Logarithmus,~\ldots)
        \item Die wichtigste Rechenregel:\pause{} \(\O(f(n) + g(n)) = \O(\max\{f(n), g(n)\})\)
        \item Neben der \(\O\) Notation, existieren noch weitere Notationen,\pause{} wie \(\Omega(n)\), welches analog die untere Grenze darstellt.
        \item In der Regel reichen die folgenden wichtigsten Komplexitätsklassen:\vskip1em\pause{}
\begin{center}%
\resizebox{\linewidth}{!}{%
\setlength{\aboverulesep}{0pt}%
\setlength{\belowrulesep}{0pt}%
\setlength{\extrarowheight}{.45ex}%
\begin{tabular}{c*{8}{c}}
    \toprule
    & \onslide<9->{\cellcolor{@secondary!100!@alternative!21} \(\O(1)\)} & \onslide<10->{\cellcolor{@secondary!85!@alternative!21}\(\O(\log n)\)} & \onslide<11->{\cellcolor{@secondary!69!@alternative!21} \(\O(n)\)} & \onslide<12->{\cellcolor{@secondary!53!@alternative!21} \(\O(n\log n)\)} & \onslide<13->{\cellcolor{@secondary!36!@alternative!21} \(\O(n^2)\)} & \onslide<14->{\cellcolor{@secondary!19!@alternative!21} \(\O(n^3)\)} & \onslide<15->{\cellcolor{@secondary!14!@alternative!21} \(\O(2^n)\)} & \onslide<16->{\cellcolor{@secondary!0!@alternative!21} \(\O(n!)\)} \\[0.45ex]\midrule
    Bsp: & \onslide<9->{\(42\)} & \onslide<10->{\(4 \log (3n)\)} & \onslide<11->{\(4n-3\)} & \onslide<12->{\(4n\log(2n)\)} & \onslide<13->{\(n^2 + 2n - 1\)} & \onslide<14->{\(n^3 - 42n^2\)} & \onslide<15->{\(14\cdot2^n\)}& \onslide<16->{\(n! \cdot 10^{-42}\)}\\
    Bez: &\onslide<9->{\footnotesize konst.} & \onslide<10->{\footnotesize logarithm.} & \onslide<11->{\footnotesize linear} &\onslide<12->{\footnotesize linear log.} & \onslide<13->{\footnotesize quadratisch} & \onslide<14->{\footnotesize kubisch} & \onslide<15->{\footnotesize exponentiell} & \onslide<16->{\footnotesize faktoriell} \\
\bottomrule
\end{tabular}}
        \end{center}
    \end{itemize}
\end{frame}

\subsection{Modellierung durch UML}
\begin{frame}{UML -- Grundlagen}
    \begin{itemize}[<+(1)->]
        \widei
        \item Die Unified Modeling Language (UML) ist eine Kollektion an UML-Diagrammarten,\pause{} die es erlaubt ein Problem\thinspace /\thinspace Programm\thinspace /\thinspace Projekt aus verschiedenen Blickwinkeln zu betrachten.
        \item Im Kontext der Vorlesung gilt es drei Typen kurz zu skizzieren: \begin{description}[Sequenzdiagramme]
            \item[Klassendiagramme] modellieren die Beziehungen und Eigenschaften der beteiligten Klassen.
            \item[Objektdiagramme] modellieren die Beziehungen und Ausprägungen (spezifischer) Objekte.
            \item[Sequenzdiagramme] modellieren den Nachrichtenaustausch in einem Programm.\pause{} Sie sind ereignisbasiert.
        \end{description}
        \item UML wird hier (wie in der Vorlesung auch) nur oberflächlich betrachtet.
    \end{itemize}
\end{frame}

\begin{frame}[c]{UML -- Ein Überblick}
\begin{center}
    \resizebox{0.95\linewidth}{!}{
\begin{tikzpicture}[every node/.append style={text width=4.5cm, align=center, execute at begin node={\strut},block,font=\small\sffamily}]
    \onslide<2->{\node[iblock,text width=] (a) at (0,0) {UML};}
    \onslide<3->{\node[] (b1) at(-5,-1) {Strukturdiagramme};}
    \onslide<4->{\node[] (b2) at (0,-1) {Verhaltensdiagramme};}
    \onslide<5->{\node (b3) at (5,-1) {Weitere Diagramme};}
    \begin{scope}[every node/.append style={text width=3.25cm,font=\scriptsize\sffamily}]
        \onslide<6->{\node[iblock,font=\scriptsize\sffamily] (c2) at (-4,-2) {\hyperlink{uml:class}{Klassendiagramm}};}
        \onslide<7->{\node[iblock,font=\scriptsize\sffamily] (c3) at (1,-2) {\hyperlink{uml:sequence}{Sequenzdiagramm}};}
        \onslide<8->{\node (c4) at (6,-2) {Kommunikations\-struktur-Diagr.};}

        \onslide<6->{\node[iblock,font=\scriptsize\sffamily] (left1) at (-4,-3) {\hyperlink{uml:object}{Objektdiagramm}};
        \node[] (left2) at (-4,-4) {Paketdiagramm};}

        \onslide<7->{\node[] (middle1) at (1,-3) {Anwendungsfalldiagr.};
        \node[] (middle2) at (1,-4) {Kommunikationsdiagr.};
        \node[] (middle3) at (1,-5) {Aktivitätsdiagmm};
        \node[] (middle4) at (1,-6) {Zustandsdiagr.};
        \node (middle5) at (1,-7) {\ldots};}

        \onslide<8->{\node (right1) at (6,-3) {Komponentendiagr.};
        \node (right2) at (6,-4) {Verteilungsdiagr.};}
    \end{scope}

    \onslide<5->{ \draw (a) -| (b1) (a) -- (b2) (a) -| (b3);}
    \onslide<9->{\draw (b2.195) |- (c3) (b3.195) |- (c4);
    \draw (b1.195) |- (c2);
    \foreach \i in {1,2} {\draw (b1.195) |- (left\i);\draw (b3.195) |- (right\i);}
    \foreach \i in {1,...,5} {\draw (b2.195) |- (middle\i);}}
\end{tikzpicture}
    }
\end{center}
\end{frame}

\begin{frame}[c]{UML -- Klassendiagramme}
    \hypertarget<1>{uml:class}{}\begin{center}
        \onslide<2->{%
\begin{tikzpicture}[scale=.65, every node/.style={transform shape}]
    \tikzumlset{fill class=white, fill note=white!20}
    \umlclass[x=0,y=0,name=stud]{Student}{
        {\umlstatic{+ studierendenzahl : int}}\\
        {- name : String}\\
        {- matrikelnummer : int}\\
        {+ besuchtVorlesungen : String[5]}}{
            {+ Student (name : String, nummer : int)}\\
            {+ getName() : String}\\
            {+ getNummer() : int}\\
            {+ addVorlesung(String name) : void}\\
            {+ getVorlesungen() : String[5]}\\
            {+ removeVorlesung(String name) : void }}
    \draw [decorate,decoration={brace,amplitude=10pt,raise=4pt,mirror},yshift=0pt] (-4,2) -- ++(0,-1.92) node [black,midway, above,rotate=90,yshift=0.75cm] {Attribute};
    \draw [decorate,decoration={brace,amplitude=10pt,raise=4pt,mirror},yshift=0pt] (-4,2-1.92) -- ++(0,-2.78) node [black,midway, above,rotate=90,yshift=.75cm] {Methoden};
    \node at(0,3.25) [above] (kln) {Klassenname};
    \draw[-Kite] (kln) -- ++(0,-0.799);
    {\node [right] at (4.5,1.82-1.54/2+0.9) {\parbox{9cm}{
        {(\textbf{--}) steht für ein \bjava{private} Attribut}\\
        {(\textbf{+}) für ein \bjava{public} Attribut}\\
        {(\textbf{\#}) für ein \bjava{protected} Attribut}\\
        {\bjava{static}-Komponenten werden unterstrichen.}}};
    }
    \onslide<3->{\draw [Kite-] (4.15,1.85-1.3) -- ++(3,0) node[right] {\small bis zu 5 Vorlesungen};}
    \onslide<4->{\umlclass[x=11,y=-2,name=gang]{Studiengang}{- name : String\\- prof : String}{ + Studiengang(name : String, prof : String)\\ + getName() : String\\ + getProfessor() : String};}
    \only<5->{\umlassoc[mult1=0..3,pos1=0.965, mult2=*, pos2=.2]{Student}{Studiengang};}
    \onslide<6->{\draw [decorate,decoration={brace,amplitude=7pt,raise=4pt,mirror},yshift=0pt] (4.125,-3.75) -- ++(2.15,0) node [black,midway, below,yshift=-.75cm] {\begin{minipage}{8cm}
        \centering\scriptsize Ein Student kann bis zu 3 Studiengänge gleichzeitig besuchen, ein Studiengang kann von unendlich vielen Studenten besucht werden.
    \end{minipage}};}
\end{tikzpicture}}
    \end{center}
\end{frame}

\begin{frame}{UML -- Klassendiagramme, II}
    \begin{itemize}[<+(1)->]
        \widei
        \item Es gibt noch weitere Assoziationen.
        \item So gibt es: \begin{itemize}
            \item gerichtete Assoziationen (A\;\tikz[baseline=-.75ex]{\draw[-{Computer Modern Rightarrow[width=6pt,length=5pt]}] (0,0) -- (1,0);}\;B).
            \item Abhängigkeiten (A\;\tikz[baseline=-.75ex]{\draw[densely dashed] (0,0) -- (1,0);}\;B).
            \item Vererbungen (A\;\tikz[baseline=-.75ex]{\draw[-{Latex[round,open,length=8pt,width=6pt]}] (0,0) -- (1,0);}\;B).
            \item Aggregationen (A\;\tikz[baseline=-.75ex]{\draw[{Turned Square[open,length=8pt,width=6pt]}-] (0,0) -- (1,0)}\;B).\pause{} \textcolor{gray}{Markiert meist \say{Besitz}: \say{A besitzt ein B}.}
            \item Kompositionen (A\;\tikz[baseline=-.75ex]{\draw[{Turned Square[length=8pt,width=6pt]}-] (0,0) -- (1,0)}\;B).\pause{} \textcolor{gray}{Markiert meist \say{ist ein Teil von, mit gleicher Lebenszeit}: \say{A besteht aus B}}.
        \end{itemize}
        \item Pakete werden durch einen extra Kasten gekennzeichnet.
        \item \textit{Wichtig:} Attribute deren Typ eine andere Klasse ist werden durch eine Assoziation gekennzeichnet,\pause{} dies gilt auch, wenn eine Klasse sich selbst referenziert\pause{} (LinkedList,~\ldots).
    \end{itemize}
\end{frame}

\begin{frame}{UML -- Objektdiagramme}
    \hypertarget<1>{uml:object}{}\begin{itemize}[<+(1)->]
        \widei
        \item Objektdiagramme ähneln Klassendiagrammen.
        \item Allerdings handelt es sich immer um explizite Ausprägungen einer Klasse\pause{} (\(\Rightarrow\) keine Methoden).
    \end{itemize}
    \vfill\pause{}
    \begin{center}
        \begin{tikzpicture}[scale=0.75]
\node[rectangle split, rectangle split parts = 2,align=center,draw,inner sep=7pt] at(0,0) {%
    \bfseries \underline{objektname : Klassenname}\nodepart{two}
        attributA = <(aktueller) Wert> \\
        attributB = <(aktueller) Wert> \\
        \(\vdots\)
};
        \end{tikzpicture}
    \end{center}\vfill\null
\end{frame}

\begin{frame}{UML -- Sequenzdiagramme}
    \hypertarget<1>{uml:sequence}{}\begin{itemize}[<+(1)->]\widei
        \item Bilden den Nachrichtenaustausch ab.\pause{} Das Senden und Empfangen wird auf Basis von Ereignissen ausgelöst,\pause{} diese rufen wiederrum Reaktionen hervor.
        \item Die Nachrichten können jeweils synchron (durchgezogene Linie) oder asynchron (gestrichelte Linie) ausgetauscht werden.
        \item Nachrichten selbst können Methodenaufrufe, Rückgabewerte oder externe Ereignisse sein (wie Zeitereignisse).
        \item Der Zeitliche Ablauf wird hierbei durch \say{Lebenslinien} gekennzeichnet.
    \end{itemize}
\end{frame}

\begin{frame}[c]{UML -- Sequenzdiagramme, ein Beispiel}
    \begin{center}
\resizebox{.6\linewidth}{!}{\begin{tikzpicture}
    \begin{umlseqdiag}
        \umlactor[class=Player]{playerA}
        \umlcontrol[class=Server,x=6]{mainServer}
        \begin{umlcall}[op={\bjava{createLobby("test")}}]{playerA}{mainServer}
        \begin{umlcallself}[op={erstelle Lobby\ldots},op-style={right}]{mainServer}
        \end{umlcallself}
        \begin{umlfragment}[type=alt, inner xsep=5]
            \begin{umlcall}[op={\bjava{true /*:ws:success */}},type=return]{mainServer}{playerA}
            \end{umlcall}
            \umlfpart[error]
            \begin{umlcall}[op={\bjava{false /*:ws:failure */}},type=return]{mainServer}{playerA}
            \end{umlcall}
        \end{umlfragment}
        \end{umlcall}
    \end{umlseqdiag}
\end{tikzpicture}}
    \end{center}
\end{frame}

\subsection{Zahlensysteme}
\begin{frame}{Darstellung ganzer Zahlen}
    \begin{itemize}[<+(1)->]
        \widei
        \item Um Zahlen darzustellen verwenden wir ein \emph{Stellenwertsystem}.
        \item Das bedeutet der Wert einer Ziffer ergibt sich nicht nur über die Basis\pause{} sondern auch über die Stelle.
    \end{itemize}
    \vfill
    \begin{definition}<5->[Stellenwertsystem]
        \pause{}Bei einer Basis \(b\) und einer Zahl \(z\) aus den Ziffern \(z = z_{n} z_{n-1}\ldots z_0\)\pause{} ergibt sich ihr Wert im uns bekannten Dezimalsystem (\(b = 10\)) durch:\pause{} \[z_{b} = \sum_{i = 0}^{n} z_i \cdot b^i\]
    \end{definition}
    \vfill\hbox{}
\end{frame}

\begin{frame}{Darstellung ganzer Zahlen, II}
\begin{itemize}[<+(1)->]
    \item Die wichtigsten Basen für uns sind:\pause{} \(b = 2\) (dual/\allowbreak binär),\pause{} \(b = 8\) (Oktal)\pause{} und \(b = 16\) (Hexadezimal)
    \item Im Hexadezimalsystem werden die Ziffern \(\geq 10\) durch die Buchstaben \(A\,\widehat{=}\,10\) bis \(F\,\widehat{=}\,15\) dargestellt.
    \item Wir können eine Zahl aus dem Dezimalsystem in jedes andere konvertieren,\pause{} indem wir sukzessiv die Ziffern durch \say{\({}\bmod b\)} generieren\pause{} und die Zahl dann (ohne Rest) durch \(b\) teilen.
    \item Beispiel: \(z = 10\) soll ins Dualsystem konvertiert werden:\pause{} \begin{alignat*}{3}
        \onslide<+->{10 & \div 2 = 5 \qquad&10 \bmod 2 &= 0 \;(\leftarrow LSB) \\}
        \onslide<+->{5 & \div 2 = 2 \qquad&5 \bmod 2 &= 1 \\}
        \onslide<+->{2 & \div 2 = 1 \qquad&2 \bmod 2 &= 0 \\}
        \onslide<+->{1 & \div 2 = 0 \qquad&1 \bmod 2 &= 1 \;(\leftarrow MSB)\\}
    \end{alignat*}
\end{itemize}
\end{frame}


\begin{frame}[fragile]{Darstellung ganzer Zahlen, III}
\begin{itemize}[<+(1)->]
    \item Dies ergibt: \(1010_{(2)}\).
    \item Hinweis: es existieren weitere (schnelle) Konvertierungsverfahren,\pause{} die es zum Beispiel erlauben, das Hexadezimalsystem direkt ins Dualsystem zu konvertieren.\pause{} Da jede Ziffer im Hexadezimalsystem vier Bits einnimmt, geht die Konvertierung schneller:\pause{} \[
        AFFE_{(16)} = \overbrace{1010}^{A}\;\overbrace{1111}^{F}\;\overbrace{1111}^{F}\;\overbrace{1110}^{E}\;{}_{(2)}\vspace*{-0.4cm}
    \]
\end{itemize}
\ifull\Task{Zahlen konvertieren}%
\begin{exercise}<+->[Zahlen konvertieren \Time{3}]
    \onslide<+->{Konvertieren Sie \(42_{(8)}\) zur Basis \(b = 16\), \(b = 3\) und ins Binärsystem.}
\end{exercise}\fi
\end{frame}

\ifull
\begin{frame}[c]{Übungsaufgabe -- Lösung}
\begin{solve}<2->[Zahlen konvertieren]
    \pause{}Konvertieren wir die Zahl zuerst ins Binärsystem (schnelle Konvertierung mit je drei Bits),\pause{} dann von dort aus ins Hexadezimalsystem,\pause{} nun ins Dezimalsystem (nicht gefordert) und dann in \(b = 3\):\pause{} \begin{description}[b = 16]
        \item[b = 2] \(42_{(8)} = \overbrace{100}^{4}\overbrace{010}^{2}\;{}_{(2)}\)\medskip
        \item[b = 16] \(100010_{(2)} = \overbrace{2}^{0010}\overbrace{2}^{0010}\;{}_{(16)}\) (wir füllen also links mit Nullen auf, wenn die Bits kein Vielfaches von vier sind)\medskip
        \item[b = 10] \(4 \cdot 8^1 + 2 \cdot 8^0 = 32 + 2 = 34_{(10)}\)
    \end{description}
\end{solve}
\end{frame}

\begin{frame}[c]{Übungsaufgabe -- Lösung}
\addtocounter{solve}{-1}
\begin{solve}<1->[Zahlen konvertieren\hfill{}(Fortsetzung)]
    \begin{description}[b = 16]
        \item[b = 3] Wir verwenden sukzessive Division um von \(34_{(10)}\) auf \(b = 3\) zu kommen:\pause{} \begin{alignat*}{3}
            \onslide<+->{34 & \div 3 = 11 \qquad&34 \mod 3 &= 1 \;(\leftarrow LSB) \\}
            \onslide<+->{11 & \div 3 = 3 \qquad&11 \mod 3 &= 2 \\}
            \onslide<+->{3 & \div 3 = 1 \qquad&3 \mod 3 &= 0 \\}
            \onslide<+->{1 & \div 3 = 0 \qquad&1 \mod 3 &= 1 \;(\leftarrow MSB)\\}
        \end{alignat*}
        Damit ergibt sich:\pause{} \(42_{(8)} = 100010_{(2)} = 22_{(16)} = 34_{(10)} = 1021_{(3)}\)
    \end{description}
\end{solve}
\end{frame}
\fi

% #region Übungsaufgaben
\fullsubsection{Übungsaufgaben}
\ifull
\begin{frame}[fragile,c]{Funktion zu Rekursion}
    \Task{Schleife zu Rekursion}
    \begin{exercise}<2->[Schleife zu Rekursion \Time{4}]
        Wandeln Sie folgenden Code in einen rekursiven Algorithmus um:\pause{}
        \begin{plainjava}
!*\onslide<3->*!public int ggT(int a, int b){
!*\onslide<4->*!    while(b != 0){
!*\onslide<5->*!        final int tmp = b;
!*\onslide<5->*!        b = a % b;
!*\onslide<5->*!        a = tmp;
!*\onslide<4->*!    }
!*\onslide<6->*!    return a;
!*\onslide<3->*!}
        \end{plainjava}
    \end{exercise}\onslide<1->
\end{frame}

\begin{frame}[fragile,c]{Funktion zu Rekursion -- Lösung}
    \begin{solve}<2->[Schleife zu Rekursion]
\begin{plainjava}
!*\onslide<3->*!public int ggT(int a, int b){
!*\onslide<4->*!    if(b == 0) return a;
!*\onslide<5->*!    else return ggT(b, a % b);
!*\onslide<3->*!}
\end{plainjava}
    \end{solve}
\end{frame}

\begin{frame}[fragile,c]{Rekursiver Konverter}
    \Task{Rekursion zur Konvertierung}
    \begin{exercise}<2->[Rekursion zur Konvertierung \Time{4}]
        \pause{}Schreiben Sie eine rekursive Java-Methode \bjava{|plain|int2bin|plain|(int)}, die einen positiven Integer in dessen Binärdarstellung (als \bjava{String}) verwandelt. Beispiel:\pause{}
\begin{plainjava}
|plain|int2bin|plain|(42) // :yields: "101010"
\end{plainjava}
    \pause{}\textit{Hinweis:} Um eine Zahl in einen String zu konvertieren, kann Konkatenation oder \bjava{Integer.toString(int)} verwendet werden.
    \end{exercise}\onslide<1->
\end{frame}

\begin{frame}[fragile,c]{Rekursion zur Konvertierung -- Lösung}
    \begin{solve}<2->[Rekursion zur Konvertierung]
\begin{plainjava}
!*\onslide<3->*!String |plain|int2bin|plain|(int num){
!*\onslide<4->*!    if(num < 2) return Integer.toString(num);
!*\onslide<5->*!    return |plain|int2bin|plain|(num / 2) + Integer.toString(num % 2);
!*\onslide<3->*!}
\end{plainjava}
    \end{solve}
\end{frame}

\begin{frame}[fragile,c]{Numerisches Palindrom}
    \Task{Numerisches Palindrom}
    \begin{exercise}<2->[Numerisches Palindrom \Time{5}]
        \pause{}Schreiben Sie eine Java-Methode \bjava{boolean check(int n)}, die eine positiven Zahl \bjava{n} erhält und entscheidet, ob diese ein Palindrom ist.\pause{}
        Eine positive Zahl ist ein Palindrom, wenn sie von vorne und hinten gelesen den gleichen Wert repräsentiert: \(42324\) ist eines, \(192\) aber nicht.
        Führende Nullen sind zu ignorieren (\(010\) ist kein Palindrom). Verwenden Sie nur Rekursion und keine Iteration.\medskip

        Sie dürfen Hilfsmethoden und die Funktionen der \bjava{Math}-Klasse verwenden, die Konvertierung der Zahl in einen String oder ein Array ist nicht gestattet.\medskip

        \textit{Vereinfachung:} Wenn Sie Probleme mit führenden Nullen in der Rekursion haben, können Sie auch nur Ziffern \(z_i \in \{1, \ldots, 9\}\) unterstützen.
    \end{exercise}\onslide<1->
\end{frame}

\begin{frame}[fragile,c]{Numerisches Palindrom -- Lösung}
    \begin{solve}<2->[Numerisches Palindrom]
        \onslide<3->{Wir können die Zahl rekursiv umdrehen und vergleichen (mit Null):}
\begin{plainjava}
!*\onslide<6->*!int reverseNum(int number, int current) {
!*\onslide<7->*!    if (number == 0) return current;
!*\onslide<8->*!    return reverseNum(!*\onslide<9->*!number / 10, !*\onslide<10->*!current * 10 + number % 10!*\onslide<8->*!);
!*\onslide<6->*!}
!*\onslide<4->*!
!*\onslide<4->*!boolean check(int n) {
!*\onslide<5->*!    return n == reverseNum(n, 0);
!*\onslide<4->*!}
\end{plainjava}
    \onslide<9->{Wir schneiden die letzte Ziffer ab (\bjava{number / 10}), fügen diese \say{von vorne nach hinten} an die neue Zahl an (\bjava{current * 10 + number \% 10}).}
    \end{solve}
\end{frame}

\begin{frame}[fragile,c]{Numerisches Palindrom -- Lösung}
    \addtocounter{solve}{-1}%
    \begin{solve}<1->[Numerisches Palindrom\hfill{}(Fortsetzung)]
        \lstfs{10}\onslide<2->{Man kann auch analog zur \bjava{String::substring} Version vorgehen und mit der Stellenanzahl \bjava{Math.log10(double)} immer die erste und letzte Zahl abschneiden. Die Variablen sind hier zur Übersicht:}
\begin{plainjava}
!*\onslide<3->*!static boolean check(int n) {
!*\onslide<4->*!    if(n <= 9) return true;
!*\onslide<5->*!    int length = (int) Math.log10(n) + 1; // Anzahl der Ziffern in n
!*\onslide<6->*!    int largest = (int) Math.pow(10, length - 1); // Höchste Wertigkeit
!*\onslide<7->*!    int first = n / largest; // Größte Ziffer
!*\onslide<7->*!    int last = n % 10; // Kleinste Ziffer
!*\onslide<8->*!    if(first != last)
!*\onslide<8->*!        return false;
!*\onslide<9->*!    return check((n % largest) / 10); // Schneide größte & kleinste ab
!*\onslide<3->*!}
\end{plainjava}
    \onslide<10->{Das Problem: Nullen in \(1\underline{00}55001\) gehen hier verloren.}
    \end{solve}
\end{frame}

\begin{frame}[fragile,c]{Übungsaufgabe}
    \Task{Rechenaufwand berechnen}
    \begin{exercise}<2->[Rechenaufwand berechnen \Time{6}]
        \pause{}Wie viele Rechenschritte benötigt das folgende Verfahren einmal im \emph{worst-} und im \emph{best-case}? Geben Sie jeweils auch die \O-Notation an: \pause{}
\begin{plainjava}
int getDistance(String a, String b){
    if(a == null || b == null) return -1;
    if(a.length() != b.length()) return -1;
    int dist = 0x0;
    for(int i = 0b0; i < a.length(); i++)
        if(a.charAt(i) != b.charAt(i))
            dist++;
    return dist;
}
\end{plainjava}
    \end{exercise}\onslide<1->
\end{frame}

\begin{frame}[c]{Übungsaufgabe -- Lösung}
    \begin{solve}<2->[Rechenaufwand berechnen]
    \pause{}Handeln wir zuerst den \emph{best-case} ab.\pause{} Hier ist der String \bjava{a} \bjava{null} und die erste \bjava{if}-Bedingung terminiert mit \bjava{-1}.\pause{} Wir haben also im \emph{best-case} mit einem Aufwand von einem Vergleich und damit: \(\O(1)\).\pause{} (Auch wenn dieser \emph{best-case} relativ sinnfrei ist.\pause{} Was wäre denn der \emph{best-case} bei einer \say{gültigen} Eingabe\pause{}, also zwei gleichlangen Strings, die nicht \bjava{null} sind?)\pause{} Kommen wir nun zum \emph{worst-case}\ldots
    \end{solve}
\end{frame}

\begin{frame}[fragile,c]{Übungsaufgabe -- Lösung}
    \addtocounter{solve}{-1}%
    \begin{solve}<1->[Rechenaufwand berechnen\hfill{}(Fortsetzung)]
\begin{plainjava}[language=xJava]
int getDistance(String a, String b){
    if(|vgl|a == null|vgl| || |vgl|b == null|vgl|) return -1;
    if(|vgl|a.length() != b.length()|vgl|) return -1;
    !**!|zws|int dist = 0x0;|zws|
    for(|zws|int i = 0b0;|zws| |vgl|i < a.length();|vgl| |inc|i++|inc|)
        if(|vgl|a.charAt(i) != b.charAt(i)|vgl|)
            !**!|inc|dist++|inc|;
    return dist;
}
\end{plainjava}
    \end{solve}
\end{frame}


\begin{frame}[c]{Übungsaufgabe -- Lösung}
    \addtocounter{solve}{-1}%
    \begin{solve}<1->[Rechenaufwand berechnen\hfill{}(Fortsetzung)]
        \pause{}Sei die Länge von \bjava{a} durch \(n\) notiert (\bjava{a.length()}\(\,\widehat{=}\, n\))
\begin{multicols}{2}
    \begin{enumerate}[<+(1)->]
        \item \bjava[language=xJava]{|zws|Zuweisungen|zws|}\,: \(2\)
        \item \bjava[language=xJava]{|vgl|Vergleiche|vgl|}\,: \(3 + n + n + 1\)
        \item \bjava[language=xJava]{|inc|Inkrementierungen|inc|}\,: \(n+n\)
        \item \bjava[language=xJava]{|dec|Dekrementierungen|dec|}\,: \(0\)
    \end{enumerate}
\end{multicols}
    \pause{}Insgesamt ergibt sich damit: \((2) + (4 + 2n) + (2n) + (0) = 6 + 4n\).\pause{} Dieser Rechenaufwand liegt in \(6 + 4n \in \O(n)\).\smallskip\par\pause{}Hinweis: Der Vergleich im Kopf einer \bjava{for}-Schleife wird einmal öfters ausgeführt,\pause{} als die \bjava{for}-Schleife selbst.
    \end{solve}
\end{frame}

\begin{frame}[fragile,c]{Übungsaufgabe}
    \Task{Komplexitätsklassen ordnen}
    \begin{exercise}<2->[Komplexitätsklassen ordnen \Time{4}]
        \pause{}Ordnen Sie die folgenden Komplexitätsklassen (ohne Beweis), von der am geringsten skalierenden zur am stärksten skalierenden.\pause{}
        \begin{multicols}{3}
            \begin{enumerate}[<+(1)->]\widei
                \item \(\O(n \cdot \log(n^2))\)
                \item \(\O(4^n)-\O(2^n)\)
                \item \(\O(\frac{n}{12}) + \O(n^3)\)
                \item \(\O(\frac{n!}{n})\)
                \item \(\O(14n+12)\)
                \item<1->[]
            \end{enumerate}
        \end{multicols}
    \end{exercise}\onslide<1->
\end{frame}

\begin{frame}[c]{Übungsaufgabe -- Lösung}
    \begin{solve}<2->[Komplexitätsklassen ordnen]
        \pause{}Es ergibt sich: \begin{enumerate}[<+(1)->]
            \item[5.] \(\O(14n+12) = \O(n)\)
            \item[1.] \(\O(n \cdot \log(n^2)) = \O(n \cdot 2 \cdot \log(n)) = \O(n \log n)\)
            \item[3.] \(\O(\frac{n}{12}) + \O(n^3) = \O(n^3)\)
            \item[2.] \(\O(4^n) - \O(2^n) = \O(4^n - 2^n) = \O(4^n)\)
            \item[4.] \(\O(\frac{n!}{n}) = \O((n-1)!) = \O(n!)\)
        \end{enumerate}
        \pause{}Eine derartige Kategorisierung kann auch (noch) genauer präzisiert werden.
    \end{solve}
\end{frame}

\begin{frame}[fragile,c]{Übungsaufgabe}
    \Task{Von Pseudocode zu Java}
    \begin{exercise}<2->[Von Pseudocode zu Java \Time{4}]
        \onslide<3->{Übersetzen Sie den Pseudocode in eine Java-Methode \say{\bjava{magic(int,int)}}. Behalten Sie die Vorgehensweise bei. \(\mathbb{N}\) darf durch einen Integer eingegrenzt, ungültige Eingaben müssen nicht abgefangen werden.}\medskip\par \onslide<4->{Beschreiben Sie auch kurz, was der Algorithmus berechnet:}\RestyleAlgo{tworuled}\par\onslide<5->{
\SetKwInput{KwIn}{Eingabe}%
\SetKwFor{While}{Solange}{tue}{}%
\SetKw{KwRet}{Gebe zurück:}%
\SetKwIF{If}{ElseIf}{Else}{Wenn}{tue:}{}{sonst:}{}
\SetAlgoVlined%
\begin{algorithm}[H]
\PreCode
\KwIn{\(a, b \in \mathbb{N}\) \((b \geq 0)\)}
\StartCode
    \If{b \textbf{ist} 0}{\KwRet1\;}

    \(wert :=\)~\textbf{Rekursion mit neu} \(a\) \textbf{ist alt} \(a\) \textbf{und neu} \(b\) \textbf{ist alt} \(b - 1\)\;
    \KwRet{\(a * wert\)}\;
\end{algorithm}}
    \end{exercise}\onslide<1->
\end{frame}

\begin{frame}[fragile,c]{Übungsaufgabe -- Lösung}
    \begin{solve}<2->[Von Pseudocode zu Java]
        \onslide<3->{Der Pseudocode lässt sich glücklicherweise fast direkt übernehmen:}
\begin{plainjava}
!*\onslide<4->*!int magic(int a, int b) {
!*\onslide<9->*!    // Abfangen ungültiger Eingaben:
!*\onslide<9->*!    // if(b < 0) throw IllegalArgumentException();
!*\onslide<5->*!    if(b == 0) return 1;
!*\onslide<4->*!
!*\onslide<6->*!    int wert = magic(a, b - 1);
!*\onslide<7->*!    return a * wert;
!*\onslide<4->*!}
\end{plainjava}
        \onslide<8->{Die Methode berechnet \(a^b\) der Eingabe~--- \(a\) wird \(b\) mal mit sich selbst multipliziert.}\onslide<1->
    \end{solve}
\end{frame}

\makeatletter
\newsavebox\pengu@rook@A
\sbox\pengu@rook@A{\begin{tikzpicture}%
    \pingu[eyes shiny,rook];%
\end{tikzpicture}}%
\newsavebox\pengu@rook@B
\sbox\pengu@rook@B{\begin{tikzpicture}\pingu[eyes shiny,rook=@alerted];\end{tikzpicture}}%
\def\@em@extension@chess@figures@rookA{\resizebox{.45cm}{!}{\usebox\pengu@rook@A}}%
\def\@em@extension@chess@figures@rookB{\resizebox{.45cm}{!}{\usebox\pengu@rook@B}}%
\begin{frame}[c]{Übungsaufgabe}
    \Task{Das \(n\)-Türme Problem}
\emSetFieldSize{4cm}{4cm}\emSetTileSize{.65cm}{.65cm}%
\emSetPlayerSize{.5cm}{.5cm}\emSetTileOffset{.65cm}{.65cm}%
\colorlet{chessfieldblack}{black!75!white}%
\colorlet{chessblack}{@secondary}%
\colorlet{chesswhite}{@alerted}%
\emSetField{0/\em@W\em@B\em@W\em@B\em@W,0/\em@B\em@W\em@B\em@W\em@B,0/\em@W\em@B\em@W\em@B\em@W,0/\em@B\em@W\em@B\em@W\em@B,0/\em@W\em@B\em@W\em@B\em@W}%
    \begin{exercise}<2->[Das \(n\)-Türme Problem \Time{7}]%
        \smallskip\par\parbox[b]{.735\linewidth}{\onslide<3->{Ein Turm kann einen anderen auf einer vertikalen und horizontalen Linie schlagen. Im rechten \(5\times 5\) Feld, zum Beispiel der blaue Turm den am blauen Punkt.}\smallskip

        \onslide<4->{Schreiben Sie eine Methode, welche ein quadratisches Array erhält und
        genau dann \bjava{true} zurückliefert, wenn mindestens ein Turm einen anderen schlagen kann: \bjava{boolean check (boolean[][] arr)}.} \onslide<5->{Ist \bjava{arr[i][j]} \bjava{true},}}\hfill%
\onslide<3->{\pgfsetlayers{background,main,middle,foreground}\begin{eagle-map*}%
\fill[rounded corners=3pt,lightgray,fill opacity=.25] ([xshift=-1ex,yshift=-1ex]emline16) rectangle ([xshift=1ex,yshift=1ex]emline20);
\fill[rounded corners=3pt,lightgray,fill opacity=.25] ([xshift=-1ex,yshift=-1ex]emline22) rectangle ([xshift=1ex,yshift=1ex]emline2);
\emChessSetBoard{%
    {,black:rookA,,,,,,black:rookA},
    {,,,,,,black:rookA,,black:rookB},
    {},{}%
}%
\pgfonlayer{foreground}
\foreach \i in {16,18,19,20,7,12,22} {
    \fill[@primary] (emline\i) circle [radius=2pt];
}
\fill[@alerted] ([yshift=-1mm]emline2) circle [radius=2pt];
\endpgfonlayer
\end{eagle-map*}}\par%
 \onslide<5->{so markiert dies einen Turm in der \(i\)-ten Zeile und \(j\)-ten Spalte.}\smallskip\par
\onslide<6->{Es ist Ihnen für diese Aufgabe {\sbfamily nicht} gestattet, Schleifen zu verwenden. Verwenden Sie \emph{ausschließlich} Fallunterscheidungen und Methodenaufrufe für den Kontrollfluss.} \onslide<7->{Nehmen Sie an, dass \T{arr} wirklich quadratisch und mindestens ein Feld groß ist. Hilfsmethoden sind gestattet.}
    \end{exercise}\onslide<1->
\end{frame}
\makeatother

\begin{frame}[fragile,c]{Übungsaufgabe -- Lösung}
    \begin{solve}<2->[Das \(n\)-Türme Problem]
\begin{plainjava}
!*\onslide<3->*!boolean check(boolean[][] arr) {
!*\onslide<3->*!    return check(arr, 0, 0); // Start left upper corner
!*\onslide<3->*!}
!*\onslide<3->*!
!*\onslide<4->*!boolean check(boolean[][] arr, int y, int x) {
!*\onslide<5->*!    if(y >= arr.length) return false; // done
!*\onslide<6->*!    // Next line
!*\onslide<6->*!    if(x >= arr.length) return check(arr, y + 1, 0);
!*\onslide<7->*!    if(arr[y][x] && canBeatOtherRook(arr, y, x)) return true;
!*\onslide<3->*!
!*\onslide<8->*!    return check(arr, y, x + 1);
!*\onslide<4->*!}
\end{plainjava}
    \end{solve}
\end{frame}

\begin{frame}[fragile,c]{Übungsaufgabe -- Lösung}
    \addtocounter{solve}{-1}%
    \begin{solve}<2->[Das \(n\)-Türme Problem\hfill(Fortsetzung)]
\begin{plainjava}
!*\onslide<3->*!boolean canBeatOtherRook(boolean[][] arr, int y, int x) {
!*\onslide<4->*!    return canBeatOtherRook(arr, y, x, 0);
!*\onslide<3->*!}
!*\onslide<3->*!
!*\onslide<5->*!boolean canBeatOtherRook(boolean[][] arr, int y, int x, int i) {
!*\onslide<6->*!    if(i >= arr.length) // checked all vertical and horizontal
!*\onslide<6->*!        return false;
!*\onslide<7->*!    if(x != i && y != i) // different position
!*\onslide<8->*!        if(arr[y][i] || arr[i][x]) // rook in line or column
!*\onslide<9->*!            return true;
!*\onslide<10->*!    return canBeatOtherRook(arr, y, x, i + 1);
!*\onslide<5->*!}
\end{plainjava}
    \end{solve}
\end{frame}
\fi
% #endregion