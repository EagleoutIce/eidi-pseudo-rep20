\section{Weiterführende Konzepte}

\subsection{Suchverfahren}

\begin{frame}{Suchverfahren formal}
    \begin{itemize}[<+(1)->]
        \item Ein Suchverfahren hat die Aufgabe in einer Folge an Elementen,\pause{} die zu finden, die einem Muster oder gewissen Eigenschaften entsprechen.
        \item Es gibt zwei Varianten: \begin{description}[Heuristisch]
            \item[Einfach] Durchlaufen Suchraum auf Basis einer Datenstruktur.
            \item[Heuristisch] Besitzen zusätzliches Wissen (Sortierung,\ldots) über die Daten die sie zur Beschleunigung der Suche verwenden.
        \end{description}
        \item Die \emph{Eigenschaften} nach denen gesucht werden soll, kann man auch als \emph{Merkmale} bezeichnen.
    \end{itemize}
\end{frame}

\begin{frame}{Klassische Suchverfahren}
    \begin{itemize}[<+(1)->]
        \widei
        \item Uns sind zwei Verfahren bekannt Elemente zu suchen: \begin{description}[Linear]
            \item[Linear] Die lineare/naive/sequentielle Suche beginnt am Anfang der Folge und prüft Element für Element ob es sich um das gesuchte handelt.\pause{} Die Komplexität beträgt \(\O(n)\).
        \end{description}
        \vfill
        \begin{center}
            \onslide<5->{\begin{tikzpicture}
                \foreach[remember=\i as \li(initially 0)] \i in {1,...,10}{
                    \ifnum\i=6
                        \node[inner sep=4.35pt,iball,double,outer sep=2pt] (\i) at(\i,0) {};
                    \else
                        \node[inner sep=4pt,ball,outer sep=2pt] (\i) at(\i,0) {};
                    \fi
                    \ifnum\i>1
                        \draw (\li.east) -- (\i.west);
                    \fi
                }
                \onslide<+->{
                    \foreach[remember=\i as \li (initially 1)] \i in {2,...,6}{
                        \draw[-Kite] (\li.80) to[bend left] (\i.100);
                    }
                }
            \end{tikzpicture}}
        \end{center}
        \vfill\hbox{}
    \end{itemize}
\end{frame}

\begin{frame}[fragile,c]{Klassische Suchverfahren}
    \begin{plainjava}
    !*\onslide<2->*!int linearSearch(int[] sequence, int key){
    !*\onslide<3->*!    for(int i = 0; i < sequence.length; i++)
    !*\onslide<4->*!        if(sequence[i] == key) return i;
    !*\onslide<5->*!    return -1; // Magic Number für "not found"
    !*\onslide<2->*!}
    \end{plainjava}
    \end{frame}

\begin{frame}{Klassische Suchverfahren}
    \begin{itemize}[<+(1)->]
        \item[]\begin{description}[Linear]
            \item[Binär] \hypertarget{mrk:binsearch}{Die} binäre Suche benötigt eine (aufsteigend) \emph{sortierte} und arbeitet sich baumartig durch die Folge.\pause{} Sie vergleicht das mittlere Element des aktuellen Suchbereiches mit dem Suchbegriff.\pause{} Wenn er größer ist, wird im Suchfenster rechts von der Mitte, wenn er kleiner links von der Mitte weitergesucht.\pause{} Die Komplexität beträgt \(\O(\log n)\) (ohne Sortieren).
        \end{description}
        \vspace*{1em}
        \begin{center}
            \onslide<7->{\scalebox{0.65}{\begin{forest}
                for tree={inner sep=4pt,ball,outer sep=1pt}
                [,name=m1[[[][[][,phantom]]][[][[][]]]][,name=m2[][,name=m3[,name=m4,iball,inner sep=4.35pt,double[,phantom][]][[]]]]]
                \onslide<+->{
                    \draw[-Kite] (m1) to[bend left,looseness=1.25] (m2);
                    \draw[-Kite] (m2) to[bend left,looseness=1.15] (m3);
                    \draw[-Kite] (m3) to[bend right,looseness=1] (m4);
                }
            \end{forest}}}
        \end{center}
        \vfill\hbox{}
    \end{itemize}
\end{frame}

\begin{frame}[fragile,c]{Klassische Suchverfahren}
    \begin{plainjava}
!*\onslide<2->*!int binarySearch(int[] sequence, int key){
!*\onslide<3->*!    int min = 0, max = sequence.length - 1;
!*\onslide<4->*!    while(min <= max) {
!*\onslide<5->*!        int middle = (min + max)/2; // Mitte
!*\onslide<6->*!        if(sequence[middle] == key) // gefunden
!*\onslide<6->*!            return middle;
!*\onslide<7->*!        else if(key < sequence[middle]) // links
!*\onslide<7->*!            max = middle - 1;
!*\onslide<8->*!        else // rechts
!*\onslide<8->*!            min = middle + 1;
!*\onslide<4->*!    }
!*\onslide<9->*!    return -1; // Magic Number für "not found"
!*\onslide<2->*!}
\end{plainjava}
\end{frame}

\subsection{Sortierverfahren}

\begin{frame}{Sortierverfahren formal}
    \begin{itemize}[<+(1)->]
        \item Beschreibt einen Prozess bei dem Elemente aufgrund eines Ordnungskriteriums ihrer Eigenschaften in eine Reihenfolge gebracht werden.
        \item Daten werden sortiert um die Suche zu vereinfachen.
        \item Wir unterscheiden zwei Sortierverfahren: \begin{description}[Extern]
            \item[Intern] Hier liegt der gesamte Datenbestand im Arbeitspeicher,\pause{} alle Elemente sind zugreifbar (Array, \ldots)
            \item[Extern] Der Datenbestand ist (überwiegend) ausgelagert,\pause{} für das Sortieren werden in der Regel nur die obersten Element eines Stapels betrachtet (Dateien, \ldots).\pause{} In der Regel lassen sich durch Modifikationen und unterstützenden Methoden die selben Verfahren wie bei der Internen Variante verwenden.
        \end{description}
    \end{itemize}
\end{frame}

\begin{frame}{Sortierverfahren formal}
    \begin{itemize}[<+(1)->]
        \widei
        \item Bei der Vorgehensweise unterscheiden wir (im Kontext der Vorlesung) drei Ansätze: \begin{description}[Divide-and-Conquer]
            \item[Sukzessiv] Schritt für Schritt wird die Anzahl an unsortierten Elementen verringert.\pause{} In der Regel iterativ implementiert.
            \item[Divide-and-Conquer] Daten werden aufgeteilt,\pause{} in Teilen sortiert und dann aus sortierten Teilmengen sortiert zusammengefügt.\pause{} In der Regel rekursiv implementiert.
            \item[Halde] Sortieren mit dafür geeigneten Datenstrukturen wie dem \emph{Heap}.\pause{} In der Regel rekursiv implementiert.
        \end{description}
    \end{itemize}
\end{frame}

\def\List#1#2{
    \foreach[count=\i,remember=\i as \li(initially 0)] \n in {#1}{
        \ifnum\i>#2
            \node[inner sep=4pt,ball,outer sep=2pt] (\i) at(1.25*\i,0) {\n};
        \else
            \node[inner sep=4pt,iball,outer sep=2pt] (\i) at(1.25*\i,0) {\n};
        \fi
        \ifnum\i>1
            \draw (\li.east) -- (\i.west);
        \fi
    }
}

\begin{frame}{Sortierverfahren: Selectionsort}\alglabel{Selectionsort}
    \begin{definition}<2->[Selectionsort \hfill{}(iterativ)]
        \onslide<3->{Prinzip: Wahl des kleinsten Elements im unsortierten Teil, \onslide<4->{Anfügen des Elements als größtes Element des sortierten Teils.}}
    \end{definition}
    \begin{itemize}[<+(1)->]
        \item Kann auch von oben nach unten implementiert werden.
        \item Wir arbeiten \emph{in-place}, das heißt wir tauschen die Elemente im Array.
        \item Ist das kleinste Element an erster Stelle des unsortierten Teils wird dennoch getauscht (nach Vorlesung).
    \end{itemize}
    \vfill
    \begin{center}\pause{}
        \begin{minipage}{0.9\linewidth}
            \begin{multicols}{2}
                \begin{enumerate}[<+(1)->]
                    \item \begin{tikzpicture}[scale=0.65,every node/.style={transform shape},baseline=-0.6ex]
                        \List{4,2,1,3}{0}
                        \draw[Kite-Kite] (1.north) to[bend left] (3.north);
                    \end{tikzpicture}
                    \item \begin{tikzpicture}[scale=0.65,every node/.style={transform shape},baseline=-0.6ex]
                        \List{1,2,4,3}{1}
                        \draw[Kite-Kite] (2) to[loop above] (2);
                    \end{tikzpicture}
                    \item \begin{tikzpicture}[scale=0.65,every node/.style={transform shape},baseline=-0.6ex]
                        \List{1,2,4,3}{2}
                        \draw[Kite-Kite] (3.north) to[bend left] (4.north);
                    \end{tikzpicture}
                    \item \begin{tikzpicture}[scale=0.65,every node/.style={transform shape},baseline=-0.6ex]
                        \List{1,2,3,4}{4}
                    \end{tikzpicture}
                \end{enumerate}
            \end{multicols}
        \end{minipage}
    \end{center}
\end{frame}

\begin{frame}[fragile,c]{Sortierverfahren: Selectionsort}
    \begin{plainjava}
!*\onslide<2->*!public static void selectionSort(int[] arr) {
!*\onslide<3->*!    for(int i = 0; i < arr.length - 1; i++) {
!*\onslide<4->*!        int min = i; // Suche Minimum
!*\onslide<5->*!        for (int j = i+1; j < arr.length; j++)
!*\onslide<6->*!            if(arr[j] < arr[min])
!*\onslide<6->*!                min = j;
!*\onslide<7->*!        // Tausche 'i' und 'min'
!*\onslide<8->*!        int tmp = arr[i];
!*\onslide<9->*!        arr[i] = arr[min];
!*\onslide<10->*!        arr[min] = tmp;
!*\onslide<3->*!    }
!*\onslide<2->*!}
    \end{plainjava}
\end{frame}

\begin{frame}{Sortierverfahren: Insertionsort}\alglabel{Insertionsort}
    \begin{definition}<2->[Insertionsort \hfill{}(iterativ)]
        \onslide<3->{Prinzip: Wahl des ersten Elements im unsortierten Teil, \onslide<4->{Sortieren des Elements in den bereits sortierten Teil.}}
    \end{definition}
    \begin{itemize}[<+(1)->]
        \item Wir arbeiten \emph{in-place}, das heißt wir tauschen die Elemente im Array.
        \item Das Tauschen erfolgt durch durchgehende Vertauschungen.
        \item Ist das Element schon an der richtigen Position wird dennoch getauscht.
    \end{itemize}
    \vfill
    \begin{center}\pause
        \begin{minipage}{0.9\linewidth}
            \begin{multicols}{2}
                \begin{enumerate}[<+(1)->]
                    \item \begin{tikzpicture}[scale=0.65,every node/.style={transform shape},baseline=-0.6ex]
                        \List{4,2,1,3}{1}
                        \draw[-Kite] (1) to (2);
                        \draw[Kite-Kite] ([xshift=-15pt]1.north) to[bend left] (2.north);
                    \end{tikzpicture}
                    \item \begin{tikzpicture}[scale=0.65,every node/.style={transform shape},baseline=-0.6ex]
                        \List{2,4,1,3}{2}
                        \draw[-Kite] (1) to (2);
                        \draw[-Kite] (2) to (3);
                        \draw[Kite-Kite] ([xshift=-15pt]1.north) to[bend left] (3.north);
                    \end{tikzpicture}
                    \item \begin{tikzpicture}[scale=0.65,every node/.style={transform shape},baseline=-0.6ex]
                        \List{1,2,4,3}{3}
                        \draw[-Kite] (3) to (4);
                        \draw[Kite-Kite] ([xshift=-15pt]3.north) to[bend left] (4.north);
                    \end{tikzpicture}
                    \item \begin{tikzpicture}[scale=0.65,every node/.style={transform shape},baseline=-0.6ex]
                        \List{1,2,3,4}{4}
                    \end{tikzpicture}
                \end{enumerate}
            \end{multicols}
        \end{minipage}
    \end{center}
\end{frame}

\begin{frame}[fragile,c]{Sortierverfahren: Insertionsort}
    \begin{plainjava}
!*\onslide<2->*!public static void insertionSort(int[] arr) {
!*\onslide<3->*!    for(int i = 0; i < arr.length; i++) {
!*\onslide<4->*!        int pos = i - 1, elem = arr[i];
!*\onslide<5->*!        while(pos >= 0 && arr[pos] > elem) { // Verschiebe
!*\onslide<6->*!            arr[pos+1] = arr[pos];
!*\onslide<6->*!            pos--;
!*\onslide<5->*!        }
!*\onslide<7->*!        arr[pos+1] = elem;
!*\onslide<3->*!    }
!*\onslide<2->*!}
    \end{plainjava}
\end{frame}

% \fi


\def\List#1#2{
    \foreach[count=\i,remember=\i as \li(initially 0)] \n in {#1}{
        \ifnum\i<#2
            \node[inner sep=4pt,ball,outer sep=2pt] (\i) at(1.25*\i,0) {\n};
        \else
            \node[inner sep=4pt,iball,outer sep=2pt] (\i) at(1.25*\i,0) {\n};
        \fi
        \ifnum\i>1
            \draw (\li.east) -- (\i.west);
        \fi
    }
}

\begin{frame}{Sortierverfahren: Bubblesort}\alglabel{Bubblesort}
    \begin{definition}<2->[Bubblesort \hfill{}(iterativ)]
        \onslide<3->{Prinzip: Vertausche benachbarte Elemente wenn sie nicht in Sortierungsreihenfolge sind.}
    \end{definition}
    \begin{itemize}[<+(1)->]
        \item Wir arbeiten \emph{in-place}, das heißt wir tauschen die Elemente im Array.
        \item Mit jedem Durchgang über das Array wird (mindestens) ein Element an die richtige Position getauscht.
        \item Verbesserung: Shaker-Sort, schiebt abwechselnd das maximale und minimale Element an die (Ziel-)Position.
    \end{itemize}
    \vfill
    \begin{center}\pause
        \begin{minipage}{0.9\linewidth}
            \begin{multicols}{2}
                \begin{enumerate}[<+(1)->]
                    \item \begin{tikzpicture}[scale=0.65,every node/.style={transform shape},baseline=-0.6ex]
                        \List{4,2,1,3}{5}
                        \draw[Kite-Kite] (1.north) to[bend left] (2.north);
                        \draw[Kite-Kite] (2.north) to[bend left] (3.north);
                        \draw[Kite-Kite] (3.north) to[bend left] (4.north);
                    \end{tikzpicture}
                    \item \begin{tikzpicture}[scale=0.65,every node/.style={transform shape},baseline=-0.6ex]
                        \List{2,1,3,4}{4}
                        \draw[Kite-Kite] (1.north) to[bend left] (2.north);
                    \end{tikzpicture}
                    \item \begin{tikzpicture}[scale=0.65,every node/.style={transform shape},baseline=-0.6ex]
                        \List{1,2,3,4}{0}
                    \end{tikzpicture}
                    \item<1->[]
                \end{enumerate}
            \end{multicols}
        \end{minipage}
    \end{center}
\end{frame}

\begin{frame}[fragile,c]{Sortierverfahren: Bubblesort}
    \begin{plainjava}
public static void bubbleSort(int[] arr) {
    for(int i = arr.length - 1; i >= 1; i--) {
        for(int j = 0; j < i; j++) {
            if(arr[j] > arr[j+1]) { // Vertausche
                int tmp = arr[j];
                arr[j] = arr[j+1];
                arr[j+1] = tmp;
            }
        }
    }
}
    \end{plainjava}
\end{frame}

% TODO: maybe make with separate circles?

\newsavebox{\dotA}
\newsavebox{\dotB}
\newsavebox{\dotC}
\newsavebox{\dotD}
\newsavebox{\dotE}
\newsavebox{\dotF}
\newsavebox{\dotG}


\def\TikzList#1#2#3{%
\savebox{#3}{\begin{tikzpicture}
    \foreach[count=\i,remember=\i as \li (initially 0)] \n in {#1}{
        \ifnum\i=#2
            \node[inner sep=4pt,iball,outer sep=0pt,scale=0.8] (\i) at(0.7*\i,0) {\n};
        \else
            \node[inner sep=4pt,ball,outer sep=0pt,scale=0.8] (\i) at(0.7*\i,0) {\n};
        \fi
        \ifnum\i>1 \draw(\li) -- (\i); \fi
    }
    \end{tikzpicture}}
}

\TikzList{4,2,1,3}{5}{\dotA}
\TikzList{4,2}{5}{\dotB}
\TikzList{1,3}{5}{\dotC}
\TikzList{4}{5}{\dotD}
\TikzList{2}{5}{\dotE}
\TikzList{1}{5}{\dotF}
\TikzList{3}{5}{\dotG}


\begin{frame}{Sortierverfahren: Mergesort}\alglabel{Mergesort}
    \begin{definition}<2->[Mergesort \hfill{}(rekursiv)]
        \onslide<3->{Prinzip: Aufteilen der Elemente in einelementige, sortierte Teilfolgen. \onslide<4->{Zusammenfügen sortierter Teilefolgen.}}
    \end{definition}
    \begin{itemize}[<+(1)->]
        % \item Wir arbeiten \emph{in-place}, das heißt wir tauschen die Elemente im Array.
        \item Verfolgt \emph{easy split}, \emph{hard join}. Das Aufteilen ist leicht (trivial), das Zusammenfügen aufwändig, da hier die Sortierung erfolgen muss.
        \item Teilt sich auf in: \emph{Split} (aufteilen) und \emph{Merge} (verschmelzen).
    \end{itemize}
    \vfill
    \begin{center}\pause
        \scalebox{0.58}{\begin{forest}
            for tree={}
            [\usebox{\dotA},[{\usebox{\dotB}}[{\usebox{\dotD}}][{\usebox{\dotE}}]][{\usebox{\dotC}}[{\usebox{\dotF}}][{\usebox{\dotG}}]]]
            \node at(0,1) {Split};
        \end{forest}} \qquad~~\qquad%
        \TikzList{1,2,3,4}{5}{\dotA}%
        \TikzList{2,4}{5}{\dotB}%
        \TikzList{1,3}{5}{\dotC}%
        \TikzList{4}{5}{\dotD}%
        \TikzList{2}{5}{\dotE}%
        \TikzList{1}{5}{\dotF}%
        \TikzList{3}{5}{\dotG}%
        \scalebox{0.58}{\begin{forest}
            for tree={grow=north}
            [\usebox{\dotA},[{\usebox{\dotC}}[{\usebox{\dotG}}][{\usebox{\dotF}}]][{\usebox{\dotB}}[{\usebox{\dotE}}][{\usebox{\dotD}}]]]
            \node at(0,3.75) {Merge};
        \end{forest}}
    \end{center}
\end{frame}

\TikzList{4,2,1,3}{4}{\dotA}
\TikzList{1,2}{2}{\dotB}
\TikzList{1,3}{5}{\dotC}
\TikzList{4}{5}{\dotD}
\TikzList{2}{5}{\dotE}
\TikzList{1}{5}{\dotF}
\TikzList{3}{5}{\dotG}

\begin{frame}{Sortierverfahren: Quicksort}\alglabel{Quicksort}
    \begin{definition}<2->[Quicksort \hfill{}(rekursiv)]
        \onslide<3->{Prinzip: Wahl eines Pivotelements (nach Vorlesung: das Letzte).
        \onslide<4->{Aufteilen der Liste in Teil größer und kleiner des Pivotelements. \onslide<5->{Wenn Teillisten nach gleichem Prinzip sortiert: zusammenfügen aus \emph{Links + Pivot + Rechts}.}}}
    \end{definition}
    \begin{itemize}[<+(1)->]
        \item Wir arbeiten \emph{in-place}, das heißt wir tauschen die Elemente im Array.
        \item Verfolgt \emph{hard split}, \emph{easy join}. Das Aufteilen ist aufwändig (sortieren), das Zusammenfügen trivial.
    \end{itemize}
    \vfill
    \begin{center}\pause
        \scalebox{0.58}{\begin{forest}
            for tree={}
            [{\usebox{\dotA}},[{\usebox{\dotB}}[{\usebox{\dotF}}][,phantom]][{{\usebox{\dotD}}}[,phantom][,phantom]]]
            \node at(0,1) {Split};
        \end{forest}} \qquad~~\qquad%
        \TikzList{1,2,3,4}{3}{\dotA}
        \TikzList{1,2}{2}{\dotB}
        \TikzList{1,3}{5}{\dotC}
        \TikzList{4}{5}{\dotD}
        \TikzList{2}{5}{\dotE}
        \TikzList{1}{5}{\dotF}
        \TikzList{3}{5}{\dotG}
        \scalebox{0.58}{\begin{forest}
            for tree={grow=north}
            [{\usebox{\dotA}},[{{\usebox{\dotD}}}[,phantom][,phantom]][{\usebox{\dotB}}[,phantom][{\usebox{\dotF}}]]]
            %[{\(1,2,\mathbf{3},4\)},iblock[{\(4\)}[,phantom][,phantom]][{\(1,\mathbf{2}\)}[,phantom][\(1\)]]]
            \node at(0,3.75) {Merge};
        \end{forest}}
    \end{center}
\end{frame}

\begin{frame}{Sortierverfahren: Heapsort}\alglabel{Heapsort}
    \begin{definition}<2->[Heapsort \hfill{}(rekursiv)]
        \onslide<3->{Prinzip: Verwenden eines (Min-)Heaps zur Speicherung der Elemente. \onslide<4->{Sukzessives Entfernen des Wurzelelements (aktuelles Minimum), wiederherstellung der Heap-Eigenschaft.}}
    \end{definition}
    \begin{itemize}[<+(1)->]
        \item Wir arbeiten \emph{in-place}, das heißt wir tauschen die Elemente im Array.
        \item Die genaue Funktionsweise eines Heaps veranschaulichen wir gleich separat
    \end{itemize}
    \vfill
    \begin{center}\pause
        \parbox{5.65em}{\centering\scalebox{0.625}{\begin{forest}
            for tree={inner sep=4pt,ball,font=\sffamily,outer sep=1pt}
            [1,iball[3[4][,phantom]][2]]
        \end{forest}}} \(\rightarrow\) \parbox{5.65em}{\centering\scalebox{0.625}{\begin{forest}
            for tree={inner sep=4pt,ball,font=\sffamily,outer sep=1pt}
            [2,iball[3][4]]
        \end{forest}}} \(\rightarrow\) \parbox{5.65em}{\centering\scalebox{0.625}{\begin{forest}
            for tree={inner sep=4pt,ball,font=\sffamily,outer sep=1pt}
            [3,iball[4][,phantom]]
        \end{forest}}} \(\rightarrow\) \parbox{5.65em}{\centering\scalebox{0.625}{\begin{forest}
            for tree={inner sep=4pt,ball,font=\sffamily,outer sep=1pt}
            [4,iball]
        \end{forest}}}
    \end{center}
\end{frame}

\begin{frame}{Exkurs: Heap}
    \hypertarget{mrk:Heap}{}%
    \begin{definition}<2->[Heap]
        \onslide<3->{Binärbaum, bei dem gilt: \(\text{Vaterknoten} < \text{Kindknoten}\).}
    \end{definition}
    \begin{itemize}[<+(1)->]
        \item Neue Elemente werden von Unten an als neues Kind hinzugefügt. \pause{}Wird die Eigenschaft verletzt, werden solange kleinere Kinder nach oben getauscht bis die Eigenschaft wiederhergestellt ist.
        \item Beim Entfernen eines Elements wird das Blatt vor der aktuellen Einfügeposition als neue Wurzel getauscht.
        \item Beispiel: Einfügen der Elemente \(4,2,1,3\).
    \end{itemize}
    \vfill
    \begin{center}\pause
        \parbox{5.65em}{\centering\scalebox{0.625}{\begin{forest}
            for tree={inner sep=4pt,ball,font=\sffamily,outer sep=1pt}
            [4[,fill=white][,phantom]]
        \end{forest}}} \(\overset{1}{\rightarrow}\) \parbox{5.65em}{\centering\scalebox{0.625}{\begin{forest}
            for tree={inner sep=4pt,ball,font=\sffamily,outer sep=1pt}
            [4,name=4[1,name=1][,fill=white]]
            \draw[Kite-Kite] (4) to[bend right] (1);
        \end{forest}}} \(\overset{2}{\rightarrow}\) \parbox{5.65em}{\centering\scalebox{0.625}{\begin{forest}
            for tree={inner sep=4pt,ball,font=\sffamily,outer sep=1pt}
            [1[4[,fill=white][,phantom]][2]]
        \end{forest}}} \(\overset{3}{\rightarrow}\) \parbox{5.65em}{\centering\scalebox{0.625}{\begin{forest}
            for tree={inner sep=4pt,ball,font=\sffamily,outer sep=1pt}
            [1[4,name=4[3,name=3][,phantom]][2]]
            \draw[Kite-Kite] (4) to[bend right] (3);
        \end{forest}}} \(\rightarrow\) \parbox{5.65em}{\centering\scalebox{0.625}{\begin{forest}
            for tree={inner sep=4pt,ball,font=\sffamily,outer sep=1pt}
            [1[3[4][,phantom]][2]]
        \end{forest}}}
    \end{center}
\end{frame}

\begin{frame}[c]{Sortierverfahren: Zusammenfassung}
    \begin{table}[H]
        \renewcommand\arraystretch{1.35}
        \begin{tabular}{p{0.2\linewidth}p{0.195\linewidth}p{0.195\linewidth}p{0.13\linewidth}p{0.13\linewidth}}
        \toprule
            Sortierverfahren & worst-case & average-case & Speicher & Ansatz \\
        \midrule
            \linksort{Selectionsort} & \(\O(n^2)\) & \(\O(n^2)\) & \(\O(1)\) & iterativ \\
            \linksort{Insertionsort} & \(\O(n^2)\) & \(\O(n^2)\) & \(\O(1)\) & iterativ \\
            \linksort{Bubblesort} & \(\O(n^2)\) & \(\O(n^2)\) & \(\O(1)\) & iterativ \\
            \linksort{Mergesort} & \(\O(n \log n)\) & \(\O(n \log n)\) & \(\O(n)\) & rekursiv \\
            \linksort{Quicksort} & \(\O(n^2)\) & \(\O(n \log n)\) & \(\O(\log n)\) & rekursiv \\
            \linksort{Heapsort} & \(\O(n\log n)\) & \(\O(n \log n)\) & \(\O(1)\) & rekursiv \\
        \bottomrule
        \end{tabular}
    \end{table}
\end{frame}

\begin{frame}{Eigenschaft: Stabil}
    \begin{definition}<2->[Stabiles Sortierverfahren]
        \onslide<3->{Ein Sortierverfahren heißt stabil, wenn es die Reihenfolge (aus Sicht des Verfahrens) gleicher Element nicht ändert.}
    \end{definition}
    \begin{itemize}[<+(1)->]
        \item Beispiel: Eine bereits alphabetisch nach Nachnamen sortierte Liste an Personen wird nun nach Alter sortiert.\pause{} Bei einem stabilen Sortierverfahren bleibt die alphabetische Sortierung bei Personen des selben Alters erhalten.
        \item Stabile Sortierverfahren (nach Implementation der Vorlesung):\pause{} \linksort{Insertionsort}, \linksort{Quicksort} und \linksort{Mergesort}.
    \end{itemize}
\end{frame}


\subsection{Konzept: Backtracking}

\begin{frame}{Backtracking}
    \begin{itemize}[<+(1)->]
        \item Backtracking ist eine rekursive Lösungsstrategie.
        \item Das Problem wird von einer Teillösung aus erweitert, bis man auf die Gesamtlösung stößt,\pause{} oder in eine \say{Sackgasse} gelangt (trial and error), in diesem Fall probiert man eine andere Erweiterung.
        \item Beispiele: \begin{itemize}
            \item \emph{Wegfindung im Labyrinth:}\pause{} Gehe vom Startfeld aus solange einen Weg entlang, bis am Ziel angekommen.\pause{} Im Fall einer Sackgasse: springe zur letzten Position zurück, an der es noch einen anderen Weg gab.
            \item \emph{Lösung eines Sudoku.}\pause{} Füge in erstes freies Feld eine der dort möglichen Zahlen ein.\pause{} Verfahre so, bis alle Felder gefüllt sind oder für ein Feld keine Zahl mehr möglich ist.\pause{} In diesem Fall: springe zum letzten Punkt zurück, an dem noch andere Zahlen möglich sind. Probiere weiter.
            \item Rucksackproblem, 8-Damen Problem, \ldots
        \end{itemize}
    \end{itemize}
\end{frame}

\fullsubsection{Übungsaufgaben}
\ifull
\begin{frame}[c]{Übungsaufgabe}
    \Task{Sortieralgorithmen abwägen}
    \begin{exercise}<2->[Sortieralgorithmen abwägen \Time{2}]
        Quicksort hat eine schlechtere \emph{worst-case} Laufzeit als Mergesort, wieso trifft man in der Praxis dennoch häufig Quicksort?
    \end{exercise}
\end{frame}

\begin{frame}[c]{Lösung}
    \begin{solve}<2->[Sortieralgorithmen abwägen]
        \pause{}Mergesort benötigt einen zusätzlichen Speicherplatz von \(\O(n)\), Quicksort ist in-place implementiert und benötigt lediglich \(\O(\log n)\) Speicherplatz zum Verwalten der rekursiven Aufrufe.
    \end{solve}
\end{frame}

\begin{frame}[c]{Übungsaufgabe}
    \Task{Sortieralgorithmen zeichnen}
    \begin{exercise}<2->[Sortieralgorithmen zeichnen \Time{5}]
        \pause{}Veranschauliche den Sortierprozess der Zahlenfolge \(5,7,2,5,3,-19,6\) unter der Verwendung von \emph{Mergesort} grafisch.\pause{} Gib jeden Schritt aus.
    \end{exercise}
\end{frame}

\begin{frame}[c]{Lösung}
    \begin{solve}<2->[Sortieralgorithmen zeichnen]
        \pause{}Hier sind Split- und Mergephase wieder aufgespalten:\pause{}
\begin{center}
    \tikzset{block/.append style={rounded rectangle}}%
    \scalebox{0.75}{
    \begin{forest}
        for tree={inner sep=4pt,block,outer sep=1pt,execute at begin node={\strut}}
        [{\(5,7,2,5,3,-19,6\)}[{\(5,7,2,5\)}[{\(5,7\)}[5][7]][{\(2,5\)}[2][5]]][{\(3,-19,6\)}[3][{\(-19,6\)}[-19][6]]]]
    \end{forest}}\qquad~~\qquad\pause{}\scalebox{0.75}{
        \begin{forest}
            for tree={inner sep=4pt,block,outer sep=1pt,execute at begin node={\strut},grow=north}
            [{\(-19,2,3,5,5,6,7\)},iblock[{\(-19,3,6\)}[{\(-19,6\)}[6][-19]][3]][{\(2,5,5,7\)}[{\(2,5\)}[5][2]][{\(5,7\)}[7][5]]]]
        \end{forest}}
\end{center}
    \end{solve}
\end{frame}
\fi