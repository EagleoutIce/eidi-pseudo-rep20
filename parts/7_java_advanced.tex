\section{Java-Advanced}
\subsection{Vererbung \& Abstraktion}
\begin{frame}{Das Konzept der Vererbung}
    \hypertarget<1>{mrk:Vererbung}{}%
    \begin{itemize}[<+(1)->]
        \widei
        \item Eine erbende Klasse (\say{Subklasse}) übernimmt alle Eigenschaften und kann diese verändern.
        \item Die Subklasse kann hier auf zwei Weisen betrachtet werden: \begin{itemize}
            \item als Spezialisierung: die Eigenschaften und Operationen des Supertyps bleiben unverändert.
            \item als Erweiterung: die Methoden werden modifiziert.
        \end{itemize}
        \item Häufig wird das Konzept der \emph{Erweiterung} verfolgt.
        \item Zwischen vererbenden Klassen gilt eine \say{is-a} Relation.\pause{} So gilt beispielsweise \say{Student \emph{is-a} Mensch \emph{is-a} Lebewesen}.
    \end{itemize}
\end{frame}

\begin{frame}{Vererbung in Java}
    \begin{itemize}[<+(1)->]
        \widei
        \item Vererbung wird in Java durch den Schlüsselbegriff \bjava{extends} realisiert.
        \item Java verwendet Vererbung zur \emph{Erweiterung}.
        \item In Java kann eine Klasse nur von maximal einer anderen Erben.
        \item Auf Attribute der Elternklasse, die \bjava{private} sind, können wir in erbenden Kindklassen nicht zugreifen.
        \item Der Konstruktor sowie Methoden der Superklasse können (soweit sichtbar) durch \bjava{super} erreicht werden.
    \end{itemize}
\end{frame}

\ifull
\begin{frame}[c,fragile]{Übungsaufgabe}
    \Task{Fehlerfinden - Vererbungs special}
    \begin{exercise}<2->[Fehler finden - Vererbung \Time{5}]
        \pause{}Erklären Sie drei Kompilierzeit-Fehler. Die Klassen befinden sich jeweils in Dateien mit dem korrekten Namen und in dem \emph{selben} Verzeichnis:\pause{}
\lstfs{9}%
\begin{plainjava}
public class A {
    A(int foo) { System.out.println("Called with " + foo); }
    public static int have() { return 42; }
    protected boolean fun() { return true; }
    private String at() { return "Fridays"; }
    public String at(String place) { return place; }
}
public class B extends A {
    public static int have() { return 0; }
    private boolean fun() { return false; }
    public String at() { return "my place"; }
    protected String at(String place) { return super.at(); }
}
\end{plainjava}
    \end{exercise}
\end{frame}

\begin{frame}[c,fragile]{Lösung}
    \begin{solve}<2->[Fehler finden - Vererbung]
        Dies sind alle Fehler, welche vom Java-Compiler erkannt werden:
        \begin{enumerate}[<+(1)->]
            \item \bjava{B} benötigt einen Konstruktor, da es für die Superklasse nur \bjava{A(int)} gibt.\pause{} \textcolor{gray}{Zum Beispiel: \say{\bjava{B() \{ super(0); \}}}.}
            \item \bjava{B::fun()} ist illegal, es schränkt die Sichtbarkeit von \bjava{A::fun()} von \bjava{protected} auf \bjava{private} ein. Dies widerspricht dem Erweiterungskonzept.
            \item \bjava{B::at(String)} ist aus dem selben Grund illegal. Es schränkt die Sichtbarkeit von \bjava{public} auf \bjava{protected} ein!
            \item Ebenso darf \bjava{B::at(String)} die Methode \bjava{super.at();} gar nicht aufrufen, da \bjava{A::at()} die Sichtbarkeit \bjava{private} besitzt.
        \end{enumerate}
        Hinweis: \bjava{B::at()} ist kein Problem, da es die Sichtbarkeit von \bjava{private} auf \bjava{public} erhöht (\say{erweitert}).
    \end{solve}
\end{frame}
\fi

\begin{frame}[fragile]{Ein Beispiel in Java}
    \begin{itemize}[<+(1)->]
        \item \textcolor{gray}{Die Attribute sind lediglich exemplarisch und nicht wertend gemeint.}
    \end{itemize}
    \pause{}\begin{center}
        \scriptsize%
        \begin{columns}
            \begin{column}{.42\textwidth}
\begin{plainjava}
class Lebewesen {
    int alter;
    float grosse;

    Lebewesen(int a, float g) {
        alter = a;
        groesse = g;
    }
}
\end{plainjava}
            \end{column}\pause{}
            \begin{column}{.5\textwidth}
\begin{plainjava}
class Mensch extends Lebewesen {
    String name;

    Mensch(int a, float g, String n) {
        super(a, g);
        name = n;
    }
}
\end{plainjava}
            \end{column}
        \end{columns}
    \end{center}
\end{frame}

\begin{frame}[fragile]{Ein Beispiel in Java}
    \begin{itemize}[<+(1)->]
        \item<1-> \textcolor{gray}{Die Attribute sind lediglich exemplarisch und nicht wertend gemeint.}
    \end{itemize}
    \pause{}\begin{center}
        \scriptsize%
        \begin{columns}
            \begin{column}{.49\textwidth}
\begin{plainjava}
class Student extends Mensch {
    long matrikelnummer;

    Student(int a, float g, String n,
            long m) {
        super(a, g, n);
        matrikelnummer = m;
    }
}
\end{plainjava}
            \end{column}\pause{}
            \begin{column}{.49\textwidth}
\begin{plainjava}
class Dozent extends Mensch {
    String vorlesung;

    Dozent(int a, float g, String n,
           String v) {
        super(a, g, n);
        vorlesung = v;
    }
}
\end{plainjava}
            \end{column}
        \end{columns}
    \end{center}
\end{frame}

\begin{frame}[fragile]{instanceof und getClass()}
    \begin{itemize}[<+(1)->]
        \widei
        \item Das Prinzip erlaubt uns Polymorphie.\pause{} So ist folgender Code valide:\pause{}
\begin{plainjava}
Mensch herbert = new Student(19, 184.12f, "herbert", 12345);
\end{plainjava}
        \pause{}Die Klasse \bjava{Student} \emph{erbt} von \bjava{Mensch}.\pause{} \textcolor{gray}{Wir können nicht auf zusätzliche Attribute und Funktionen von \bjava{Student} zugreifen.}
        \item Ob \bjava{herbert} wirklich ein Student ist,\pause{} kann man mit \bjava{instanceof} prüfen:\pause{}
\begin{plainjava}
herbert instanceof Student // :yields: true
herbert instanceof Mensch // :yields: true
herbert instanceof Lebewesen // :yields: true
herbert instanceof Dozent // :yields: false
\end{plainjava}
        Dabei liefert \bjava{null instanceof X} immer \bjava{false}, für beliebige Klassen \bjava{X}.
    \end{itemize}
\end{frame}

\begin{frame}{Kommentare}
    \begin{itemize}[<+(1)->]
        \widei
        \item Das Prinzip der Vererbung ist integral für die Objektorientierung und unglaublich mächtig.
        \item Methoden der Superklasse lassen sich überladen (gleicher Name, andere Signatur),\pause{} als auch überschreiben (gleiche Signatur)
        \item Vererbung sollte nur dann verwendet werden,\pause{} wenn es sich wirklich um eine Erweiterung handelt und nicht wenn die Klasse \say{nur} die Eigenschaften (logisch) besitzt.\pause{} \medskip\par Beispiel: \solGet{keywordC}{Auto} sollte nicht von \solGet{keywordC}{Motor} erben, da es zwar einen besitzt, aber kein Motor ist.
    \end{itemize}
\end{frame}

\begin{frame}[fragile]{Überschreiben}
    \begin{itemize}[<+(1)->]
        \item Analog zum \hyperlink{mrk:Signatur}{\emph{Überladen}} (gleicher Name, unterschiedliche Signatur) gibt es das \say{Überschreiben}.
        \item Eine Methode in einer erbenden Klasse \textit{überschreibt} eine (sichtbare) Methode in der Superklasse mit der gleichen Signatur.
    \end{itemize}
\begin{plainjava}[morekeywords={[3]{A,B}}]
!*\onslide<4->*!public class A {
!*\onslide<4->*!    public int foo(int i) { return i + 1; }
!*\onslide<4->*!}
!*\onslide<5->*!public class B extends A {
!*\onslide<5->*!    public int foo(int i) { return i * i; }
!*\onslide<5->*!}
\end{plainjava}
\begin{itemize}
    \item<6-> Hier überschreibt \bjava[morekeywords={[3]{A,B}}]{B::foo(int)} die Methode \bjava[morekeywords={[3]{A,B}}]{A::foo(int)}.
    \item<7-> \bjava{new B().foo(3)} produziert damit \bjava{9}. In \solGet{keywordC}{B} steht \bjava[morekeywords={[3]{A,B}}]{A::foo(int)} über \bjava{super} zur Verfügung.
\end{itemize}
\end{frame}

\begin{frame}{Abstrakte Klassen}
    \begin{itemize}[<+(1)->]
        \widei
        \item Vererbung ermöglicht Polymorphie (erst richtig).
        \item So lässt sich eine Klasse \bjava{List} schaffen \textcolor{gray}{(in Java: \bjava{AbstractList}, bereits vorhanden)}, welche die Zugriffe definiert und entsprechend Klassen \bjava{ArrayList}, \bjava{LinkedList},~\ldots\pause{} die diese Operationen implementieren.
        \item Sind diese Klassen (wie \bjava{List}) wirklich nur Blaupausen/\allowbreak Vorlagen sind,\pause{} können wir sie als \bjava{abstract} bezeichnen.
        \item Von \bjava{abstract} Klassen darf kein Objekt erstellt werden.\par\textcolor{gray}{Von ihnen kann aber geerbt werden.}
        \item Eine abstrakte Klasse kann abstrakte Methoden (ohne Rumpf!) definieren.\pause{} Diese \emph{müssen} von einer erbenden, nicht-abstrakten Klasse implementiert werden.
    \end{itemize}
\end{frame}

\begin{frame}[fragile,c]{Abstrakte Klassen, ein Beispiel}
\begin{plainjava}
!*\onslide<2->*!abstract class AbstractList {
!*\onslide<2->*!
!*\onslide<3->*!    // Für 'super()'
!*\onslide<3->*!    public AbstractList() { }
!*\onslide<2->*!
!*\onslide<4->*!    abstract void add(Element e);
!*\onslide<4->*!
!*\onslide<5->*!    int size() {
!*\onslide<5->*!        // ...
!*\onslide<5->*!    }
!*\onslide<6->*!    // ...
!*\onslide<2->*!}
\end{plainjava}
\end{frame}

\begin{frame}[fragile,c]{Abstrakte Klassen, ein Beispiel}
\begin{plainjava}
!*\onslide<2->*!class ArrayList extends AbstractList {
!*\onslide<3->*!    Element[] elems;
!*\onslide<2->*!
!*\onslide<4->*!    public ArrayList() {
!*\onslide<5->*!        super();
!*\onslide<5->*!        elems = new Element[0];
!*\onslide<4->*!    }
!*\onslide<2->*!
!*\onslide<6->*!    public void add(Element e) {
!*\onslide<6->*!        // ...
!*\onslide<6->*!    }
!*\onslide<7->*!    // ...
!*\onslide<2->*!}
\end{plainjava}
\end{frame}


\begin{frame}[fragile]{Statische Bindung}
    \begin{itemize}[<+(1)->]
        \item Attribute werden Java \emph{statisch} und Methoden \emph{dynamisch} gebunden.
        \item Doch was bedeuten diese Bindungsarten: \begin{description}[dynamisch]
            \item[statisch] Nehmen wir an, wir haben zwei Klassen \solGet{keywordC}{A} und \solGet{keywordC}{B}, wobei \solGet{keywordC}{B} von \solGet{keywordC}{A} erbt.\pause{} Beide deklarieren die Variable \bjava{i}, in \solGet{keywordC}{A} wird sie auf \(32\), in \solGet{keywordC}{B} auf \(42\) gesetzt.\pause{} Es ergibt sich:
\begin{plainjava}[morekeywords={[3]{A,B}}]
!*\onslide<7->*!A a = new A();
!*\onslide<7->*!B b = new B();
!*\onslide<7->*!A c = new B();
!*\onslide<8->*!System.out.println(a.i); // :yields: 32
!*\onslide<9->*!System.out.println(b.i); // :yields: 42
!*\onslide<10->*!System.out.println(c.i); // :yields: 32
            \end{plainjava}
        \end{description}
    \end{itemize}
\end{frame}

\begin{frame}[fragile]{Dynamische Bindung}
    \begin{itemize}[<+(1)->]
        \item<1-> Attribute werden Java \emph{statisch} und Methoden \emph{dynamisch} gebunden.
        \item<1-> Doch was bedeuten diese Bindungsarten: \begin{description}[dynamisch]
            \item[dynamisch] Angenommen die Klasse \bjava{Mensch} hat die Methode \bjava{hallo()}, die einfach \say{Hallo ich bin ein Mensch} ausgibt.\pause{} \bjava{Student} überschreibt nun \bjava{hallo()}. Sie gibt nun \say{Hallo ich bin ein Student} aus.\pause{} Wir erhalten: \begin{plainjava}
!*\onslide<5->*!Mensch a = new Mensch(/*...*/);
!*\onslide<5->*!Student b = new Student(/*...*/);
!*\onslide<5->*!Mensch c = new Student(/*...*/);
!*\onslide<6->*!a.hallo(); // :yields: Hallo ich bin ein Mensch
!*\onslide<7->*!b.hallo(); // :yields: Hallo ich bin ein Student
!*\onslide<8->*!c.hallo(); // :yields: Hallo ich bin ein Student
            \end{plainjava}
        \end{description}
    \end{itemize}
\end{frame}


\subsection{Interfaces}
\begin{frame}{Wenn Vererbung nicht reicht}
    \begin{itemize}[<+(1)->]
        \widei
        \item Java erlaubt \emph{keine} Mehrfachvererbung.\pause{}\par \textcolor{gray}{Allerdings können Klassen mehrere \say{Eigenschaften} erfüllen.}
        \item So können wir über unsere Listen iterieren,\pause{} über Bäume aber auch.
        \item Eine Java Klasse kann (beliebig viele) Schnittstellen mittels \bjava{implements} implementieren.
        \item Ein solches \bjava{interface} fordert gewisse Methoden vom Implementor.
        \item Weiter kann ein \bjava{interface} seit Java \(8\) durch \bjava{default} Standardimplementationen für die Methoden liefern.
        \item Ein \bjava{interface} kann \emph{keine} Attribute definieren, diese werden automatisch zu Konstanten.
    \end{itemize}
\end{frame}

\begin{frame}[fragile]{Wenn Vererbung nicht reicht}
    \begin{itemize}[<+(1)->]
        \widei
        \item \label{mrk:int-iterable}So gibt es zum Beispiel das Java-Interface \bjava{Iterable} welches anzeigt, dass man über den Datentyp (mittels for-each) iterieren kann.
        \item Da \emph{Generics} kein Teil der Vorlesung sind, wurde das Interface \emph{Sortable} definiert:\pause{}
\begin{plainjava}
public interface Sortable {
    boolean le(Sortable o); // '<='
    boolean lt(Sortable o); // '<'
    boolean eq(Sortable o); // '=='
}
\end{plainjava}
        \item Dies zeigt auch, wie man Interface-Bezeichnet wie (abstrakte) Klassen als Typanforderung liefern kann. \textcolor{gray}{(Polymorphieeee)}
    \end{itemize}
\end{frame}

\begin{frame}[fragile,c]{Interface Beispiel}
\begin{plainjava}
public class Element implements Sortable /*, interfaceB, ... */ {
    int value;
    Element next;

    // Nur als Beispiel:
    public boolean le(Sortable o) {
        return this.value <= ((Element)o).value;
    }

    // ...
}
\end{plainjava}
\end{frame}

\begin{frame}{Weitere Kommentare}
    \begin{itemize}[<+(1)->]
        \widei
        \item Auch für ein Interface wird die \emph{is-a} Relation durch \bjava{instanceof} erfüllt.
        \item Ein Interface wie \bjava{Sortable} wird in Java durch \bjava{Comparable} mit nur einer einzelnen Funktion \bjava{compareTo()} gelöst.
        \item Interfaces sollten dann verwendet werdenn wenn es um eine Funktionalität geht (wie \bjava{Drawable}, \bjava{Moveable}, \bjava{Consumeable},~\ldots)
        \item Das Implementieren eines Interfaces kennzeichnet UML durch eine gestrichelte, das Erweitern einer Klasse durch eine durchgezogene Linie.
    \end{itemize}
\end{frame}

\subsection{Fehlerbehandlungen}
\begin{frame}{Arten von Fehlern}
    \begin{itemize}[<+(1)->]
        \widei
        \item Wir unterscheiden zwei Arten von Fehlern: \begin{description}[Schwere]
            \item[Schwere] solche Fehler sind kritisch und führen zu einem Programmabbruch.
            \item[Leichte] solche Fehler können (zur Laufzeit) korrigiert werden.
        \end{description}
        \item Wie schwer ein Fehler ist, hängt von der Situation ab.
        \item Die Fehlerbehandlung erfolgt meist durch die aufrufende Methode.
        \item Fehler sind in Java Objekte von Klasse,\pause{} die von \bjava{Exception} erben.
    \end{itemize}
\end{frame}

\begin{frame}[fragile]{Arten von Fehlern}
    \begin{itemize}[<+(1)->]
        \widei
        \item Eine \emph{explizite} Ausnahme können wir mittels \bjava{throw} werfen.
        \item Eine \emph{implizite} Ausnahme wird von der Java Virtual Machine geworfen \textcolor{gray}{(Division durch Null, Zugriff auf \bjava{null},~\ldots)}
        \item Wir werfen eine explizite Ausnahme (allgemein: ein \bjava{Throwable}): \pause{}
\begin{plainjava}
throw new IndexOutOfBoundsException();
\end{plainjava}
        \item Wird eine Ausnahme nicht behandelt, bricht das Programm ab.
        \item Java-Errors, wie ein \bjava{InternalError} lassen sich nicht sinnvoll behandeln!
    \end{itemize}
\end{frame}

\begin{frame}[fragile]{Behandeln von Fehlern}
    \begin{itemize}[<+(1)->]
        \widei
        \item Um einen Fehler zu behandeln, bietet Java das \bjava{try-catch}-Konstrukt
\begin{plainjava}
try {
    // Anweisungen, Methodenaufrufe, :ldots:
} catch(:lan:FehlerKlasseA:ran: :lan:Bezeichner:ran:) {
    // Behandle Fehler der Klasse :lan:FehlerKlasseA:ran:
} catch(:lan:FehlerKlasseB:ran: :lan:Bezeichner:ran:) {
    // Behandle Fehler der Klasse :lan:FehlerKlasseB:ran:
} finally {
    // Anweisungen die immer ausgeführt werden.
}
\end{plainjava}
        \item Das \bjava{finally} ist optional, ebenso kann nur ein \bjava{catch}-Block auftreten.
    \end{itemize}
\end{frame}

\begin{frame}[fragile]{Checked- \& Unchecked Exceptions}
    \begin{itemize}[<+(1)->]
        \widei
        \item Alle Exceptions erben in Java von \bjava{Exception}.
        \item Solche, die von \bjava{RuntimeException} erben sind \textit{unchecked}, die anderen sind \textit{checked}.
        \item \textit{Checked} Exceptions \textit{müssen} behandelt werden: \begin{itemize}
            \widei
            \item Entweder direkt durch ein \bjava{try-catch} (welches die Exception behandelt).
            \item Oder durch eine Weiterreichung per \bjava{throws} bei der Methoden-Deklaration:
\begin{plainjava}
public void foo() throws Exception/*, ExceptionB, ...*/ {
    throw new Exception();
}
\end{plainjava}
\end{itemize}
    \end{itemize}
\end{frame}

\begin{frame}[fragile]{Exceptions weiterwerfen}
    \begin{itemize}[<+(1)->]
        \widei
        \item Die anderen Blöcke eines \bjava{try}-\bjava{catch} sind nicht magisch. Auch in ihnen können Exceptions geworfen werden.
        \item So lassen sich übrigens auch \bjava{try}-\bjava{catch}-Konstrukte verschachteln.
        \item Rethrows können die Verantwortung auch weiterreichen:
\begin{plainjava}
!*\onslide<3->*!public void foo() throws Exception {
!*\onslide<4->*!    try {
!*\onslide<4->*!        throw new Exception();
!*\onslide<4->*!    } catch (Exception ex) {
!*\onslide<6->*!        // do stuff ..
!*\onslide<7->*!        throw ex; // oder sowas wie: throw new Exception(ex)
!*\onslide<4->*!    }
!*\onslide<3->*!}
\end{plainjava}
    \end{itemize}
\end{frame}


\begin{frame}[fragile]{Behandeln von Fehlern}
    \begin{itemize}[<+(1)->]
        \widei
        \item Bei mehreren \bjava{catch}-Blöcken wird von oben nach unten ein passender gesucht.\pause{} Erben die Fehler also voneinander (so wie alle von \bjava{Exception}),\pause{} sollten die \say{allgemeineren} weiter unten stehen.
        \item Beispiel:\pause{}
\begin{plainjava}
try {
    System.out.println(42 / 0);
} catch (ArithmeticException ex) {
    // Behandlung
    System.err.println("Division durch 0!");
    ex.printStackTrace();
}
\end{plainjava}
    \end{itemize}
\end{frame}

% #region Übungsaufgaben
\fullsubsection{Übungsaufgaben}
\ifull
\begin{frame}[c,fragile]{Übungsaufgabe}
    \Task{Fehlerfinden - Exception special}
    \begin{exercise}<2->[Fehler finden - Exceptions \Time{5}]
        \pause{}Erklären Sie alle Verletzungen der Java-Syntax (die Import sind korrekt). \bjava{IOException} und \solGet{keywordC}{FileNotFoundException} sind checked-Exceptions. \solGet{keywordC}{ArithmeticException} ist eine unchecked-Exception.\pause{}
\lstfs{9}
\begin{plainvoid}
public class A {
  public static void main(String[] args) throws FileNotFoundException {
    try { doStuff(); } catch (Exception ex) {}
    try { doElse(); } catch (RuntimeException ex) {}
    throw new IOException("ABC");
    try {
      throw new ArithmeticException();
    } catch(IOException ignored) {}
  }
  public static void doStuff() { throw new FileNotFoundException(); }
  public static void doElse() { throw RuntimeException; }
}
\end{plainvoid}
    \end{exercise}
\end{frame}

\begin{frame}[c]{Lösung}
    \begin{solve}<2->[Fehler finden - Exceptions]
        \begin{enumerate}[<+(1)->]
            \item \bjava{throw RuntimeException} liefert einen Symbolfehler: es muss ein Objekt geworfen werden.\pause\par \textcolor{gray}{Korrekt wäre \bjava{throw new RuntimeException();}}
            \item \bjava{catch(IOException ignored)} für das dritte try-catch ist illegal: Der try-Block wirft keine \bjava{IOException}.\pause\par \textcolor{gray}{Korrekt, wenn auch unsinnig, wäre das Abfangen der \bjava{ArithmeticException}.}
            \item Der Code nach der \say{ABC}-\solGet{keywordC}{IOException} ist illegal, die Anweisungen werden nie ausgeführt.\pause\par \textcolor{gray}{Dafür müsste der Fehler abgefangen werden.}
        \end{enumerate}
    \end{solve}
\end{frame}

\begin{frame}[c]{Lösung}
    \addtocounter{solve}{-1}
    \begin{solve}<2->[Fehler finden - Exceptions\hfill(Fortsetzung)]
        \begin{enumerate}[<+(1)->]
            \setcounter{enumi}{3}
            \item Die \solGet{keywordC}{IOException}, die in der \T{main} geworfen wird, muss markiert oder abgefangen werden.
            \item In \bjava{doStuff()} muss die \solGet{keywordC}{FileNotFoundException} entweder markiert werden (mittels \bjava{throws}) oder abgefangen (mittels \bjava{try}).\pause\par \textcolor{gray}{Dies gilt für alle Exceptions, bei denen es sich nicht um Runtime-Exceptions handelt.}
        \end{enumerate}
        Hinweis: Mit \bjava{throws} lassen sich auch mehr Exceptions markieren. Für Java ist \bjava[morekeywords={[3]{FileNotFoundException}}]{throws FileNotFoundException} kein Fehler.
    \end{solve}
\end{frame}

\begin{frame}[c]{Übungsaufgabe}
    \Task{Statische und Dynamische Bindung}
    \begin{exercise}<2->[Statische und Dynamische Bindung \Time{4}]
        Erklären Sie kurz den Unterschied zwischen statischer und dynamischer Bindung und wann Java was verwendet.
    \end{exercise}
\end{frame}

\begin{frame}[c]{Lösung}
    \begin{solve}<2->[Statische und Dynamische Bindung]
        Die \emph{statische} Bindung verwendet Java bei Attributen. Sie wird zur Kompilierzeit aufgelöst und sorgt dafür, dass in Subklassen überschriebene Variablen stets der verwendeten Klasse zugeordnet sind. In \bjava[morekeywords={[3]{A,B}}]{A a = new B();} würde sich \bjava{a.x} also auf das \T{x} der Klasse \bjava{A} beziehen. In \bjava[morekeywords={[3]{A,B}}]{B a = new B();} auf das der Klasse \bjava[morekeywords={[3]{A,B}}]{B}.\medskip\par
        \pause{}Die \emph{dynamische} Bindung verwendet Java bei Methoden. Sie wird zur Laufzeit aufgelöst und bedeutet, dass im Falle des Überschreibens von Methoden einer Subklasse die Klasse des Objekts entscheidet welche verwendet wird und nicht die der Variable die darauf zeigt.\pause\medskip\par
        Hier hilft auch Aufgabe~\ref{tsk:java-bindings-result}.
    \end{solve}
\end{frame}

\begin{frame}[c,fragile]{Übungsaufgabe}
    \Task{Bindungen Praktisch}
    \begin{exercise}<2->[Was liefert dieser Code? \Time{4}]\label{tsk:java-bindings-result}\lstfs{9}
\begin{plainjava}[morekeywords={[3]{A,B}}]
public class A {
    int a = 0;
    void x(int a) { this.a = 2 - a; }

    static class B extends A {
        int a = 1;
        void x(int a) { this.a = 2 * a; }
    }

    public static void main(String[] args) {
        A a = new A(); a.x(4);
        B b = new B(); b.x(4);
        A c = new B(); c.x(4);
        System.out.format("%d %d %d", a.a, b.a, c.a);
    }
}
\end{plainjava}
    \end{exercise}
\end{frame}

\begin{frame}[c]{Lösung}
    \begin{solve}<2->[Was liefert dieser Code?]
        Er liefert \say{\bjava{-2 8 0}}. Warum?
        \begin{description}[XX:]
            \item[-2] Es wird \bjava[morekeywords={[3]{A,B}}]{A::x} aufgerufen, damit ist \T{a}: \(2 - 4 = -2\).
            \item[8] Es wird \bjava[morekeywords={[3]{A,B}}]{B::x} aufgerufen, damit ist \T{a}: \(2 \cdot 4 = 8\).
            \item[0] Da Attribute statisch gebunden werden, verändert \bjava[morekeywords={[3]{A,B}}]{B::x} \emph{nicht} das \T{a} in \bjava[morekeywords={[3]{A,B}}]{A} und dieses bleibt damit beim initialen Wert: \(0\).
        \end{description}
    \end{solve}
\end{frame}


\begin{frame}[c, fragile]{Übungsaufgabe}
\Task{Vererbungshierarchien}
\begin{exercise}<2->[Vererbungshierarchien \Time{6}]
    {\small\onslide<3->{Konstruieren Sie eine \textit{abstrakte} Klasse \solGet{keywordC}{Entity}, die die gemeinsamen Eigenschaften folgender Klassen eint, die Sichtbarkeiten und Modifikatoren bestmöglich rekreiert, sich aber in einem anderen Ordner befindet. \onslide<4->{Schreiben Sie die Klassen so um, dass sie nun \solGet{keywordC}{Entity} verwenden.}}\par}
\begin{columns}[c,onlytextwidth]
\lstfs{6}%
\begin{column}{.475\linewidth}
\begin{plainjava}[morekeywords={[3]{Enemy,Companion}},belowskip=0pt,aboveskip=3pt]
!*\onslide<5->*!class Enemy {
!*\onslide<5->*!   private final long id;  protected int hp, ap, x, y;
!*\onslide<5->*!   public Enemy(long id) { this.id = id; }
!*\onslide<5->*!   public void move(int x, int y) {
!*\onslide<5->*!      this.x = x; this.y = y;
!*\onslide<5->*!   }
!*\onslide<5->*!   Enemy whoIsEvil() {
!*\onslide<5->*!      System.out.println("I aaaam ("+ id +")");
!*\onslide<5->*!      return this;
!*\onslide<5->*!   }
!*\onslide<5->*!}
!*\onslide<6->*!class Companion {
!*\onslide<6->*!   private final long id;  protected int x, y;
!*\onslide<6->*!   public Companion(long id) { this.id = id; }
!*\onslide<6->*!   public void move(int x, int y) {
!*\onslide<6->*!      this.y = y; this.x = x;
!*\onslide<6->*!   }
!*\onslide<6->*!}
\end{plainjava}
\end{column}
\begin{column}{.4\linewidth}
\begin{plainjava}[morekeywords={[3]{NPC}},belowskip=0pt,aboveskip=3pt]
!*\onslide<7->*!class NPC {
!*\onslide<7->*!   private final long id;  protected int x, y;
!*\onslide<7->*!   public final String visibleName;
!*\onslide<7->*!   public NPC(String name, long id) {
!*\onslide<7->*!      this.visibleName = name; this.id = id;
!*\onslide<7->*!   }
!*\onslide<7->*!
!*\onslide<7->*!   public void move(int x, int y) {
!*\onslide<7->*!      this.x = x; this.y = y;
!*\onslide<7->*!      annoyingInterruption();
!*\onslide<7->*!   }
!*\onslide<7->*!
!*\onslide<7->*!   void annoyingInterruption() {
!*\onslide<7->*!      System.out.println("Movin' ("+ pos +")");
!*\onslide<7->*!   }
!*\onslide<7->*!}
\end{plainjava}
\end{column}
\end{columns}
    \end{exercise}
\end{frame}

\begin{frame}[c,fragile]{Lösung}
\begin{solve}<2->[Vererbungshierarchien]
\lstfs{7}%
\begin{plainjava}[morekeywords={[3]{Entity}},belowskip=0pt,aboveskip=5pt]
!*\onslide<3->*!abstract class Entity {
!*\onslide<3->*!   protected final long id;
!*\onslide<3->*!   protected int x, y; // oder private mit Getter
!*\onslide<3->*!   public Entity(long id) { this.id = id; }
!*\onslide<3->*!   public void move(int x, int y) {
!*\onslide<3->*!      this.x = x; this.y = y;
!*\onslide<3->*!   }
!*\onslide<3->*!}
\end{plainjava}
\begin{columns}[c, onlytextwidth]
\begin{column}{.465\linewidth}
\begin{plainjava}[morekeywords={[3]{Enemy,Companion,Entity}},belowskip=0pt,aboveskip=0pt]
!*\onslide<4->*!class Enemy extends Entity {
!*\onslide<4->*!   protected int hp, ap;
!*\onslide<4->*!   public Enemy(long id) { super(id); }
!*\onslide<4->*!
!*\onslide<4->*!   Enemy whoIsEvil() {
!*\onslide<4->*!      System.out.println("I aaaam ("+ id +")");
!*\onslide<4->*!      return this;
!*\onslide<4->*!   }
!*\onslide<4->*!}

!*\onslide<5->*!class Companion extends Entity {
!*\onslide<5->*!  public Companion(long id) { super(id); }
!*\onslide<5->*!}
\end{plainjava}
\end{column}
\begin{column}{.465\linewidth}
\begin{plainjava}[morekeywords={[3]{NPC,Entity}},belowskip=0pt,aboveskip=0pt]
!*\onslide<6->*!class NPC extends Entity {
!*\onslide<6->*!   public final String visibleName;
!*\onslide<6->*!   public NPC(String name, long id) {
!*\onslide<6->*!      super(id); this.visibleName = name;
!*\onslide<6->*!   }
!*\onslide<6->*!   public void move(int x, int y) {
!*\onslide<6->*!      super.move(x, y);
!*\onslide<6->*!      annoyingInterruption();
!*\onslide<6->*!   }
!*\onslide<6->*!   void annoyingInterruption() {
!*\onslide<6->*!      System.out.println("Movin' ("+ pos +")");
!*\onslide<6->*!   }
!*\onslide<6->*!}
\end{plainjava}
\end{column}
\end{columns}
\end{solve}
\end{frame}

\begin{frame}[c,fragile]{Übungsaufgabe}
    \Task{Suche syntaktischer Fehler, V}
    \begin{exercise}<2->[Fehler finden, V \Time{4}]
        \pause{}Finden und korrigieren Sie alle (syntaktischen und semantischen) Fehler:\pause{}
{\footnotesize
        \begin{plainvoid}
public class A {
    private static final class B extends A {
        public B() {
            super(this);
        }
    }

    public A(A other){}
    public abstract class C extends A.B {
        protected abstract boolean _$get$(String);
    }
}
        \end{plainvoid}
}
    \end{exercise}
\end{frame}

\begin{frame}[c,fragile]{Lösung}
    \begin{solve}<2->[Fehler finden, V]
        \pause{}\footnotesize\begin{plainjava}[morekeywords={[3]{A,B,C}}]
public class A {
    private static class B extends A {
        public B() {
            super(null);
        }
    }

    public A(A other){}
    public abstract class C extends A.B {
        protected abstract boolean _$get$(String a);
    }
}
        \end{plainjava}
    \end{solve}
\end{frame}

\begin{frame}[c]{Lösung}
    \addtocounter{solve}{-1}
    \begin{solve}<1->[Fehler finden, V\hfill{}(Fortsetzung)]
        \begin{enumerate}
            \item Das \bjava{final} ist an sich nicht falsch, dann kann aber \solGet{keywordC}{C} nicht mehr von \solGet{keywordC}{B} erben.
            \item Der Bezeichner \bjava{this} darf nicht innerhalb eines Konstruktoraufrufs verwendet werden.
            \item In der Methode \bjava{_$get$} genügt nicht die Signatur,\pause{} Java erwartet (nicht-bindende) Variablenbezeichner.
        \end{enumerate}
        \pause{}\textit{Hinweis:}\pause{} Die Dollarzeichen sind kein Fehler,\pause{} aber es ist Konvention sie nicht zu verwenden (Java verwendet sie intern für zum Beispiel generierten Code).
    \end{solve}
\end{frame}

\begin{frame}[c,fragile]{Übungsaufgabe}
    \Task{Suche syntaktischer Fehler, VI}
    \begin{exercise}<2->[Fehler finden, VI \Time{4}]
        \pause{}Finden und erklären Sie alle (Kompilier-)Fehler:\pause{}
{\footnotesize
        \begin{plainvoid}
public class MyL1ttl3Cl4ss extends MyL1ttl3Cl4ss.A {
    private class A {
        public int x;
        A(int x) { this.x = x; }
    }

    public void MyEp1cM3th0d {
        System.out::println(5 = = 2 + + x);
    }
}
        \end{plainvoid}
}
    \end{exercise}
\end{frame}

\begin{frame}[c,fragile]{Lösung}
    \begin{solve}<2->[Fehler finden, VI]
        \pause{}\footnotesize\begin{plainjava}[add to literate={MyL1ttl3Cl4ss}{{\solGet{keywordC}{MyL1ttl3Cl4ss}}}{13},morekeywords={[3]{A}}]
public class MyL1ttl3Cl4ss extends MyL1ttl3Cl4ss.A {
    private class A {
        public int x;
        A(int x) { this.x = x; }
    }

    public void MyEp1cM3th0d {
        System.out::println(5 = = 2 + + x);
    }
}
        \end{plainjava}
    \end{solve}
\end{frame}

\begin{frame}[c]{Lösung}
    \addtocounter{solve}{-1}
    \begin{solve}<1->[Fehler finden, VI\hfill{}(Fortsetzung)]
        \begin{enumerate}
            \item \bjava[add to literate={MyL1ttl3Cl4ss}{{\solGet{keywordC}{MyL1ttl3Cl4ss}}}{13},morekeywords={[3]{A}}]{MyL1ttl3Cl4ss extends MyL1ttl3Cl4ss.A} ist eine Variante des zyklischen Vererbungs-Problems und in Java nicht gestattet.
            \item Die Methode \bjava{MyEp1cM3th0d} benötigt die Klammern zur Kennzeichnung der Parameteranzahl.
            \item Der Ausdruck \bjava{System.out::println} ist eine Methoden-Referenz. Für den Aufruf benötigen wir \bjava{System.out.println}.
            \item \bjava[showspaces]{= =} ist kein gültiger Java-Ausdruck! Nur \bjava{==} oder \bjava{=}.
        \end{enumerate}
        \pause{}Dabei ist \bjava{2 + + x} kein Problem! es wird als die Addition von \bjava{2} zu \bjava{+x} (analog zu \bjava{+1}) verstanden.\smallskip\pause\par
        Übrigens: \bjava{MyL1ttl3Cl4ss.A} statisch zu machen, hilft nicht. Java Liebe!% hat natürlich Gründe :D
    \end{solve}
\end{frame}

\fi
% #endregion